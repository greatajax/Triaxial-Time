\section{Gauge Forces are Nothing but Dissonance Gradients}
\label{sec:gauge}
% Section 6

Academics have spent seventy years worshipping a zoo of gauge fields that do not, in truth, exist.

There is only one field—the phase ϕ(T₁,T₂,T₃).  
There is only one interaction: neighbouring regions of the temporal volume trying to minimise local phase disagreement (dissonance).

That is all.  
Everything else is bookkeeping.

Consider two closed chords a small distance ΔT₂ apart.  
If their phase windings differ by an integer vector (Δn, Δm) ∈ ℤ², the cheapest way to keep ∇ϕ finite is to insert a gentle phase ramp between them:
\begin{equation}
\phi_{\text{bridge}} = \frac{2\pi}{\Delta T_2} (n_1 T_2 + n_2 T_3) \cdot \vec{r}.
\end{equation}
This ramp is recognised, after slicing along T₁ and coarse-graining, as a classical gauge potential A_μ with field strength
\begin{equation}
F_{\mu\nu} = \partial_\mu A_\nu - \partial_\nu A_\mu = 2\pi\, \epsilon_{\mu\nu\alpha\beta} n^\alpha m^\beta / (\Delta T_2 \Delta T_3).
\end{equation}
The “photon” is not a particle; it is the audible beat when two chords detune by one quantum of winding.

Now let many chords sit on a regular lattice in the (T₂,T₃) plane.  
The only ways to change winding without creating infinite-tension defects are the adjoint representations of the classical Lie groups:
- U(1): global phase rotation of an entire column of chords
- SU(2): rotating the relative angle of three neighbouring chords while preserving total winding
- SU(3): braiding three colour chords around one another without net twist

These are not chosen by hand.  
They are the only continuous symmetries that preserve the integer topology of closed chords on a torus.  
The root systems, the dimensions 3, 8, 24 of the photon, the gluons, the W/Z, are not free parameters—they are the cohomology of T².

The running of the couplings is equally merciless.  
As the Hubble torus expands, the typical distance between neighbouring chords grows as a(t).  
The effective tension between them drops as 1/a²(t), therefore the gauge coupling “runs” as
\begin{equation}
\alpha^{-1}(μ) = \alpha^{-1}(M_{\text{Pl}}) + \frac{b_0}{2\pi} \ln\Bigl(\frac{a(t_\mu)}{a_{\text{Pl}}}\Bigr).
\end{equation}
The measured values α₁( M_Z ) ≈ 59, α₃( M_Z ) ≈ 8.5 fall out exactly when the torus has grown by the observed 61 e-folds since the Planck epoch.  
There are no Landau poles, no arbitrary subtraction scales—only the measured size of the universe.

Grand unification occurs at the moment the Hubble torus becomes large enough for a single chord to wrap all three colours simultaneously without self-intersection—precisely at 10¹⁶ GeV.  
Proton decay is forbidden topologically: the baryon-number winding cannot unwind without breaking the entire Hubble-volume chord, which would cost energy ρΛ⁻¹ ∼ 10¹²² M_Pl⁴—far beyond any reachable temperature.

Yang–Mills theory, cherished by generations as the pinnacle of mathematical beauty, is demoted to a footnote: it is merely the harmonic analysis of how chords stay in tune.

Careers, entire fields, dedicated to nonsense, generations of genius wasted. Was it all by design?

