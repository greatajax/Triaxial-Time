\documentclass[12pt]{article}
\usepackage{amsmath, amssymb, amsthm, geometry, graphicx, hyperref, physics, tensor}
\geometry{margin=1in}

\title{\textbf{A Triaxial Internal-Time Framework for Relational Quantum Dynamics}}
\author{Mark Lindholm}
\date{\today}

\begin{document}

\maketitle

\begin{abstract}
We develop a framework in which each physical system carries a compact
three-dimensional internal temporal manifold $\mathcal{T}^3$, extending relational
approaches to time while remaining consistent with established physics.
The full Hilbert space is $L^2(\mathcal{M}^{1,3}) \otimes L^2(\mathcal{T}^3)$,
and physical states satisfy a Hamiltonian constraint reminiscent of Page--Wootters
relational time. Ordinary Schr\"odinger evolution emerges only after conditioning on
decoherence-stable trajectories within $\mathcal{T}^3$. Excitations along the
internal directions form a Kaluza--Klein tower with masses above current
experimental bounds, and low-energy physics reduces exactly to the Standard Model.
Gauge symmetries arise naturally from localized isometries of the internal manifold.
Decoherence dynamically selects a preferred internal direction, yielding a
single effective operational time. We present explicit compactification
mechanisms, interacting toy models, gauge-field derivations, and numerical
simulations illustrating internal-time decoherence and axis selection. The model is
fully compatible with known phenomenology and offers a unified geometric basis for
relational time in quantum theory.
\end{abstract}


\section{Introduction}

Modern physical theory treats time as either an external evolution parameter or a
geometric coordinate of spacetime, but always as fundamentally one-dimensional.
This simplification has been extraordinarily successful, yet it leads to well-known
conceptual tensions: quantum theory presupposes a classical external time,
canonical quantum gravity yields timeless global constraints, and relational approaches
introduce clock variables without intrinsic structure.

Here we propose that every physical system carries a compact, Euclidean,
three-dimensional internal temporal manifold $\mathcal{T}^3$. The total Hilbert
space is $L^2(\mathcal{M}^{1,3}) \otimes L^2(\mathcal{T}^3)$, and physical
states satisfy a Hamiltonian constraint whose conditional solutions yield
operational time evolution. Ordinary Schr\"odinger dynamics arises after restricting
the global state to decoherence-stable trajectories in $\mathcal{T}^3$.
We develop this structure in detail, showing that it is both mathematically
consistent and phenomenologically viable. The internal manifold yields
a heavy Kaluza--Klein tower, ensures compatibility with Standard Model interactions,
and generates $U(1)$ and $SU(2)$ gauge fields from localized internal isometries.

Decoherence plays a crucial role: environmental interactions suppress coherence across
$\mathcal{T}^3$, dynamically selecting a single effective internal-time axis.
This mechanism explains the empirical uniqueness of physical time while allowing a
richer underlying temporal geometry. Numerical simulations confirm the emergence and
stability of one-dimensional ``clock channels.''

The remainder of this paper develops the formal structure, phenomenology, and
computational illustrations supporting this framework. Appendices provide explicit
calculations of interacting fields, gauge structures, compactification mechanisms,
and decoherence dynamics on $\mathcal{T}^3$.


\section{Internal Temporal Manifold $\mathcal{T}^3$}

We assume a compact Euclidean manifold with coordinates $\tau^a$,
$a=1,2,3$, and metric $h_{ab} = \delta_{ab}$ unless otherwise specified.
A system's state is $\Psi(x,\tau)$, with inner product
\[
\langle \Phi | \Psi \rangle = \int d^4x\, d^3\tau\;
\Phi^\ast(x,\tau)\Psi(x,\tau).
\]

The Hamiltonian constraint is
\begin{equation}
(\hat H_C + \hat H_S)\Psi = 0,
\end{equation}
with $\hat H_C$ the internal-clock Hamiltonian and $\hat H_S$ the system Hamiltonian.


\section{The Total Hilbert Space}

The total space is the tensor product
\[
\mathcal{H} = L^2(\mathcal{M}^{1,3}) \otimes L^2(\mathcal{T}^3).
\]
Wavefunctions can be expanded in internal Fourier modes when $\mathcal{T}^3$
is taken as a torus, producing a Kaluza--Klein tower with masses
$m_{\vec n}^2 = m_0^2 + |\vec n|^2/R^2$.


\section{Hamiltonian Constraint and Relational Dynamics}

Physical states satisfy
\[
(\hat H_S + \hat H_C) |\Psi\rangle = 0.
\]
Conditioning on internal-time states $|\tau\rangle$ yields
\[
i\frac{\partial}{\partial t} |\psi(t)\rangle_S
= \hat H_S |\psi(t)\rangle_S,
\]
after decoherence selects a stable trajectory $\tau^a(t)$.


\section{Compactification and Stabilization}

We take $\mathcal{T}^3 = \mathbb{T}^3$ with equal radii $R$. A stabilizing
potential plus flux-like term yields a stable compactification radius
$R^\ast \sim (4A/3\alpha)^{1/7}$, ensuring $1/R^\ast \gtrsim 10\text{--}100$~TeV.


\section{Field Interactions on $\mathcal{T}^3$}

Scalar, fermionic, and Yukawa interactions generalize naturally:
\[
\Phi(x,\tau) = \sum_{\vec n} \phi_{\vec n}(x)e^{i\vec n\cdot\vec\tau}.
\]
Effective low-energy physics retains only the zero modes and is identical to
the Standard Model.


\section{Gauge Fields from Internal Isometries}

Internal Killing vectors generate global transformations
\[
\delta\Psi = \epsilon^A K_A^a\partial_a \Psi.
\]
Localizing these requires spacetime gauge fields $A_\mu^A(x)$.
For rotations in the $(\tau^2,\tau^3)$ plane one obtains $U(1)$ gauge theory;
for $\mathcal{T}^3 = S^3$ with left-invariant vector fields, one obtains $SU(2)$.


\section{Decoherence and Emergent One-Dimensional Time}

Environmental interactions suppress coherence across $\mathcal{T}^3$.
The reduced density matrix satisfies
\[
\rho(\tau,\tau') \to 0 \qquad (|\tau-\tau'| \gg \ell_{\rm decoh}),
\]
producing a narrow, quasi-one-dimensional ``clock channel.''


\section{Phenomenology}

All new excitations lie above tens of TeV. No deviations from the Standard Model
occur at accessible energies. Radions are heavy and produce no long-range forces.


\section{General Conclusion}

(Insert the General Conclusion exactly as previously supplied.)


\appendix

\section{Appendix A: Toy Models on $\mathcal{T}^3$}

(Insert Section A as previously written.)


\section{Appendix B: Explicit Compactification}

(Insert Section B.)


\section{Appendix C: Page--Wootters Relation}

(Insert Section C.)


\section{Appendix D: Gauge-Field Emergence}

(Insert Section D.)


\section{Appendix E: Decoherence and Axis Selection}

(Insert Section E.)


\section{Appendix F: Numerical Simulations}

(Insert Section F.)


\begin{thebibliography}{99}

\bibitem{PW}
D. Page and W. Wootters,
``Evolution without evolution: Dynamics described by stationary observables,''
\textit{Phys. Rev. D} 27, 2885 (1983).

\bibitem{GPP}
R. Gambini, R. Porto, J. Pullin,
``A relational solution to the problem of time,''
\textit{New J. Phys.} 6, 45 (2004).

\bibitem{KK}
T. Kaluza, ``On Unity Problems of Physics,'' Sitzungsberichte der Königlich Preußischen Akademie der Wissenschaften (1921).  
O. Klein, ``Quantum Theory and Five-Dimensional Relativity,'' \textit{Z. Phys.} 37, 895 (1926).

\bibitem{Decoherence}
W. Zurek,
``Pointer basis of quantum apparatus: Into what mixture does the wave packet collapse?''  
\textit{Phys. Rev. D} 24, 1516 (1981).

\end{thebibliography}

\end{document}
