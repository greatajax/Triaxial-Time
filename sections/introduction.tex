%===================================================
\section{Problem Statement and Aim}
%===================================================

Our starting point is the mismatch between the one-dimensional time parameter used in physics and the lived, thick experience of duration. That mismatch is treated here as a physical gap: a sign that the formal picture is missing the machinery that actually stitches events together. The aim is to show that a three-dimensional temporal manifold, unburdened by any background space, can generate all familiar physics while finally explaining why memory, possibility, and intention feel real and behave consistently.

This paper proceeds with a single, uncompromising voice. Earlier drafts and dialogues have been merged into a coherent narrative, stripping away fictional interlocutors and self-referential asides. The focus is on what the framework claims, where it holds together, and where it still needs proof.

\subsection{Core claim}
The essential proposal is that every event carries an internal temporal volume \(\bm{T} = (T_1, T_2, T_3)\). There is no primitive spatial arena. Apparent spatial order is a bookkeeping device that summarizes how adjacent temporal volumes braid and rebalance. Physical law remains intact when expressed on the observed one-dimensional projection, but new explanatory power appears once the hidden axes and their conservation constraints are tracked explicitly.

\subsection{Guiding principles}
Three principles shape the construction:
\begin{itemize}
    \item \textbf{Conservation of empirical success:} All tested relativistic and quantum predictions must survive. Any new dynamics must reduce to the standard model on the observed projection.
    \item \textbf{Zero-net-energy:} Flow along the three axes must sum to zero at each event, preventing hidden perpetual motion and anchoring conservation of energy-momentum after projection.
    \item \textbf{Causal clarity:} The extended manifold must admit a global ordering compatible with the effective physical time so that no new paradoxes are introduced.
    \item \textbf{Minimal new objects:} Beyond the extra temporal dimensions, no speculative fields or exotic matter are assumed; explanatory gains come from geometry and projection alone.
\end{itemize}

\subsection{What is preserved and what changes}
Metric relations, gauge structure, and Hilbert-space dynamics remain as projected shadows of triaxial geometry. What changes is the interpretation of superposition, decoherence, inertia, and agency: they are rephrased as specific shapes and flows inside the \(T_1\)--\(T_3\) volume that respect the zero-net-energy balance.

\subsection{Roadmap}
The remainder of the paper is structured to replace the tangle of appendices with a clear progression:
\begin{enumerate}
    \item \textbf{Framework overview} (Section~\ref{sec:framework}): defines the purely temporal arena, the meaning of the three axes, and how apparent spatial order emerges.
    \item \textbf{Dynamics and projections} (Sections~\ref{sec:geometry}--\ref{sec:dynamics}): develops the geometry of temporal braiding, the zero-net-energy balance, and the induced field behavior.
    \item \textbf{Phenomenology} (Section~\ref{sec:phenomenology}): translates the geometry into recognizable physical and cognitive effects and identifies where existing appendices supply detail.
    \item \textbf{Open problems} (Section~\ref{sec:outlook}): lists the gaps that remain after the cleanup and directs future work.
\end{enumerate}
