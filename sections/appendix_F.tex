\section{F. Mathematical Appendix and Toy Models}

This appendix sketches simple toy models that illustrate how the general
ideas in the main text can be implemented in explicit equations. The goal is
not completeness but concreteness.

\subsection{Toy model: linearized membrane in flat temporal space}

Consider a flat temporal space \(\mathbb{R}^3_{\bm{T}}\) with coordinates
\((T_1,T_2,T_3)\) and a membrane defined by
\begin{equation}
    T_2 = \epsilon f(\mathbf{x}), \quad T_3 = 0,
\end{equation}
with small parameter \(\epsilon\) and smooth function \(f\). The base time
coordinate is identified with \(T_1\).

To first order in \(\epsilon\), the induced metric on the membrane acquires
corrections of order \(\epsilon^2\), while the extrinsic curvature has
components proportional to \(\epsilon\,\nabla_i \nabla_j f\). For a scalar
field localized around the membrane, one finds effective corrections to its
mass and kinetic terms controlled by these curvature components.

This simple model shows how spatial variations in the embedding can generate
effective potentials at low energy, providing a crude analog of gravitational
backgrounds.

\subsection{Toy model: modal axis as continuous label}

To connect with quantum-like structure, consider a scalar field
\(\phi(\mathbf{x}, T_1, T_3)\) with no explicit dependence on \(T_2\). The
Lagrangian is
\begin{equation}
    \mathcal{L} = \frac{1}{2}\left[
        (\partial_{T_1}\phi)^2 - (\nabla\phi)^2
        - c_m^2 (\partial_{T_3}\phi)^2 - m^2 \phi^2
    \right].
\end{equation}
Fourier-transforming along \(T_3\),
\begin{equation}
    \phi(\mathbf{x}, T_1, T_3) =
    \int \frac{dk_3}{2\pi}\,e^{ik_3 T_3}\,\tilde{\phi}(\mathbf{x}, T_1; k_3),
\end{equation}
we obtain a tower of modes \(\tilde{\phi}(\mathbf{x}, T_1; k_3)\) with
effective masses
\begin{equation}
    m_\text{eff}^2 = m^2 + c_m^2 k_3^2.
\end{equation}
The distribution over \(k_3\) plays the role of a wavefunction spread over
modal configurations. A process that localizes the field in \(T_3\) (e.g. via
interaction with a large environment extended in \(T_2\)) corresponds to
projection onto a narrower band of \(k_3\), echoing quantum collapse.

\subsection{Constraints on signatures}

A central mathematical concern is avoiding multiple independent time-like
directions in the effective four-dimensional physics. In this framework, this
is accomplished by:

\begin{itemize}
    \item Assigning a positive-definite metric to the temporal sector
    \(\mathbb{R}^3_{\bm{T}}\) at the level of the embedding space.
    \item Defining the physical time direction \(U^a\) as a distinguished
    tangent vector on the membrane, with only that direction contributing a
    negative signature to the induced metric.
\end{itemize}

This effectively realizes the extra temporal axes as internal parameters
rather than additional physical time directions, despite their ontological
temporal interpretation.

\subsection{Open technical problems}

Several mathematical challenges remain:

\begin{itemize}
    \item Constructing explicit, globally well-behaved embeddings \(\Sigma
    \subset \mathcal{E}\) that reproduce known cosmological solutions.
    \item Developing a full field-theoretic treatment of gauge fields as
    connections associated with isometries in the temporal sector.
    \item Formulating a rigorous Hilbert-space structure tied to distributions
    along \(T_3\), with a clear derivation of the Born rule.
\end{itemize}

These tasks are left for future work.
