\documentclass[12pt]{article}

\usepackage[margin=1in]{geometry}
\usepackage{amsmath, amssymb}
\usepackage{bm}
\usepackage{graphicx}
\usepackage{hyperref}
\usepackage{authblk}

\title{Three-Dimensional Time:\\
A Causal and Ontological Reconstruction of Physical Law}

\author{Mark Lindholm}
\affil{\small Independent Researcher}
\date{\today}

\begin{document}
\maketitle

\begin{abstract}
We explore a model in which the primitive structure underlying physics is not a
single time parameter but a three-dimensional temporal manifold, with axes
denoted \(T_1, T_2, T_3\). The observed \(3+1\)-dimensional spacetime of
standard physics is treated as a membrane embedded in this higher-order
temporal structure. Local physical processes appear as projections of
higher-dimensional temporal modes onto the membrane. We show that this
construction can reproduce the kinematics of special relativity, accommodate a
general-relativistic limit in the presence of curvature and shear of the
membrane, and admit a natural place for probabilistic quantum dynamics and
mental/intentional phenomena as extended structures along \(T_2\) and \(T_3\).
We emphasize causal consistency, mathematical legality, and empirical
non-contradiction, and we identify concrete phenomenological consequences that
could, in principle, distinguish a triaxial temporal ontology from standard
monotemporal frameworks.
\end{abstract}

\tableofcontents

%===================================================
\section{Problem Statement and Aim}
%===================================================

Our starting point is the mismatch between the one-dimensional time parameter used in physics and the lived, thick experience of duration. That mismatch is treated here as a physical gap: a sign that the formal picture is missing the machinery that actually stitches events together. The aim is to show that a three-dimensional temporal manifold, unburdened by any background space, can generate all familiar physics while finally explaining why memory, possibility, and intention feel real and behave consistently.

This paper proceeds with a single, uncompromising voice. Earlier drafts and dialogues have been merged into a coherent narrative, stripping away fictional interlocutors and self-referential asides. The focus is on what the framework claims, where it holds together, and where it still needs proof.

\subsection{Core claim}
The essential proposal is that every event carries an internal temporal volume \(\bm{T} = (T_1, T_2, T_3)\). There is no primitive spatial arena. Apparent spatial order is a bookkeeping device that summarizes how adjacent temporal volumes braid and rebalance. Physical law remains intact when expressed on the observed one-dimensional projection, but new explanatory power appears once the hidden axes and their conservation constraints are tracked explicitly.

\subsection{Guiding principles}
Three principles shape the construction:
\begin{itemize}
    \item \textbf{Conservation of empirical success:} All tested relativistic and quantum predictions must survive. Any new dynamics must reduce to the standard model on the observed projection.
    \item \textbf{Zero-net-energy:} Flow along the three axes must sum to zero at each event, preventing hidden perpetual motion and anchoring conservation of energy-momentum after projection.
    \item \textbf{Causal clarity:} The extended manifold must admit a global ordering compatible with the effective physical time so that no new paradoxes are introduced.
    \item \textbf{Minimal new objects:} Beyond the extra temporal dimensions, no speculative fields or exotic matter are assumed; explanatory gains come from geometry and projection alone.
\end{itemize}

\subsection{What is preserved and what changes}
Metric relations, gauge structure, and Hilbert-space dynamics remain as projected shadows of triaxial geometry. What changes is the interpretation of superposition, decoherence, inertia, and agency: they are rephrased as specific shapes and flows inside the \(T_1\)--\(T_3\) volume that respect the zero-net-energy balance.

\subsection{Roadmap}
The remainder of the paper is structured to replace the tangle of appendices with a clear progression:
\begin{enumerate}
    \item \textbf{Framework overview} (Section~\ref{sec:framework}): defines the purely temporal arena, the meaning of the three axes, and how apparent spatial order emerges.
    \item \textbf{Dynamics and projections} (Sections~\ref{sec:geometry}--\ref{sec:dynamics}): develops the geometry of temporal braiding, the zero-net-energy balance, and the induced field behavior.
    \item \textbf{Phenomenology} (Section~\ref{sec:phenomenology}): translates the geometry into recognizable physical and cognitive effects and identifies where existing appendices supply detail.
    \item \textbf{Open problems} (Section~\ref{sec:outlook}): lists the gaps that remain after the cleanup and directs future work.
\end{enumerate}

\section{A. Geometry of Triaxial Time}

\subsection{The extended temporal manifold}

Let \(M\) denote the usual \(3+1\)-dimensional spacetime manifold of standard
relativistic physics, with local coordinates \((t, \mathbf{x}) = (t, x^i)\),
\(i = 1,2,3\), and with a Lorentzian metric \(g_{\mu\nu}\).

In the present framework we replace the single time coordinate \(t\) with a
three-component temporal vector
\begin{equation}
    \bm{T} = (T_1, T_2, T_3) \in \mathbb{R}^3.
\end{equation}
The extended kinematic arena is then
\begin{equation}
    \mathcal{E} = \mathbb{R}^3_{\mathbf{x}} \times \mathbb{R}^3_{\bm{T}},
\end{equation}
equipped with a metric that is locally the product of a spatial metric and a
positive-definite metric on the temporal sector:
\begin{equation}
    ds^2 = h_{ij}\,dx^i dx^j - c^2\,\delta_{ab}\,dT^a dT^b,
\end{equation}
where \(h_{ij}\) may be taken to be Euclidean at small scales and \(a,b\)
index the temporal coordinates.

We deliberately \emph{do not} identify any of the \(T_a\) individually with
the physical time coordinate of ordinary relativity. Instead, the physical
time direction emerges as a \emph{derived} one-dimensional subspace in the
three-dimensional temporal manifold.

\subsection{The spacetime membrane}

An observer does not have direct access to all three temporal axes. Rather,
physical processes are confined to a four-dimensional submanifold
\(\Sigma \subset \mathcal{E}\) of codimension two in the temporal sector. At
each point \((\mathbf{x}, \bm{T})\in \Sigma\), there is:

\begin{itemize}
    \item A \emph{tangent direction} in the temporal space, identified with
    the effective physical time direction \(u^a(\mathbf{x}, \bm{T})\).
    \item Two \emph{normal directions} in the temporal space, which we
    interpret as directions along which the underlying mode-structure can
    extend without being directly available to the observer.
\end{itemize}

Locally, one can choose coordinates so that the membrane is specified by
constraints of the form
\begin{equation}
    \Phi_\alpha(\mathbf{x}, \bm{T}) = 0, \quad \alpha = 1,2,
\end{equation}
with the effective physical time coordinate given by a parameter \(\tau\)
along the integral curves of the vector field
\begin{equation}
    U = U^a \frac{\partial}{\partial T^a}
\end{equation}
tangent to \(\Sigma\) in the temporal sector.

In this language, standard spacetime is the image of \(\Sigma\) under the
projection
\begin{equation}
    \pi: \Sigma \to M, \qquad
    \pi(\mathbf{x}, \bm{T}) = (t(\bm{T}), \mathbf{x}),
\end{equation}
for some function \(t: \mathbb{R}^3_{\bm{T}} \to \mathbb{R}\) that extracts
the scalar ``clock time'' from the triaxial temporal coordinates.

\subsection{Interpretation of the axes}

While the theory does not require a unique psychological interpretation of
the three axes, it is useful to have a working picture:

\begin{itemize}
    \item \(T_1\): the direction approximately aligned with the parameter
    measured by physical clocks; it represents the advance of events.
    \item \(T_2\): the direction in which persistent structures---records,
    correlations, and memories---extend, giving them effective temporal
    thickness.
    \item \(T_3\): the direction along which near-by alternative evolutions
    and modal branches are parameterized; quantum amplitudes can be thought of
    as distributions across this axis.
\end{itemize}

The effective physical time direction \(U^a\) at a point is, in general,
a linear combination of these axes. Its orientation reflects the relative
weight that the local mode-structure places on persistence versus
exploration. A fully deterministic classical trajectory corresponds to a mode
that is sharply localized in \(T_3\) and extended primarily along a
\(T_1\)-\(T_2\) plane, while a quantum-coherent process has a nontrivial
spread along \(T_3\).

\subsection{Shear and curvature in temporal space}

The geometry of the membrane is controlled by how the physical time direction
\(U^a\) and the normal directions vary with position \(\mathbf{x}\) and
temporal location \(\bm{T}\). The key quantities are:

\begin{itemize}
    \item \textbf{Temporal curvature:} variations of the normal vectors as
    one moves along spatial directions, analogous to curvature of an embedded
    surface.
    \item \textbf{Temporal shear:} gradients in the orientation of \(U^a\)
    across space, which we will later see can be associated with gravitational
    and gauge-like fields.
\end{itemize}

Informally, gravity arises from the way the membrane bends and shears in the
three-dimensional temporal space. A freely falling particle follows a
geodesic of the induced metric on \(\Sigma\); but that geodesic, when viewed
in the embedding space \(\mathcal{E}\), may correspond to a nontrivial motion
through the temporal manifold.

The rest of the paper develops these geometric ideas into a dynamical
framework consistent with known physics.

\section{B. Dynamics, Modes, and Shear}
\label{sec:dynamics}

\subsection{Field modes as temporal wavepackets}

Let \(\phi(\bm{T})\) denote a generic field defined on the tri-temporal manifold. A physical excitation is a localized braid with amplitude concentrated near a path \(\bm{T}(\tau)\) tangent to \(U^a\). Its profile can be factorized as
\begin{equation}
    \phi(\bm{T}) \approx \psi(\tau)\,\chi(n^A),
\end{equation}
where \(\tau\) parameterizes motion along \(U^a\) and \(n^A\) are coordinates in the two-dimensional normal plane. The zero-net-energy condition demands that the action carried by \(\psi\) and by the transverse shape \(\chi\) cancel when integrated over \(T_1, T_2, T_3\), preventing hidden energy sources.

A representative Lagrangian in the purely temporal setting is
\begin{equation}
    \mathcal{L} = \frac{1}{2}\left( (\partial_\tau \phi)^2 - \delta^{AB}\partial_A \phi\,\partial_B \phi - m^2 \phi^2 \right),
\end{equation}
which yields, after integrating out the normal directions, an effective equation for \(\psi\) that mirrors Klein--Gordon dynamics without presuming space.

\subsection{Shear as an effective interaction}

Variations in \(U^a\) across \(\mathcal{T}\) introduce mixing between motion along the braid and motion in the normal plane. Writing
\begin{equation}
    \Xi_{ab} = \nabla_a U_b - \nabla_b U_a,
\end{equation}
one finds corrections to the effective Lagrangian of the form
\begin{equation}
    \mathcal{L}_\text{eff} \supset - V_\text{geom}(\tau)\,\psi^2 - A(\tau)\,\psi\,\partial_\tau \psi + \cdots,
\end{equation}
where \(V_\text{geom}\) and \(A\) are functionals of \(\Xi_{ab}\). Smooth components give rise to gravitational effects; structured, oscillatory components behave like gauge couplings. All contributions respect the zero-net-energy constraint because \(\Xi_{ab}\) is derived from the balanced temporal flow.

\subsection{Stable modes and quantization}

Standing waves in the normal plane yield discrete mode families \(\chi_k(n^A)\) and corresponding effective fields \(\psi_k(\tau)\). Quantization proceeds as usual on the \(\psi_k\), but the spectrum and coupling structure are fixed by the temporal geometry rather than by an assumed spatial lattice. The absence of space does not remove locality; it reframes locality as limited overlap of braids within \(\mathcal{T}\).

\subsection{Causality and consistency}

Causality is enforced with respect to the monotonic parameter \(\tau\) along \(U^a\). No closed causal loops are allowed in \(\mathcal{T}\), and changes in \(\psi_k\) depend only on data in their past along \(\tau\). Emergent lightcones in the projected description arise from the connection \(\Gamma^i{}_a\) defined in Section~\ref{sec:geometry}; their stability traces back to the balanced flow condition and the bounded shear encoded in \(\Xi_{ab}\).

\section{C. Mapping to Standard Relativistic Field Theory}

\subsection{Recovering special relativity}

To connect the triaxial time picture with ordinary special relativity, we
specialize to the case of a flat, unsheared membrane. Concretely:

\begin{itemize}
    \item The membrane \(\Sigma\) is a flat subspace defined by
    \(n^A = 0\), with constant normal vectors in \(\mathbb{R}^3_{\bm{T}}\).
    \item The effective time direction \(U^a\) is constant and aligned
    with one of the Cartesian axes, say \(T_1\).
\end{itemize}

In this regime, the projection
\(\pi: (\mathbf{x}, \bm{T}) \mapsto (t, \mathbf{x})\) reduces to
\(t = T_1/c\), and the induced metric on \(\Sigma\) is simply the Minkowski
metric
\begin{equation}
    ds^2 = -c^2 dt^2 + d\mathbf{x}^2.
\end{equation}

Field modes localized in the normal directions obey effective equations that
reduce to the standard relativistic wave equations for fields on Minkowski
spacetime. Lorentz invariance arises as the symmetry group of the induced
metric, independent of the choice of coordinates in the embedding
\(\mathcal{E}\).

Thus, in the absence of shear and curvature, the triaxial time framework
exactly reproduces special relativity.

\subsection{Relation to general relativity}

In general relativity, gravity is encoded in the curvature of a four-dimensional
Lorentzian manifold \((M, g_{\mu\nu})\). In the present framework, the
induced metric on the membrane \(\Sigma\) plays the role of \(g_{\mu\nu}\),
while the embedding of \(\Sigma\) in \(\mathcal{E}\) provides additional data
in the form of extrinsic curvature.

Working at a formal level, we can write the effective Einstein equations on
\(\Sigma\) as
\begin{equation}
    G_{\mu\nu}[g] = 8\pi G\,T_{\mu\nu}^\text{eff},
\end{equation}
where the effective stress-energy tensor \(T_{\mu\nu}^\text{eff}\) includes
both matter contributions from the projected fields \(\psi_k\) and geometric
contributions from the extrinsic curvature and shear of the membrane.

This is reminiscent of brane-world scenarios, but with a crucial difference:
the extra dimensions here are temporal rather than spatial, and the metric in
the temporal sector is chosen to avoid multiple physical time-like directions
on the membrane itself.

In a low-shear regime, the extrinsic contributions can be small, and one
recovers ordinary general relativity to high accuracy. In highly curved or
sheared regions, deviations from Einstein gravity can occur, potentially
providing phenomenological handles on the triaxial temporal structure.

\subsection{Quantum theory as distribution along \(T_3\)}

Quantum theory introduces complex amplitudes over configuration space and
time. In the triaxial time picture, the formal role of these amplitudes is
played by distributions over the modal axis \(T_3\).

Consider a state localized at spatial position \(\mathbf{x}\) and with some
spread along \(T_3\). Denote the corresponding field configuration as
\(\phi(\mathbf{x}, T_1, T_2, T_3)\). The probability amplitudes in standard
quantum mechanics correspond to the coefficients of a decomposition along
\(T_3\),
\begin{equation}
    \phi(\mathbf{x}, \bm{T}) =
    \int dT_3\,\psi(\mathbf{x}, T_1, T_2; T_3)\,\chi(T_3),
\end{equation}
with the normalization condition
\begin{equation}
    \int dT_3\,|\psi(\mathbf{x}, T_1, T_2; T_3)|^2 = 1
\end{equation}
corresponding to conservation of total modal weight.

The unitary time evolution of the quantum state is then reinterpreted as the
deterministic evolution of the full field \(\phi\) along the effective time
direction \(U^a\), with the Born rule arising as a measure over cross-sections
of the \(T_3\) distribution when a process forces localization in that
direction.

This is not yet a complete quantum theory. However, it suggests a program:
translate standard quantum dynamics into constraints on how mode-structures
may be distributed and evolve along the modal axis \(T_3\), subject to the
geometric and causal structure introduced earlier.

\subsection{Internal symmetries from temporal rotations}

Internal symmetries, such as those associated with gauge groups, can be
encoded as rotations and deformations within the temporal sector. For
example, a simple \(U(1)\) symmetry may be associated with phase rotations
that correspond to rotations in a plane embedded in the \(T_2\)-\(T_3\)
subspace.

Higher non-Abelian gauge symmetries can arise from more elaborate isometries
of the temporal metric \(\delta_{ab}\) that leave the membrane structure
invariant. The corresponding gauge fields appear as components of the
connection describing how the effective time direction \(U^a\) and the
normal vectors twist across space.

This picture parallels certain Kaluza--Klein constructions but assigns a
distinct ontological role to the extra dimensions as temporal rather than
compact spatial directions.

\section{D. Empirical Constraints and Observational Windows}

\subsection{Compatibility with existing tests}

Any extension of the temporal structure of physics is severely constrained by
the empirical success of special and general relativity and of quantum field
theory. At a minimum, the triaxial time model must satisfy:

\begin{itemize}
    \item No detectable violation of Lorentz invariance in regimes where it
    has been well tested.
    \item No gross modification of gravitational dynamics at solar system and
    binary-pulsar scales.
    \item No conflict with high-energy collider experiments and precision
    quantum tests.
\end{itemize}

In the present framework, these constraints are met by requiring that, on
accessible scales:

\begin{enumerate}
    \item The membrane \(\Sigma\) is nearly flat in the temporal sector, with
    small extrinsic curvature relative to experimentally probed lengths.
    \item Shear in the effective time direction \(U^a\) is mild and slowly
    varying, such that its contribution to the effective dynamics can be
    renormalized into standard couplings.
\end{enumerate}

This is analogous to how many higher-dimensional models evade low-energy
constraints by confining accessible physics to a nearly flat submanifold.

\subsection{Where deviations might appear}

If the triaxial temporal structure is real, it should manifest in regimes
where the membrane is significantly curved or sheared in the temporal sector,
or where mode-structures become highly extended in \(T_2\) or \(T_3\).

Candidate regimes include:

\begin{itemize}
    \item \textbf{Strong gravity:} near black holes or in the early universe,
    extrinsic curvature in the temporal sector could be large, leading to
    deviations from classical general relativity in the form of modified
    horizon structure, evaporation dynamics, or singularity resolution.
    \item \textbf{Macroscopic quantum coherence:} systems that maintain
    coherent superpositions across long durations (in \(T_1\)) and over many
    degrees of freedom may probe the structure of the modal axis \(T_3\).
    \item \textbf{Low-energy anomalies:} certain persistent anomalies in
    cosmology or astrophysics (e.g. dark energy phenomenology, long-range
    correlations) might conceivably be reinterpreted as manifestations of
    temporal shear rather than new matter components.
\end{itemize}

At this stage the discussion is necessarily qualitative. The purpose of this
section is less to claim concrete predictions than to delineate where a
triaxial temporal ontology would be most plausibly testable.

\subsection{Conceptual predictions}

The model also makes conceptual predictions about how certain puzzles should
eventually be resolved:

\begin{itemize}
    \item \textbf{Arrow(s) of time:} rather than a single thermodynamic
    arrow aligned with a unique time direction, there should be several
    distinct arrows associated with different projections of the mode-structure
    onto the three temporal axes. Entropy increase in physical processes
    corresponds to a particular alignment of these arrows with the effective
    time direction \(U^a\).
    \item \textbf{Quantum measurement:} wavefunction ``collapse'' should be
    replaced by a geometric localization of support along \(T_3\), triggered
    when mode-structures become entangled with sufficiently extended
    structures along \(T_2\) (records, memories).
    \item \textbf{Nonlocal correlations:} spacelike quantum correlations may
    be reinterpreted as local relations in the extended temporal manifold,
    where modes share structure along \(T_2\) and \(T_3\) even when separated
    along \(\mathbf{x}\).
\end{itemize}

These are not yet empirical predictions in the strict sense, but they
constitute constraints on how a completed triaxial time theory would have to
interface with unresolved foundational issues.


\section{E. Mind, Information, and Extended Temporal Structure}

\subsection{Minds as extended regions in temporal space}

In ordinary neuroscience and physics, a mind is modeled as a dynamical
pattern in the brain over time. In the present framework, a mind is modeled
as a \emph{temporally extended region} of the membrane and its neighborhood
in the temporal sector.

Informally, one can picture:

\begin{itemize}
    \item A three-dimensional spatial region corresponding to the organism.
    \item A thick ``trail'' along \(T_2\) encoding persistent records and
    synaptic structures: memory as literal extension in the persistence axis.
    \item A cloud-like spread along \(T_3\) corresponding to contemplated
    alternatives, simulations, and counterfactuals: imagination and
    deliberation as extension along the modal axis.
\end{itemize}

The effective physical time direction \(U^a\) in and near such a region is
then shaped by the interplay of these extensions. The mind is not an
instantaneous point but a coherent, self-maintaining mode-structure occupying
finite volume in the three-dimensional temporal manifold.

\subsection{Free will as temporal self-organization}

In this picture, free will is not a primitive, unexplained power, nor is it
an illusion. Rather, it is a specific type of self-organization in the
temporal manifold:

\begin{quote}
A system exhibits free will to the extent that it can stably reshape the
orientation and extension of its own mode-structure along \(T_2\) and \(T_3\)
in a way that is causally consistent along \(T_1\) but not reducible to a
fixed law acting on instantaneous states.
\end{quote}

This is compatible with the dynamical laws being local and deterministic in
the full embedding space \(\mathcal{E}\). The crucial point is that the
state of a mind at a given \((\mathbf{x}, \bm{T})\) includes, as intrinsic
data, its extended structure along \(T_2\) and \(T_3\). That extended
structure embodies past commitments and future evaluations, which in turn
shape the local effective dynamics.

Nothing in this reconstruction violates causal closure along \(T_1\). It
simply recognizes that the causal domain of a mental event is not a thin
slice at a single physical time but a thickened region across the three
temporal axes.

\subsection{Information and records}

Records, memories, and informational structures correspond to patterns that
are extended primarily along \(T_2\). For example:

\begin{itemize}
    \item A written page is a physical system whose field configuration is
    relatively invariant under translations along \(T_2\), up to slow
    degradation.
    \item A neural memory trace is a mode-structure in the brain that
    maintains its shape along \(T_2\) while participating in ongoing
    dynamics along \(T_1\).
\end{itemize}

The robustness of a record correlates with the degree to which its associated
modes are extended and stabilized along the persistence axis. This provides a
geometric underpinning for the intuitive idea that information is that which
\emph{can be carried forward} through time.

\subsection{Modal exploration and deliberation}

Deliberation---considering alternative actions, simulations, or outcomes---is
modeled as temporary expansion of certain structures along \(T_3\). A mind
``explores'' branches in the modal axis, propagating virtual dynamics there
while remaining anchored to a single realized trajectory on the membrane.

When a decision is made, the extended modal structure contracts or reorients,
effectively selecting a narrower band of \(T_3\) consistent with the chosen
course. This is not a literal branching of physical worlds; rather, it is a
reconfiguration of mode-structure in the extended temporal manifold that
leaves a trace along \(T_2\) (a memory of the decision and its rationale).

A full theory of such processes would require coupling the triaxial temporal
geometry to models of neural dynamics. The present section serves only to
indicate how the framework admits a place for mental and intentional
phenomena without abandoning physical causality.

\section{F. Mathematical Appendix and Toy Models}

This appendix sketches simple toy models that illustrate how the general
ideas in the main text can be implemented in explicit equations. The goal is
not completeness but concreteness.

\subsection{Toy model: linearized membrane in flat temporal space}

Consider a flat temporal space \(\mathbb{R}^3_{\bm{T}}\) with coordinates
\((T_1,T_2,T_3)\) and a membrane defined by
\begin{equation}
    T_2 = \epsilon f(\mathbf{x}), \quad T_3 = 0,
\end{equation}
with small parameter \(\epsilon\) and smooth function \(f\). The base time
coordinate is identified with \(T_1\).

To first order in \(\epsilon\), the induced metric on the membrane acquires
corrections of order \(\epsilon^2\), while the extrinsic curvature has
components proportional to \(\epsilon\,\nabla_i \nabla_j f\). For a scalar
field localized around the membrane, one finds effective corrections to its
mass and kinetic terms controlled by these curvature components.

This simple model shows how spatial variations in the embedding can generate
effective potentials at low energy, providing a crude analog of gravitational
backgrounds.

\subsection{Toy model: modal axis as continuous label}

To connect with quantum-like structure, consider a scalar field
\(\phi(\mathbf{x}, T_1, T_3)\) with no explicit dependence on \(T_2\). The
Lagrangian is
\begin{equation}
    \mathcal{L} = \frac{1}{2}\left[
        (\partial_{T_1}\phi)^2 - (\nabla\phi)^2
        - c_m^2 (\partial_{T_3}\phi)^2 - m^2 \phi^2
    \right].
\end{equation}
Fourier-transforming along \(T_3\),
\begin{equation}
    \phi(\mathbf{x}, T_1, T_3) =
    \int \frac{dk_3}{2\pi}\,e^{ik_3 T_3}\,\tilde{\phi}(\mathbf{x}, T_1; k_3),
\end{equation}
we obtain a tower of modes \(\tilde{\phi}(\mathbf{x}, T_1; k_3)\) with
effective masses
\begin{equation}
    m_\text{eff}^2 = m^2 + c_m^2 k_3^2.
\end{equation}
The distribution over \(k_3\) plays the role of a wavefunction spread over
modal configurations. A process that localizes the field in \(T_3\) (e.g. via
interaction with a large environment extended in \(T_2\)) corresponds to
projection onto a narrower band of \(k_3\), echoing quantum collapse.

\subsection{Constraints on signatures}

A central mathematical concern is avoiding multiple independent time-like
directions in the effective four-dimensional physics. In this framework, this
is accomplished by:

\begin{itemize}
    \item Assigning a positive-definite metric to the temporal sector
    \(\mathbb{R}^3_{\bm{T}}\) at the level of the embedding space.
    \item Defining the physical time direction \(U^a\) as a distinguished
    tangent vector on the membrane, with only that direction contributing a
    negative signature to the induced metric.
\end{itemize}

This effectively realizes the extra temporal axes as internal parameters
rather than additional physical time directions, despite their ontological
temporal interpretation.

\subsection{Open technical problems}

Several mathematical challenges remain:

\begin{itemize}
    \item Constructing explicit, globally well-behaved embeddings \(\Sigma
    \subset \mathcal{E}\) that reproduce known cosmological solutions.
    \item Developing a full field-theoretic treatment of gauge fields as
    connections associated with isometries in the temporal sector.
    \item Formulating a rigorous Hilbert-space structure tied to distributions
    along \(T_3\), with a clear derivation of the Born rule.
\end{itemize}

These tasks are left for future work.

\input{sections/conclusion}

\bibliographystyle{unsrt}
\bibliography{bibliography}

\end{document}
