\documentclass[12pt]{article}

% =======================
% Packages
% =======================
\usepackage{amsmath, amssymb}
\usepackage{geometry}
\usepackage{graphicx}
\usepackage{hyperref}
\usepackage{physics}
\usepackage{bm}
\usepackage{setspace}
\geometry{margin=1in}
\setstretch{1.2}

% =======================
% Metadata
% =======================
\title{Temporal Harmonics and the Emergence of Space}
\author{Mark Lindholm}
\date{\today}

% =======================
% Document
% =======================
\begin{document}

\maketitle
\tableofcontents
\newpage

% =======================
% Abstract
% =======================
\begin{abstract}
We propose a unifying ontology in which reality consists solely of duration structured as a three-dimensional temporal volume with axes $T_1$ (generative advance), $T_2$ (memory depth), and $T_3$ (transverse superposition). The sole degree of freedom is a phase field $\phi(T_1,T_2,T_3)$. Space emerges as the relational geometry of phase gradients within successive $T_1$ cross-sections. Particles are closed phase windings; fields are open propagations; minds are recursive structures of sufficient depth and width.

Uniform phase is silence. Structure appears only where phase varies. Persistent entities are not substances but topological windings—patterns that cannot unwind under continuous deformation. Their inertia reflects the depth of structure required to reorganize them; their apparent location is the manner in which they intersect moving slices of the temporal volume.

Motion is re-encounter across slices. Continuity belongs to slices; quantization belongs to topology. Phase gradients relax monotonically along $T_1$, defining the arrow of time. Causality is constraint propagation between slices; memory is preserved where topology and local minima resist relaxation.

Interaction requires no forces. Structures couple when internal spectra overlap; coupling strength is resonance. Coherent relaxation propagates as light; incoherent relaxation as heat. Mind emerges when a structure can model its own relaxation and hold multiple possible excitations simultaneously, enabling lawful selection among futures.

Locality is not absolute: proximity in memory or possibility may dominate spatial separation. The theory reproduces established physics on the observed projection while accounting for memory, intention, and macroscopic coherence within a single, time-only framework.
\end{abstract}

\newpage

% ============================================================
% I. Ontological Foundations
% ============================================================

\section{The Exhaustion of Spacetime}

The conceptual framework of spacetime has reached its point of exhaustion. This exhaustion is not empirical failure—spacetime continues to function as a powerful predictive instrument—but ontological saturation. All remaining paradoxes in fundamental physics arise not from missing terms in spacetime-based equations, but from category errors imposed by treating spacetime as primitive.

Spacetime was never an ontology; it was a bookkeeping device elevated by success. By unifying spatial coordinates with temporal ordering, it allowed dynamics to be expressed geometrically and relativistic invariants to be preserved. Yet from its inception, spacetime has required the silent importation of non-spatiotemporal structures to function: quantum state spaces, probability amplitudes, Hilbert geometries, internal symmetries, and observer-dependent collapses. These additions are not optional embellishments. They are structural compensations for spacetime’s inability to account for persistence, identity, memory, or possibility.

Three persistent failures signal this exhaustion.

First, spacetime cannot account for persistence without reification. Objects are treated as worldlines—extended geometric traces—yet nothing within spacetime explains why a given worldline remains the \emph{same} object across slices. Identity is assumed, not derived. Mass, charge, and spin are assigned as intrinsic labels rather than emergent invariants. Inertia is treated as a property of matter rather than the cost of reorganizing structure.

Second, spacetime cannot reconcile locality with quantum coherence. Entanglement, superposition, and nonlocal correlations do not merely strain spacetime intuitions; they violate its ontological commitments. Attempts to preserve spacetime locality require either hidden variables, many worlds, or observer-dependent collapses, each of which imports additional ontological machinery that spacetime itself cannot host.

Third, spacetime cannot accommodate memory, intention, or agency without contradiction. The arrow of time is imposed as boundary condition rather than derived. Observation is treated as an intervention rather than a process. Consciousness is either externalized or reduced to epiphenomenal noise. Spacetime offers no place for memory to reside, no mechanism for futures to coexist, and no account of how selection among possibilities occurs without violating causal closure.

These are not isolated problems. They share a common source: spacetime conflates ordering with existence. It treats temporal succession as an axis alongside spatial extension, thereby flattening duration into a coordinate rather than recognizing it as the generative substrate of structure.

The present work adopts the opposite stance. Time is not a parameter within a larger manifold. Duration is primary. Space, matter, fields, and observers are secondary—patterns that arise from structured variation within duration itself.

This shift is not a rejection of relativistic or quantum formalisms. It is a reclassification of what those formalisms describe. Spacetime becomes an emergent projection: a convenient geometry that appears when a higher-dimensional temporal structure is intersected by moving perceptual surfaces. Its success is explained, not denied. Its limits are exposed, not ignored.

By exhausting spacetime as a candidate for fundamental ontology, we clear the ground for a simpler primitive: a volume of duration endowed with internal structure. The remainder of this work develops that primitive explicitly and shows how the familiar physical world arises from it without remainder.

\section{The Temporal Volume as Primitive}

Having exhausted spacetime as a viable ontological foundation, we now state the positive alternative. The primitive of this theory is neither space, nor spacetime, nor matter, nor fields, but \emph{duration itself}: a continuous, extended volume of time within which structure may arise.

This temporal volume is not a metaphor. It is not an embedding space, not an abstract parameter set, and not a reinterpretation of spacetime coordinates. It is the sole arena of existence. All entities, processes, and appearances are patterns internal to it.

\subsection*{Primitive Assumptions}

The ontology rests on a minimal set of assumptions:

\begin{enumerate}
    \item \textbf{Continuity of Duration.}  
    Duration exists as a continuous volume, not as a sequence of instants. There is no privileged slicing, no smallest unit of time, and no external clock.

    \item \textbf{Dimensionality of Duration.}  
    Duration possesses three independent, orthogonal axes. These axes are temporal in character and together span the full volume of existence. They are not derived from space, nor reducible to a single temporal line.

    \item \textbf{Single Degree of Freedom.}  
    The only fundamental field is a scalar phase field $\phi(T_1,T_2,T_3)$ defined everywhere on the temporal volume. All structure arises from variation in this field.

    \item \textbf{No Primitive Objects.}  
    There are no particles, points, fields, forces, or observers at the fundamental level. These emerge as stable or metastable configurations of the phase field.
   \item \textbfd(Planck tone]
    Phase gradients are bounded by the universal Planck tone κ: |∇ϕ| ≤ κ—the irreducible tension stabilizing integer windings and emergent discretization.
    \item \textbf{No External Lawgiver.}  
    There are no imposed equations of motion acting on the system from outside. All lawful behavior emerges from internal constraints, symmetries, and relaxation dynamics of the phase field.
\end{enumerate}

Nothing else is assumed. In particular, the theory does \emph{not} assume space, spatial distance, mass, energy, causation, probability, or measurement as primitives.

\subsection*{Uniform Phase and Silence}

The baseline state of the temporal volume is uniform phase:
\[
\nabla \phi = 0.
\]
This state corresponds to complete silence: no structure, no persistence, no distinction. Importantly, this state is not “nothing.” The temporal volume remains fully present, fully extended, and fully capable of supporting structure. Uniformity is absence of differentiation, not absence of being.

Structure appears only where phase varies. All phenomena correspond to gradients, windings, or constraints within $\phi$. There is no additional substance beneath the phase field. Phase is not carried by something else; it is the thing.

\subsection*{Symmetry and Invariance}

The temporal volume admits the following fundamental symmetries:

\begin{itemize}
    \item \textbf{Translational invariance} along each temporal axis: there is no privileged origin in duration.
    \item \textbf{Rotational invariance} in temporal directions: no axis is inherently distinguished prior to structure.
    \item \textbf{Gauge freedom} in absolute phase: only phase differences are physically meaningful.
\end{itemize}

These symmetries replace the role traditionally played by spacetime covariance. Conservation laws arise not from imposed symmetries of space, but from invariances of the temporal volume itself.

\subsection*{Why a Volume, Not a Line}

A single temporal axis cannot support persistence without reintroducing objects by fiat. On a line of time, structure must be either instantaneous or externally preserved. By contrast, a volume of duration allows structures to be extended \emph{within time itself}. Persistence becomes a geometric fact rather than a dynamical miracle.

Closed paths, knots, and resonant loops are possible only in a space of sufficient dimensionality. The three-dimensional temporal volume is the minimal structure capable of supporting:
\begin{itemize}
    \item stable identities (closed phase windings),
    \item memory (extension along one axis),
    \item coexistence of possibilities (extension along another),
    \item and generative advance (relaxation along a third).
\end{itemize}

These roles will be made precise in the following sections. At this stage, it suffices to note that dimensionality is not introduced to match observed space, but to permit persistence and differentiation without contradiction.

\subsection*{Emergence Without Projection}

Nothing in the temporal volume resembles space, distance, or geometry as ordinarily conceived. There are no points, no metrics, and no neighborhoods in the spatial sense. Nevertheless, relational structure exists: proximity, orientation, and direction are defined internally through phase relationships and coherence constraints.

What is later perceived as space arises only when a moving two-dimensional section through the temporal volume is mistaken for an arena of existence rather than recognized as an interface. This misidentification is systematic, lawful, and unavoidable for embedded observers, which explains both the success and the limits of spatial description.

The remainder of this work develops the internal structure of the temporal volume, defines its axes explicitly, and demonstrates how matter, fields, gravitation, quantum phenomena, and mind arise as necessary consequences—not additions—to this primitive.

% ============================================================
% II. Structure of the Temporal Volume
% ============================================================

\addendum
\section{The Three Temporal Axes}
\label{sec:threeaxes}
The temporal volume is structured by three independent axes of duration, denoted $T_1$, $T_2$, and $T_3$. These axes are not spatial dimensions nor introduced by analogy with spacetime. They constitute the minimal orthogonal degrees of freedom required for a phase field to support irreversible evolution, persistent structure, and non-destructive coexistence of multiple configurations.
Each axis governs a distinct functional role in the phase field's behavior. Together they define the kinematic and dynamical substrate from which all observed phenomena emerge.
\subsection{The Generative Axis $T_1$: Irreversible Evolution}
The axis $T_1$ is the sole parameter of irreversible evolution. Dynamics along $T_1$ are governed by a gradient flow
$$\partial_{T_1} \phi = -\frac{\delta E}{\delta \phi},$$
where $E[\phi]$ is a bounded energy functional constructed from local phase gradients and stabilizing terms. This flow is strictly dissipative: $E$ decreases monotonically along $T_1$, yielding an intrinsic arrow of time without external boundary conditions.
All causal ordering, apparent change, and dynamical progression correspond to projections of this irreversible relaxation. $T_1$ encodes neither memory nor multiplicity; it governs only the forward unfolding of phase configuration.
\subsection{The Memory Axis $T_2$: Persistence and Structural Depth}
The axis $T_2$ encodes persistence. Extension along $T_2$ permits phase configurations to retain structure across successive advances of $T_1$ without continual regeneration.
Phase windings that extend deeply along $T_2$ resist reconfiguration. The cost of altering such structures increases with their $T_2$ depth, providing a natural measure of stability and inertia. Conventional inertial mass corresponds to the degree of $T_2$ extension required to reorganize a closed phase configuration.
Information along $T_2$ is not overwritten by $T_1$ evolution. This orthogonality prevents irreversible dynamics from erasing stored structure, enabling long-lived entities and memory without violating dissipation along $T_1$.
\subsection{The Transverse Axis $T_3$: Coexisting Configurations}
The axis $T_3$ permits the coexistence of multiple compatible phase configurations within the same $T_1$ slice. Extension along $T_3$ allows phase windings to occupy distinct transverse positions, avoiding immediate destructive interference.
Closed windings may circulate across finite regions of $T_3$, appearing in $T_1$ cross-sections as probabilistic multiplicity. Interaction strength between configurations depends on their overlap along $T_3$: strong coupling from alignment, weak or negligible interaction from separation.
This axis provides a geometric account of superposition and branching behavior without parallel worlds or stochastic mechanisms.
\subsection{Orthogonality and Minimal Dimensionality}
The three axes are orthogonal in function. Embedding memory ($T_2$) in the irreversible flow of $T_1$ would destroy persistence. Collapsing multiplicity ($T_3$) into either $T_1$ or $T_2$ would force coexisting configurations to interfere destructively or overwrite one another.
Only the combined structure of all three axes permits stable persistence, irreversible evolution, and co-present possibility to coexist within a single continuous temporal volume. Three dimensions are both necessary and sufficient for the full resonant structure observed in physical reality.
The temporal geometry emerging from these axes will be developed in the following section.

\section{Temporal Geometry Without Space}
\label{sec:temporalgeometry}
The temporal volume admits no primitive spatial geometry. There is no fundamental metric, no predefined notion of distance, and no intrinsic angular structure. All geometric relations arise internally from the phase field $\phi(T_1,T_2,T_3)$ and its variation across the temporal axes.
A canonical flat structure may be introduced for coordinate convenience, for example by treating each axis as compact. This structure carries no physical meaning. Observable relations depend exclusively on phase gradients, coherence relations, and topological invariants, not on any imposed metric signature.
\subsection*{Neighborhoods and Proximity}
Proximity is defined by phase coherence rather than distance. Two points $P$ and $Q$ in the temporal volume are considered proximate if the phase difference between them is small:
$$|\phi(P) - \phi(Q)| \ll 1,$$
with the bound ultimately constrained by the maximal allowable gradient $\kappa$ introduced later.
This relation induces an effective topology of connectedness: regions of smoothly varying phase form coherent domains, while regions where $|\nabla\phi|$ approaches its bound act as separatrices that inhibit resonance and interaction. No absolute scale of separation exists; connectivity is relational and local.
Apparent spatial separation arises as the cumulative phase dissonance encountered along coherent paths within a perceptual slice.
\subsection*{Directionality and Orientation}
Directionality is inherited from the functional roles of the temporal axes:

Alignment along $T_1$ corresponds to irreversible ordering.
Alignment along $T_2$ corresponds to persistence and structural depth.
Alignment along $T_3$ corresponds to coexistence without immediate interaction.

Orientation arises from the handedness of phase circulation. Closed phase windings possess intrinsic chirality, defined by the direction of circulation in transverse temporal planes. This chirality is a topological property of the winding itself, independent of any embedding space.
\subsection*{Phase Circulation and Rotational Structure}
Angular structure emerges from phase circulation around closed loops. The integral
$$\oint d\phi = 2\pi n$$
defines an integer winding number $n$, providing a measure of rotational structure without presupposing geometric circles.
What is later perceived as rotation corresponds to redistribution of phase alignment across $T_3$. Fractional shifts in transverse circulation replace geometric angles; invariants are preserved under the compact topology of the temporal volume.
\subsection*{Projected Metric Appearance}
When a perceptual slice intersects the temporal volume, phase gradients are registered as relational distances. Within such a slice, an effective line element may be defined:
$$ds^2_{\text{eff}} \propto (\Delta\phi)^2,$$
where $\Delta\phi$ is the minimal phase difference along coherent paths.
This metric is emergent and approximate. It reflects the registration of temporal dissonance under constrained perception, not an underlying spatial structure. Large-scale flatness corresponds to low average curvature of phase gradients across coherent domains.
\subsection*{Summary}
The geometry of the temporal volume is entirely relational:

Proximity arises from phase coherence.
Separation arises from phase dissonance.
Direction arises from axis alignment.
Rotation arises from phase circulation.
Metric structure arises only upon projection.

No primitive space is required. All geometric intuitions follow necessarily from structured variation within duration.
The emergence of apparent three-dimensional space from moving perceptual slices is treated in the following section.

% ============================================================
% III. Temporal Mathematics
% ============================================================
\section{Temporal Mathematics}
\label{sec:temporalmaths}
The mathematics appropriate to the temporal volume is not inherited from spatial geometry. Trigonometric functions, rotations, and wave behavior arise intrinsically from phase circulation within compact temporal directions. Sine and cosine are not ratios of lengths in space; they are the natural projections of uniform temporal circulation as registered by a moving perceptual slice.
\subsection*{Closed Temporal Circulation and Phasors}
A persistent resonant structure corresponds to uniform phase circulation along a compact temporal axis. Consider a closed winding along $T_3$, described by
$$\phi(T_3) = \omega T_3 + \theta_0.$$
This represents a phasor: a phase configuration of constant magnitude undergoing uniform circulation. The circulation is literal and internal to the temporal volume; no spatial embedding or geometric circle is presupposed.
The stability of the phasor follows from compactness. Returning to congruence after a full cycle enforces periodicity and quantization of phase structure.
\subsection*{Projection and Trigonometric Structure}
When the advancing perceptual slice intersects a circulating phase configuration, components of the circulation are registered along orthogonal degrees of freedom internal to the slice. These registered components correspond to
$$\cos\theta \quad \text{and} \quad \sin\theta,$$
where $\theta$ is the accumulated phase advance along the compact axis.
These functions are not derived from triangles. They are the projections of uniform temporal circulation onto independent slice coordinates. The identity
$$\sin^2\theta + \cos^2\theta = 1$$
expresses conservation of resonant intensity under projection.
Addition theorems arise from superposition of multiple circulating modes and encode phase alignment and coupling, not geometric angle addition.
\subsection*{Angles as Phase Fractions}
An angle is defined as a fractional phase advance along a compact temporal direction. The full cycle $2\pi$ corresponds to one complete circulation returning to congruence.
What is conventionally described as rotation corresponds to redistribution of phase alignment, particularly across $T_3$. Fractional shifts in transverse circulation replace geometric angles; invariants are preserved under the compact topology of the temporal volume.
\subsection*{Topological Invariants and Conservation}
Compact temporal topology enforces quantization of circulation:
$$\oint d\phi = 2\pi n, \quad n \in \mathbb{Z}.$$
These integers are invariant under continuous deformation, providing topological protection for persistent resonant structures. While phase values along open paths are gauge-dependent, closed-loop integrals and gradient magnitudes constitute true observables.
\subsection*{Propagation and Dispersion}
Open phase variation along $T_1$ appears in perceptual slices as wave-like propagation. The apparent propagation speed reflects the fixed advance of the slice intersecting structured temporal gradients, not a fundamental spacetime metric.
Dispersion arises from collective phase coherence and gradient flow dynamics rather than an underlying hyperbolic geometry. Wave equations emerge as effective descriptions of coordinated temporal circulation under projection.
\subsection*{Summary}
All familiar trigonometric and wave mathematics arise from a single structural fact: phase circulates in compact temporal directions and is registered by a moving slice.
Trigonometric functions are projections.
Angles are phase fractions.
Rotations are shifts in alignment.
Waves are open temporal gradients under advance.
The mathematics is not imported from space.
Space inherits it secondhand.

\section{No Metaphor Required}
\label{No Metaphor Required}

The terminology of resonance, harmony, circulation, and chord is employed here deliberately and literally. These terms are not illustrative devices imported from music; rather, musical theory is itself a precise descriptive language for periodic structure, phase alignment, and invariant circulation. Where the underlying structure is oscillatory, the musical vocabulary is exact.

The temporal volume admits compact directions along which phase circulates. Once compact circulation exists, harmonic structure follows necessarily. Frequencies, phases, overtones, beats, and resonances are not optional interpretations but intrinsic features of the mathematics. In this context, a \emph{chord} is not a metaphor for structure; it is a name for a closed, coherent superposition of circulating phases stabilized by topology.

Conventional physics already relies extensively on this language while concealing its origin. Fourier modes, eigenfrequencies, normal modes, standing waves, and dispersion relations are all explicitly harmonic constructs. They are typically described using spatialized terminology—modes in cavities, waves in boxes, particles in wells—yet their defining features are phase relations and resonance conditions. The present framework removes that indirection by treating harmonic structure as fundamental rather than emergent from space.

Accordingly, terms such as \emph{phasor}, \emph{circulation}, \emph{overtone}, and \emph{chord} are used to denote precise mathematical objects:
\begin{itemize}
\item circulation refers to compact phase advance;
\item resonance refers to stable phase alignment under bounded gradients;
\item overtones refer to nested or coupled circulations;
\item a chord refers to a closed, topologically protected resonant configuration.
\end{itemize}

This usage carries no aesthetic or experiential claim. No appeal is made to sound, perception, or analogy. The language is structural. Music is the metaphor only in the historical sense that musical theory developed early and accurately as a calculus of periodic phenomena.

In a temporal ontology where circulation precedes extension, harmonic language is not poetic. It is the shortest path to precision.

\addendum
\section{Dynamics and Matter}
\label{sec:dynamicsmatter}

Matter and dynamics are not primitive ingredients of reality. They emerge from structured variation of a single phase field $\phi$ defined on the temporal volume. No additional substances, forces, or spatial mechanisms are assumed.

The phase field is the sole dynamical entity. All apparent physical phenomena—persistent objects, fields, inertia, and interaction—are expressions of its constrained evolution.

\subsection{The Phase Field}

The phase field $\phi(T_1,T_2,T_3)$ is a smooth scalar field defined on the temporal volume, taking values in $\mathbb{R}/2\pi\mathbb{Z}$ to enforce single-valuedness under compact circulation.

Uniform $\phi$ corresponds to the absence of structure. All differentiation arises from variation in $\phi$. Phase gradients quantify local dissonance; coherence corresponds to regions of slowly varying phase.

Admissible gradients are bounded:
\[
|\nabla \phi| \le \kappa,
\]
where $\kappa$ is a universal bound on phase tension. This bound is not a discretization rule but a stability condition: gradients exceeding $\kappa$ cannot be sustained as coherent structures.

Irreversible evolution proceeds along $T_1$ via gradient flow:
\[
\partial_{T_1} \phi = -\frac{\delta E}{\delta \phi},
\]
with energy functional
\[
E[\phi] = \int_{T^2_{(T_2,T_3)}} \left( |\partial_{T_2}\phi|^2 + |\partial_{T_3}\phi|^2 + V(\phi) \right)\, dT_2\, dT_3.
\]

The functional depends only on transverse ($T_3$) and memory ($T_2$) structure. The bounded, periodic potential $V(\phi)$ ensures stability under circulation. Energy decreases monotonically along $T_1$, supplying an intrinsic arrow of time and thermodynamic direction.

\subsection{Closed Chords and Persistent Matter}

Persistent physical entities correspond to \emph{closed chords}: topologically protected phase windings that return to congruence after integer circulation along compact temporal directions. For any closed loop $\gamma$,
\[
\oint_\gamma d\phi = 2\pi n, \quad n \in \mathbb{Z}.
\]

These integers are invariants under continuous deformation. A closed chord cannot be eliminated without violating compactness or exceeding the gradient bound $\kappa$. Persistence is therefore structural, not imposed.

A minimal closed chord is localized primarily along one compact axis while extending finitely along the others. Composite chords arise from coupled or nested circulations.

Inertial mass corresponds to the depth of temporal structure required to reconfigure a chord. Chords with greater extension along $T_2$, or with tightly matched internal circulations producing long beat envelopes, resist change more strongly and therefore exhibit greater inertia.

\subsection{Fields, Interaction, and Charges}

Phase variation that does not close into integer circulation propagates under the gradient flow. Such open propagation constitutes a field.

Interaction does not require fundamental forces. Coupling occurs when the circulation spectra of two chords overlap. Alignment produces strong interaction; mismatch produces weak or negligible coupling. Interaction strength is therefore a function of resonant compatibility.

Conjugate handedness of transverse circulation provides the structural origin of oppositely coupled entities. What is conventionally identified as charge corresponds to this chirality. Internal circulation patterns within closed chords give rise to intrinsic angular properties, providing the origin of spin.

Coherent propagation appears in perceptual slices as radiation. Incoherent propagation relaxes gradients and manifests as thermal behavior.

\subsection{Summary}

Matter is not substance embedded in time. It is persistent temporal structure: closed chords stabilized by compact circulation and bounded gradients.

Dynamics are not forces acting across space. They are the relaxation, propagation, and coupling of phase structure under irreversible advance.

No metaphor is required. All physical phenomena arise from the constrained evolution of a single phase field within duration itself.

\section{The Moving Two-Surface and the Appearance of Space}
\label{sec:movingsurface}

Space does not exist as a primitive arena. What is experienced as space arises from the intersection of a higher-dimensional temporal volume with a lower-dimensional perceptual interface. This interface is a moving two-dimensional surface advancing through the temporal volume along the generative axis $T_1$.

This section makes that statement precise.

\subsection{The Perceptual Surface}

Consider a finite observer embedded within the temporal volume. Such an observer cannot access the full three-dimensional structure of duration simultaneously. Instead, access is restricted to a two-dimensional section of the temporal volume parameterized by $(T_2,T_3)$ at a given value of $T_1$.

This section constitutes the observer’s perceptual surface. It is not a physical object, nor a material boundary. It is a constraint on access: the subset of the temporal volume whose phase relations can be simultaneously registered by a finite structure.

As $T_1$ advances irreversibly, this surface moves continuously through the temporal volume. The observer does not experience this motion as motion of the surface itself. Instead, the sequence of surfaces is experienced as temporal succession.

\subsection{Intersection with Structure}

Closed chords and open propagations exist throughout the temporal volume. What the observer registers at any moment is not the chord itself, but its intersection with the moving surface.

A closed chord that persists across many values of $T_1$ intersects successive surfaces in a sequence of related phase configurations. This repeated intersection is experienced as the continued presence of an object.

Similarly, relative changes in intersection pattern across the surface are experienced as motion. Motion is not displacement through space; it is re-encounter of structured phase across successive surfaces.

Localization arises because chords intersect the surface in confined regions of $(T_2,T_3)$. Persistence arises because the same chord intersects many successive surfaces. Identity is maintained not by worldlines in space, but by topological continuity in the temporal volume.

\subsection{The Origin of Spatial Ordering}

Within a single perceptual surface, relations between intersections are ordered by phase coherence and dissonance. Regions of low phase difference appear proximate; regions separated by steep gradients appear distant.

This relational ordering is interpreted by the observer as spatial extension. The surface itself is mistaken for an arena rather than recognized as an interface. Because this misidentification is lawful and stable, it gives rise to a consistent geometry.

Dimensionality follows directly. The perceptual surface has two internal degrees of freedom. Combined with the irreversible ordering of successive surfaces along $T_1$, this yields the familiar appearance of three-dimensional space evolving in time.

No additional spatial dimensions are required. Apparent space is the composite of:
\begin{itemize}
    \item relational structure within a surface, and
    \item ordered succession of surfaces.
\end{itemize}

\subsection{Metric Appearance and Stability}

Distances, angles, and volumes arise as stable patterns of phase variation across the perceptual surface. The effective metric introduced earlier,
\[
ds^2_{\text{eff}} \propto (\Delta\phi)^2,
\]
is not fundamental. It is a bookkeeping device summarizing how phase dissonance is registered under the constraints of the surface.

The apparent rigidity of space reflects the slow evolution of large-scale phase structure relative to the advance of the surface. Where gradients are shallow and coherent across wide regions, space appears flat and stable. Where gradients curve or concentrate, space appears curved or distorted.

\subsection{Why Space Appears Fundamental}

For an embedded observer, nothing exists outside the perceptual surface. All interactions, measurements, and memories are formed from relations within it. The temporal volume itself is never directly accessible.

As a result, the surface is mistaken for reality rather than recognized as a cross-section. This mistake is not cognitive error but structural necessity. Any finite observer constrained to successive surfaces will construct an ontology in which the surface is fundamental and time is an external parameter.

Spacetime is the formalization of this misidentification.

\subsection{Summary}

Space is not a container in which events occur. It is the ordered appearance of relational structure on a moving two-dimensional surface intersecting a higher-dimensional temporal volume.

Objects are persistent intersections.
Motion is re-encounter.
Distance is phase dissonance.
Geometry is projection.

\section{Gravitation as Curvature of Coherence}
\label{sec:gravitationcoherence}

Conventional descriptions treat gravitation either as a force sourced by mass or as curvature of spacetime sourced by stress--energy. Both take space (or spacetime) as primitive and then explain motion within it. In the present framework, space is not primitive: it is the registered geometry of a moving two-surface intersecting a higher-dimensional temporal volume. Gravitation is therefore not a force acting across a pre-existing arena. It is a modification of the \emph{registered geometry} induced by persistent, high-tension phase structure.

In brief: \emph{gravitation is curvature of coherence on the perceptual surface}.

\subsection{Dense Chords and Coherence Distortion}

A massive object corresponds to a dense, persistent closed chord: a topologically protected winding with substantial extension along $T_2$ and strong transverse circulation structure in $T_3$. Such chords maintain elevated gradient energy over wide regions of the temporal volume while remaining stable under $T_1$ evolution.

The perceptual surface $\Sigma_{T_1}$ (a two-dimensional section parameterized by $(T_2,T_3)$ at fixed $T_1$) registers relational structure by phase coherence. In the presence of a dense chord, the local phase landscape becomes highly structured: coherent paths, separatrices, and regions of rapid phase variation appear in the slice.

This deformation of the phase landscape induces a deformation of the \emph{effective} geometry inferred on the surface. Using the effective line element introduced earlier,
\[
ds^2_{\mathrm{eff}} \propto (\Delta \phi)^2,
\]
the presence of a dense chord changes the local ``distance'' assignments by changing the minimal coherent phase differences $\Delta\phi$ between nearby surface points. The result is an effective metric that is no longer globally flat on $\Sigma_{T_1}$.

This is the operational content of gravitational curvature in the theory: local distortion of coherence relations, expressed as curvature of the effective surface geometry.

\subsection{Free Fall as Minimal Re-Phasing}

A closed chord evolving under irreversible advance encounters successive surfaces $\Sigma_{T_1}$. What is classically called inertial motion corresponds to re-encounter with minimal internal reconfiguration: the chord tracks those sequences of intersections that minimize required re-phasing of its internal circulation patterns.

In regions where the coherence geometry is distorted by a dense chord, the paths of minimal re-phasing coincide with geodesics of the effective metric on $\Sigma_{T_1}$. Thus, what appears as gravitational attraction is the tendency of chords to follow the least-cost re-phasing paths in a distorted coherence landscape.

No force is required. ``Acceleration'' occurs because the registered geometry has changed: the chord remains locally in minimal reconfiguration while the surface geometry defines the curved geodesic.

Stress arises only when a chord is constrained away from these paths (e.g.\ by contact constraints or sustained thrust), forcing additional internal re-phasing. This accounts for the phenomenology of weight and proper acceleration as costs of departing from coherence geodesics.

\subsection{Equivalence as a Surface Statement}

The equivalence principle follows because the observer's registered physics is defined on the perceptual surface. Locally, only relative coherence structure and the chord's re-phasing costs are measurable.

A local distortion of coherence induced by nearby dense chords and a uniform imposed deviation of the surface's advance correspond to the same operational effect: both change the re-phasing demand required to maintain stable intersection patterns across successive $\Sigma_{T_1}$.

Therefore, within a sufficiently small region of the surface, no measurement can distinguish curvature induced by external chord structure from a uniform non-geodesic constraint on the observer's own chord. Equivalence is not an additional axiom; it is a consequence of defining observables through surface registration and chord re-phasing.

\subsection{Newtonian Limit and Scaling}

In the weak-distortion regime---far from an isolated dense chord and where phase gradients vary slowly across the surface---the effective geometry can be linearized about a near-flat background. In this limit, coherence distortions behave as a potential-like modulation of the effective metric, and the resulting geodesic deviations reproduce Newtonian gravitational phenomenology to leading order.

The specific inverse-square scaling is not taken as primitive. It arises when (i) the chord's coherence distortion is approximately isotropic in the registered surface geometry and (ii) the dominant contribution is the lowest-order far-field term of a localized distortion. Under those conditions, the leading behavior decays with increasing coherence separation in the same manner as familiar $1/r^2$ laws in emergent three-space.

Deviations from Newtonian behavior correspond to regimes where these approximations fail: strong gradients near dense chords, anisotropic structure, overlapping distortions, or significant curvature of the effective surface geometry.

\subsection{Relativistic Correspondence}

General relativistic phenomena correspond to properties of the effective surface geometry and its evolution under $T_1$ advance, rather than curvature of a fundamental spacetime manifold. The correspondence is structural:

\begin{itemize}
\item Spacetime curvature $\longleftrightarrow$ curvature of the effective coherence geometry on $\Sigma_{T_1}$ and its variation across successive surfaces.
\item Geodesic motion $\longleftrightarrow$ minimal re-phasing paths (geodesics) defined by $ds^2_{\mathrm{eff}}$.
\item Gravitational time dilation $\longleftrightarrow$ differential re-encounter structure: the number of stable chord updates per unit $T_1$ registered varies with local coherence distortion.
\end{itemize}

The temporal volume remains the sole primitive. Curvature enters only as an emergent property of the observer's interface geometry.

\subsection{Summary}

Gravitation arises as curvature of coherence on the moving perceptual surface:

\begin{itemize}
\item Massive objects are dense, persistent closed chords that distort local coherence relations.
\item Free fall is geodesic motion in the effective surface metric, i.e.\ minimal re-phasing across successive surfaces.
\item Weight and acceleration are costs of enforced departure from these minimal re-phasing paths.
\item Newtonian gravity emerges in the weak-distortion, far-field limit; relativistic effects appear when distortions are strong or rapidly varying.
\end{itemize}

Gravitation is therefore not a primitive force and not curvature of a fundamental spacetime. It is the registered geometry of phase coherence under the constraints of a moving two-surface.

\section{Quantum Phenomena as Transverse Structure}
\label{sec:quantumtransverse}

Conventional accounts treat quantum phenomena as amendments to an underlying classical spacetime: superposition as probabilistic waves, measurement as collapse, and entanglement as nonlocal influence. These constructions preserve spacetime primacy while introducing auxiliary postulates whose interpretation remains contested.

In the present framework, quantum phenomena are not additional principles. They arise directly from transverse structure along $T_3$, the axis that permits co-present phase configurations. No modification of the dynamics is required.

\subsection{Superposition as Transverse Coexistence}

A closed chord may possess finite extent along $T_3$, allowing multiple compatible phase configurations to coexist within a single $T_1$ slice. This coexistence does not represent uncertainty about a definite state; it represents the simultaneous presence of multiple phase-aligned possibilities.

When registered on the perceptual surface, such a chord intersects in a distributed pattern. This distribution is identified phenomenologically as wave-like behavior or probabilistic multiplicity. The apparent indeterminacy reflects finite transverse width, not ignorance or stochasticity.

Superposition is therefore geometric: it is the registration of $T_3$ extension under projection. No probabilistic interpretation is imposed at the fundamental level.

\subsection{Measurement as Resonant Alignment}

Measurement corresponds to dynamical coupling between phase structures. When a stable chord (the measuring system) interacts with a transversely extended chord, overlap in $T_3$ circulation produces resonant alignment.

This interaction dampens incompatible transverse components and stabilizes a dominant alignment nested along $T_2$. The registered outcome corresponds to the resulting coherent intersection pattern on the advancing surface.

No collapse postulate is required. Reduction of multiplicity occurs through irreversible relaxation along $T_1$ driven by coupling, not by observer intervention or external randomness.

\subsection{Entanglement as Shared Transverse Structure}

Entanglement arises when multiple closed chords share correlated circulation structure across $T_3$. Their transverse phase relations are not independent; they are components of a single composite configuration.

Because this shared structure exists outside the perceptual surface, correlations appear nonlocal when viewed through spatial ordering alone. Changes registered in one intersection pattern correspond to reconfiguration of the shared transverse structure, not signal transmission.

Violations of Bell inequalities follow naturally: correlations are established by common $T_3$ structure prior to measurement. Measurement selects compatible alignments rather than revealing preassigned local values.

\subsection{Quantization and Discrete Spectra}

Quantization arises from compact temporal topology and bounded gradients. Phase circulation along compact axes enforces integer winding:
\[
\oint d\phi = 2\pi n,\quad n\in\mathbb{Z}.
\]

Distinct winding numbers and nested overtones along $T_2$ correspond to discrete excitation states. Transitions between such states release open propagations, registered as quanta of radiation.

Discrete spectra therefore follow from topology, not from imposed boundary conditions or operator postulates.

\subsection{Summary}

Quantum phenomena originate in transverse temporal structure:

\begin{itemize}
\item Superposition reflects finite extension along $T_3$.
\item Measurement is resonant alignment under coupling and relaxation.
\item Entanglement arises from shared transverse circulation.
\item Quantization follows from compact topology and bounded gradients.
\end{itemize}

No collapse mechanism, hidden variables, or nonlocal signaling are required. Quantum behavior is the registered projection of co-present temporal structure onto a moving perceptual surface.

% ============================================================
% VI. Observers and Measurement
% ============================================================

\section{The Observer as a Finite Chord}
\label{sec:observerchord}

The present framework admits no primitive observers and no privileged measurement process. An observer is a particular kind of physical structure: a closed chord with sufficient internal complexity to register, retain, and compare phase relations across successive perceptual surfaces.

\subsection{Finite Access and Partial Registration}

A closed chord cannot access the temporal volume globally. Its structure limits which phase relations can be simultaneously registered. In particular, a finite chord accesses only a subset of the perceptual surface at each advance along $T_1$, and retains information only to the extent permitted by its extension along $T_2$.

Observation is therefore partial by construction. It consists of registering relational phase structure within a moving two-surface and nesting selected relations into memory depth. No observer ever accesses the full state of a transversely extended configuration.

This structural limitation explains the necessity of effective descriptions, coarse-graining, and statistical regularities without invoking epistemic ignorance as a fundamental principle.

\subsection{Measurement as Structured Interaction}

Measurement occurs when two chords interact strongly enough that their phase structures couple and reorganize. The measuring chord provides a stable internal reference, while the measured configuration provides transversely extended structure.

Coupling aligns compatible circulation modes and suppresses incompatible ones through irreversible relaxation along $T_1$. The result is a shared phase alignment nested into the observer’s $T_2$ structure.

The registered outcome is not a revelation of a pre-existing value. It is the stabilized result of interaction under bounded gradients and compact topology. Measurement records are therefore physical structures, not abstract data.

\subsection{Simultaneity and Temporal Ordering}

Simultaneity is defined only within a perceptual surface. Because observers register relations within slices parameterized by $(T_2,T_3)$ at fixed $T_1$, simultaneity is surface-relative and observer-dependent.

Temporal ordering arises from irreversible advance along $T_1$, not from spatial relations. Differences in perceived ordering correspond to differences in how chords intersect successive surfaces, not to contradictions in underlying structure.

This resolves apparent paradoxes associated with observer-dependent collapse, delayed choice, and relativistic simultaneity: the disagreements arise from surface-relative registration, not from inconsistency in the temporal volume.

\subsection{Summary}

An observer is a finite closed chord with limited access to the temporal volume.

\begin{itemize}
\item Observation is partial registration on a moving surface.
\item Measurement is resonant coupling and stabilization.
\item Records are nested temporal structure.
\item Simultaneity is surface-relative.
\end{itemize}

No special role for consciousness is assumed. Observers are physical structures governed by the same dynamics as all others.

---

% ============================================================
% VII. Locality, Causality, and Signal Structure
% ============================================================

\section{Locality and Causality Without Spacetime}
\label{sec:localitycausality}

Locality and causality are not fundamental constraints imposed by spacetime geometry. They are emergent properties of how phase structure propagates and couples under irreversible advance along $T_1$.

\subsection{Locality as Coherence Constraint}

What appears as spatial locality corresponds to coherence locality: strong coupling occurs only when phase gradients overlap sufficiently within the perceptual surface.

Structures separated by large phase dissonance do not interact appreciably, regardless of their apparent spatial proximity. Conversely, structures with shared transverse or memory structure may exhibit strong correlations despite apparent spatial separation.

Locality is therefore relational, not geometric. It is defined by coherence accessibility, not by metric distance.

\subsection{Causality as Constraint Propagation}

Causality arises from the monotonic relaxation of phase gradients along $T_1$. Influences propagate only through allowed gradient flows consistent with the bound $\kappa$ and the energy functional governing relaxation.

Because $T_1$ advance is irreversible, causal structure is directed. Effects follow causes not because of imposed light cones, but because reconfiguration of phase structure cannot occur backward along $T_1$.

This provides a causal ordering without requiring spacetime as a mediating arena.

\subsection{Signal Propagation and Speed Limits}

Signals correspond to organized open propagations of phase gradients. The apparent finite speed of signal transmission arises from limits on how rapidly coherent phase structure can reorganize across successive surfaces.

These limits are set by gradient bounds and collective coherence, not by a fundamental spacetime metric. What is conventionally identified as the speed of light reflects the maximal rate at which coherent transverse structure can be registered under $T_1$ advance.

No superluminal signaling is possible because no structure can induce coherent reconfiguration outside the permitted relaxation pathways.

\subsection{Nonlocal Correlations Without Nonlocal Influence}

Entanglement and other quantum correlations do not violate locality at the fundamental level. They arise from shared transverse structure that exists prior to surface registration.

Because this structure is not confined to the perceptual surface, correlations appear nonlocal when interpreted spatially. However, no signal is exchanged, and no causal influence propagates faster than allowed gradient relaxation.

Local causality is preserved; spatial intuition is what fails.

\subsection{Summary}

Locality and causality are emergent constraints on phase interaction:

\begin{itemize}
\item Locality reflects coherence accessibility.
\item Causality reflects irreversible relaxation.
\item Signal limits arise from bounded reconfiguration rates.
\item Nonlocal correlations reflect shared transverse structure.
\end{itemize}

Spacetime locality is a projection of these deeper constraints. When spacetime is treated as fundamental, the projection appears paradoxical. When duration is treated as primitive, the paradox dissolves.

\addendum
\section{Cosmology as Coherence Widening}
\label{sec:cosmologycoherence}

Conventional cosmology interprets large-scale structure through the expansion of spacetime from an initial singularity, supplemented by dark matter and dark energy to reconcile observation with theory. These constructs preserve spacetime as a primitive arena while introducing components whose physical status remains unclear.

In the present framework, cosmological phenomena arise from the progressive widening of collective phase coherence under irreversible advance along $T_1$. No expanding space, singular origin, or additional dark components are required.

\subsection{Apparent Expansion and Coherence Growth}

Observed redshift and the Hubble--Lemaître relation are conventionally attributed to recession in an expanding spatial manifold. Here, they arise from the gradual increase in the coherence domain accessible to collective resonant structures.

At earlier values of $T_1$, phase gradients are steep and coherence domains are small. Resonant coupling between distant closed chords is weak, and propagations intersect the perceptual surface with significant phase dissonance. When registered, this dissonance appears as redshift.

As $T_1$ advances, irreversible relaxation reduces average gradient tension and extends coherent domains across the perceptual surface. Structures that previously intersected with high dissonance now intersect more coherently. The resulting reduction in registered phase mismatch appears as increasing separation and systematic redshift when interpreted spatially.

The apparent expansion of the universe is therefore the registered effect of coherence widening, not physical stretching of space. The Hubble parameter measures the rate at which accessible coherence domains grow relative to the advance of $T_1$.

\subsection{Origin Without Singularity}

The conventional Big Bang model posits a singular initial state of infinite density and temperature. In the temporal volume, no such singularity exists.

The temporal volume is complete and unbounded. The apparent beginning of cosmological history corresponds to the earliest regime in which large-scale coherent structure becomes registrable by collective chords. In this regime, phase gradients approach the bound $\kappa$, coherence domains are minimal, and transverse and memory structure are largely unresolved.

When projected onto spatial descriptions, this regime appears as extreme energy density, rapid expansion, and thermal saturation. These features reflect overdamped phase structure under steep gradients, not a breakdown of physical law.

The cosmic microwave background corresponds to residual incoherent propagation from this early regime, preserved as collective relaxation proceeds and coherence domains widen.

\subsection{Dark Components as Projection Effects}

Dark energy corresponds to the continued, nonlinear widening of coherence domains. As relaxation progresses, the effective relational geometry registered on the perceptual surface expands at an accelerating rate. This produces distance--redshift relations conventionally attributed to a repulsive energy density, without requiring any additional physical field.

Dark matter arises from unresolved temporal structure. Closed chords with significant extension along $T_3$ and $T_2$ contribute to coherence curvature and gravitational effects while intersecting the perceptual surface weakly or intermittently. When projected spatially, their influence appears as missing mass.

Both effects result from incomplete registration: the temporal volume contains all structure, while the perceptual surface records only those aspects compatible with coherent intersection.

\subsection{Summary}

Cosmological phenomena arise from coherence widening under irreversible temporal advance:

\begin{itemize}
\item Apparent expansion reflects growth of coherent domains.
\item Redshift reflects residual phase dissonance from earlier narrow coherence regimes.
\item Accelerated expansion reflects nonlinear coherence relaxation.
\item Dark components reflect unregistered transverse and memory structure.
\end{itemize}

No singular origin, no expanding spacetime, and no auxiliary cosmological substances are required. The universe is the temporal volume itself, progressively revealed through an advancing interface whose accessible coherence increases with $T_1$.

The apparent temporal beginning is therefore a boundary of coherent registration, not a boundary of existence.

\section{Experimental Consequences}
\label{sec:experimental}
The present framework does not predict arbitrary deviations from established physics. It is constructed to reproduce relativistic and quantum phenomena in regimes where those theories are empirically successful. Experimental consequences arise primarily where spacetime-based descriptions are extrapolated beyond their natural domain: coherence limits, long-duration correlations, and projection effects involving memory and transverse structure.
This section outlines concrete observational and experimental consequences that distinguish temporal-volume dynamics from spacetime-primitive theories.
\subsection{Deviation from Strict Metric Locality}
Locality in the present framework is defined by coherence accessibility rather than metric distance. Correlations may persist or decay in ways not fully captured by spatial separation alone.
Observable consequence:
Correlation strength in quantum and mesoscopic systems should depend not only on spatial separation but on coherence history, including shared preparation pathways and memory depth.
This predicts that certain decoherence times and entanglement decay rates may show systematic dependence on preparation history beyond what is accounted for in standard environmental models. Existing quantum optics and superconducting qubit datasets may already contain such residual structure.
\subsection{Limits on Gravitational Quantization}
Gravitation arises from coherence curvature on the perceptual surface, not from a fundamental quantum field. Attempts to quantize gravity as a force carrier should fail beyond effective, emergent descriptions.
Observable consequence:
No scale-independent graviton signature should exist.
Quantum-gravity effects should appear only as corrections to coherence geometry, not as particle exchange.
This predicts that efforts to detect gravitons directly, or to observe gravitation-induced quantum superposition collapse as a fundamental process, will continue to yield null results.
\subsection{Anomalies in Early-Universe Inference}
If cosmological redshift arises from coherence widening rather than metric expansion, certain inferences drawn from early-universe models may be systematically biased.
Observable consequence:
Apparent horizon and flatness problems should admit reinterpretation without inflation.
The inferred equation-of-state parameters for dark energy should show scale-dependent deviations when fit across different coherence regimes.
Future high-precision redshift surveys may detect inconsistencies when a single spacetime expansion model is forced across all scales.
\subsection{Memory-Dependent Inertial Effects}
Inertia corresponds to depth of temporal structure along $T_2$. Systems with identical instantaneous energy but different structural histories may exhibit measurably different responses to perturbation.
Observable consequence:
Hysteresis-like effects in inertial response for systems with deep internal coherence (e.g., strongly correlated materials, macroscopic quantum states).
While subtle, such effects could be probed in precision mechanical oscillators, ultra-cold matter systems, or long-lived resonant cavities.
\subsection{Observer-Dependent Temporal Ordering}
Because simultaneity and ordering are surface-relative, the framework predicts no absolute global time ordering even in principle. It also predicts strict consistency within any single observer’s registered surface.
Observable consequence:
Apparent paradoxes involving delayed choice, retrocausality, or observer-dependent collapse should always admit a surface-consistent explanation without requiring global reordering.
This does not contradict existing experiments, but constrains future interpretations: any observation implying true backward causation would falsify the theory.
\subsection{Summary of Falsifiability}
The framework is falsified if any of the following are observed:

A fundamental spacetime-local mechanism that fully accounts for quantum nonlocal correlations without auxiliary structure.
A scale-independent graviton detected as a force mediator.
Experimental evidence of absolute simultaneity or global temporal ordering.
A necessity for singular initial conditions to explain cosmological coherence.

Absent such findings, the theory remains empirically viable and offers a unified explanatory framework for phenomena currently treated as conceptually disjoint.

\section{Limits and Open Questions}
\label{sec:limitsquestions}

The present framework offers a unified ontological foundation in which duration is primitive and observed physical structure arises through projection and coherence constraints. While this approach resolves several long-standing conceptual tensions, it remains incomplete. Certain phenomena are addressed only qualitatively or at the level of principle, and others await explicit derivation.

This section enumerates the principal limitations and open questions without evasion. The long-term viability of the framework depends on whether these issues can be resolved within the same ontology.

\subsection{Limitations of the Current Formulation}

\textbf{Fermionic Statistics.}  
The present formulation naturally produces bosonic-like behavior from topologically stable windings in a scalar phase field. Fermionic statistics—half-integer spin, antisymmetry, and Pauli exclusion—are not yet derived. Possible routes include effective braiding in emergent dimensions, multi-valued phase structure, or collective spinorial modes, but no formal construction is provided.

\textbf{Exact Lorentz Invariance.}  
Relativistic effects arise from surface geometry and coherence constraints, reproducing special-relativistic kinematics in appropriate limits. However, an explicit mapping to full Lorentz symmetry across arbitrary perceptual surfaces, and the precise status of local Lorentz invariance at the fundamental level, remain to be demonstrated.

\textbf{Scale Setting and Dimensionless Constants.}  
The Planck tone $\kappa$ and the compactification scales of the temporal axes are introduced as universal bounds but not derived from deeper principles. The numerical values of dimensionless constants (e.g.\ the fine-structure constant and particle mass ratios) are not predicted.

\textbf{Probability and the Born Rule.}  
Probabilistic multiplicity arises from transverse extent along $T_3$, and stable outcome frequencies from collective coherence and relaxation. However, the precise emergence of squared-amplitude weighting analogous to the Born rule remains an effective description rather than a fundamental derivation.

\textbf{Gravitational Coupling Strength.}  
Gravitation as coherence curvature accounts for equivalence and relativistic effects, but the numerical value of Newton’s constant and its relation to other physical scales are not obtained from first principles.

\subsection{Open Questions}

\textbf{Emergence of Vector and Gauge Structure.}  
Electromagnetism and other gauge interactions must arise from scalar phase dynamics, likely through effective connections associated with gradient flow or collective modes. The detailed mechanism remains undeveloped.

\textbf{Early Coherence Regimes.}  
While the framework avoids singular initial conditions, it does not yet explain why early coherence domains were narrow or why particular topological classes dominate at early times.

\textbf{Consciousness and Agency.}  
Mind is identified with sufficiently recursive and transversely extended chords capable of modeling their own relaxation. The threshold for subjective experience and the mechanism of intentional selection remain phenomenological rather than derived.

\textbf{Unification Across Scales.}  
The relationship between microscopic knot stability (set by $\kappa$) and macroscopic coherence domains (cosmological structure) is currently qualitative. A principled bridge between these regimes is required.

\textbf{Experimental Discrimination.}  
Although the framework is consistent with existing observations, sharply distinguishing predictions in accessible regimes—such as coherence-history dependence in quantum systems or memory-dependent inertial effects—require detailed quantitative modeling.

\subsection{Summary}

The framework resolves several foundational tensions—spacetime primacy, quantum measurement, gravitational quantization, and cosmological origin—by treating them as projection effects of structured temporal duration. It remains limited in its treatment of fermionic statistics, gauge structure, precise coupling constants, and probability laws.

These gaps are not ad hoc omissions but define the program of future work. The strength of the framework lies in its minimal assumptions and broad explanatory compression. Its ultimate status will be determined by whether these open questions can be resolved internally or by identifying empirical domains in which the projection-based account fails.


% ============================================================
% IX. Conclusion
% ============================================================

\section{Conclusion}
\label{sec:conclusion}

This work has shown that a three-dimensional temporal volume, endowed with a single scalar phase field and minimal structural constraints, is sufficient to account for the principal features of observed physical reality. Space emerges as relational ordering on moving perceptual surfaces. Matter arises as topologically protected phase windings. Gravitation appears as curvature of coherence on those surfaces. Quantum phenomena follow from transverse superposition, and cosmological evolution reflects the progressive widening of collective phase coherence.

No primitive spacetime, no fundamental forces, no auxiliary quantum postulates, and no external observers are required. The apparent complexity of the physical world—its geometry, dynamics, discreteness, and long-range correlations—arises from structured variation within duration itself.

The spacetime-based paradigm fragmented description into incompatible domains: classical geometry for gravitation, quantum formalisms for microscopic phenomena, and separate interpretive layers for measurement and cosmology. These divisions were not imposed by nature but by the assumption that spatial projection constituted fundamental ontology.

By restoring duration to primacy and treating harmonic circulation as the native structure of persistence and interaction, the present framework unifies these domains within a single explanatory substrate. The harmonic language employed throughout—circulation, resonance, chord, overtone—is not metaphorical but descriptive, reflecting the mathematics of compact phase structure prior to spatial interpretation.

The framework remains incomplete in certain respects. Fermionic statistics, precise coupling constants, and the detailed emergence of gauge structure are not yet derived. These limitations define a research program rather than a defect: either such structures will be obtained within the same ontology, or empirical domains will be identified in which the projection-based account fails.

The contribution of this work is not the rejection of established physics, but its reclassification. Relativity and quantum theory are preserved as effective descriptions of how a finite perceptual surface registers deeper temporal structure. Their successes are explained; their conceptual tensions are dissolved.

Duration is not a parameter within the world. It is the world’s sustaining structure. When treated as such, the apparent paradoxes of modern physics resolve into consistent consequences of limited access to a richer temporal volume.

% ============================================================
% Appendices
% ============================================================


\appendix
\section{Formal Definitions}
\label{app:formaldefinitions}

This appendix provides precise mathematical definitions for the core entities of the theory. All structures are constructed from the primitive temporal volume and a single scalar phase field. No spatial geometry, spacetime metric, or additional dynamical fields are presupposed.

\subsection{The Temporal Volume}

The primitive manifold is the three-dimensional torus
\[
\mathcal{T} = T^3 = S^1 \times S^1 \times S^1,
\]
with orthogonal temporal axes labeled $T_1$, $T_2$, and $T_3$.

Each $S^1$ factor is equipped with its canonical smooth structure and periodic identification. Local coordinates $(T_1,T_2,T_3)\in[0,2\pi)^3$ are used for convenience only. No fundamental metric or notion of spatial distance is imposed; the product structure serves solely as a coordinate chart and topological constraint.

\subsection{The Phase Field}

The sole dynamical degree of freedom is a smooth scalar phase field
\[
\phi : \mathcal{T} \rightarrow \mathbb{R}/2\pi\mathbb{Z},
\]
defined modulo $2\pi$ to ensure single-valuedness under closed circulation.

Physical structure arises exclusively from variation in $\phi$. The uniform configuration
\[
\nabla \phi = 0
\]
corresponds to the absence of differentiation and persistent structure.

\subsection{The Planck Tone and Gradient Bound}

Admissible phase variation is constrained by a universal bound $\kappa$ (the Planck tone):
\[
|\nabla \phi| \le \kappa .
\]

Here $|\nabla\phi|$ denotes the coordinate-invariant magnitude of phase variation per unit parameter length along the temporal axes. This bound is not a discretization rule; it is a stability constraint preventing arbitrarily sharp phase gradients and enabling topologically protected configurations.

\subsection{The Energy Functional}

Dynamics are generated by an energy functional defined on each $T_1=\text{const}$ slice:
\[
E[\phi] = \int_{T^2_{(T_2,T_3)}} 
\left(
|\partial_{T_2}\phi|^2 + |\partial_{T_3}\phi|^2 + V(\phi)
\right)\, dT_2\, dT_3 ,
\]
where $V(\phi)$ is a bounded, periodic potential. Its explicit form is not fixed by the framework; it serves only to stabilize circulation under compact topology.

The functional depends solely on transverse ($T_3$) and memory ($T_2$) structure. No $T_1$ derivatives appear in $E$.

\subsection{Evolution Law}

Irreversible evolution along $T_1$ is governed by gradient flow:
\[
\partial_{T_1}\phi = -\frac{\delta E}{\delta \phi}.
\]

This evolution satisfies the monotonicity condition
\[
\frac{dE}{dT_1} = - \int_{T^2_{(T_2,T_3)}} 
\left| \partial_{T_1}\phi \right|^2 \, dT_2\, dT_3 \le 0,
\]
providing an intrinsic arrow of time without external boundary conditions.

\subsection{Persistent Windings and Topological Charge}

Closed phase windings are classified by an integer topological invariant
\[
n = \frac{1}{2\pi} \oint_{\gamma} d\phi ,
\quad \gamma \subset \mathcal{T}.
\]

Nonzero $n$ cannot be removed by continuous deformation without violating compactness or exceeding the gradient bound $\kappa$. These invariants correspond to persistent resonant structures.

\subsection{The Perceptual Surface}

A finite observer registers phase relations on a two-dimensional surface
\[
\mathcal{S}(T_1) = \{ (T_2,T_3) \mapsto \phi(T_1,T_2,T_3) \},
\]
defined by fixed $T_1$.

The surface advances irreversibly along $T_1$. All observable quantities correspond to relational features of $\phi$ as registered on $\mathcal{S}(T_1)$.

\subsection{Emergent Observables}

Observable physical quantities are defined relationally:

\begin{itemize}
\item \textbf{Relational proximity}: small phase differences $|\Delta\phi|$ along coherent paths.
\item \textbf{Effective inertial depth}: curvature of $E$ in the neighborhood of stable windings.
\item \textbf{Apparent position}: intersection patterns of persistent windings on $\mathcal{S}(T_1)$.
\item \textbf{Transverse multiplicity}: finite extension of phase structure along $T_3$.
\end{itemize}

All conventional physical quantities—energy density, interaction strength, propagation characteristics—are derived from these primitives.

\subsection{Closure}

The formalism is complete under the stated assumptions. Extensions such as multi-component phase fields or additional symmetry constraints may be introduced, but are not required for the core derivation presented in this work.

\appendix
\section{Temporal Coordinate Transformations and Invariants}
\label{app:temporaltransformations}

This appendix specifies the admissible coordinate transformations on the temporal volume and identifies the quantities invariant under such transformations. These invariants define the physical content of the theory independent of parametrization or observer description.

\subsection{Coordinate Freedom on the Temporal Volume}

The temporal volume $\mathcal{T}=T^3$ admits independent reparameterizations along each axis:
\[
T_i \;\mapsto\; T_i' = f_i(T_i), \quad i \in \{1,2,3\},
\]
where each $f_i$ is a smooth, monotonic function respecting periodic identification on $S^1$.

These transformations correspond to changes of temporal gauge. They do not alter physical structure, provided they preserve orientation along $T_1$ (i.e.\ $df_1/dT_1 > 0$), ensuring irreversibility of advance.

No transformation mixes $T_1$ with $(T_2,T_3)$ at the fundamental level. Mixing appears only at the level of projected observables on perceptual surfaces.

\subsection{Phase Gauge Freedom}

The phase field admits global gauge freedom:
\[
\phi \;\mapsto\; \phi' = \phi + \phi_0,
\]
with $\phi_0 \in \mathbb{R}/2\pi\mathbb{Z}$ constant.

All physical quantities depend only on phase differences, gradients, and closed-loop integrals. Absolute phase carries no observable meaning.

Local gauge transformations $\phi \mapsto \phi + \chi(T_1,T_2,T_3)$ are not symmetries unless $\nabla\chi = 0$, as they alter gradient structure and therefore physical configuration.

\subsection{Invariant Quantities}

The following quantities are invariant under admissible coordinate and phase transformations:

\begin{itemize}
\item Topological winding numbers
\[
n = \frac{1}{2\pi} \oint_\gamma d\phi.
\]

\item The sign of $\partial_{T_1} E$, enforcing irreversibility.

\item The existence and stability class of local minima of $E[\phi]$.

\item Relative phase differences $\Delta\phi$ along coherent paths.

\item The spectrum of circulation frequencies associated with closed chords.
\end{itemize}

These invariants define persistent physical structure independent of coordinate choice.

\subsection{Observer-Dependent Quantities}

Quantities that depend on the choice of perceptual surface $\mathcal{S}(T_1)$ include:

\begin{itemize}
\item Apparent position and spatial separation.
\item Temporal simultaneity relations.
\item Observed propagation speed.
\item Ordering of events within a slice.
\end{itemize}

Such quantities are not invariant and should not be treated as fundamental. They reflect how a finite observer samples the temporal volume rather than properties of the volume itself.

\subsection{Effective Lorentz Transformations}

Effective Lorentz transformations arise as mappings between different observer descriptions of intersection patterns on perceptual surfaces. These transformations preserve relational ordering and causal structure within slices while redistributing phase gradients between apparent spatial and temporal components.

They are emergent symmetries of the projection process, not fundamental symmetries of the temporal volume. Their validity is restricted to regimes of low curvature and weak transverse structure.

\subsection{Causal Consistency}

Causality is preserved under all admissible transformations because:

\begin{itemize}
\item Evolution along $T_1$ is monotonic.
\item No transformation reverses the sign of $\partial_{T_1}$.
\item Phase reconfiguration propagates only through allowed gradient flow.
\end{itemize}

No coordinate choice permits backward influence or closed causal loops in $T_1$.

\subsection{Closure}

Physical predictions of the theory depend only on invariant phase structure and topological constraints. Coordinate systems, observer descriptions, and projection choices affect representation but not content.

This ensures that the framework possesses a well-defined notion of physical equivalence without reliance on spacetime covariance.

\section{Psychological Interpretation of $T_2$ and $T_3$}
% Cognitive mapping

\section{Historical and Conceptual Comparisons}
% Optional comparison to existing frameworks

% ============================================================
% Bibliography
% ============================================================

\bibliographystyle{plain}
\bibliography{references}

\end{document}
