\section{Deriving the Einstein Field Equations from Dissonance Relaxation}
\label{sec:gr}

The greatest triumph of the old ontology was Einstein's general relativity: a theory of gravity as the curvature of a fictitious spacetime manifold. In the unitary harmonic ontology of pure duration, gravity is not curvature of space but curvature of dissonance gradients in the temporal volume itself. The Einstein field equations emerge as the macroscopic, averaged relaxation dynamics of these gradients—no postulate of spacetime, no arbitrary metric ansatz, no geometrodynamic mysticism required.

Consider a region of the temporal volume containing many closed chords (massive bodies). Each chord maintains its topological winding by curving the phase field ϕ around itself to remain single-valued. This curvature slows the rate at which distant regions can relax toward uniform phase—consonance. The effective "clock rate" (the pace at which sub-chords oscillate) therefore varies with proximity to dense knots of chords.

To derive the field equations, coarse-grain the phase field over scales much larger than individual Compton tori but smaller than the Hubble volume. The local dissonance density is
\begin{equation}
\rho = \frac{1}{2\kappa^2} \langle (\nabla \phi)^2 \rangle,
\end{equation}
where the average is taken over many neighbouring chords. The relaxation equation for ϕ in the presence of this density is the simplest antisymmetric form permitted by the 3-form ontology:
\begin{equation}
\Box \phi + \kappa^2 \rho \sin \phi = 0,
\end{equation}
the driven wave equation for phase in a medium with variable tension.

In the weak-dissonance limit (ϕ ≪ 1), linearize:
\begin{equation}
\Box \phi + \kappa^2 \rho \phi \approx 0.
\end{equation}
Define the induced metric along T₁ slices as in Section 1:
\begin{equation}
g_{\mu\nu} = \eta_{\mu\nu} + h_{\mu\nu}, \quad h_{\mu\nu} = \frac{2}{\kappa^2} \partial_\mu \phi \partial_\nu \phi.
\end{equation}
The linearized relaxation equation then becomes the trace-reversed Einstein equation in the harmonic gauge:
\begin{equation}
\Box \bar{h}_{\mu\nu} = -16\pi G \left( T_{\mu\nu} - \frac{1}{2} \eta_{\mu\nu} T \right),
\end{equation}
with Newton's constant emerging as
\begin{equation}
G = \frac{\hbar \kappa}{c^5}.
\end{equation}

The full nonlinear Einstein equations follow from retaining higher orders in the expansion of sin ϕ and averaging over the stochastic background of vacuum ripples. The cosmological term arises naturally from the non-zero vacuum dissonance ρ_Λ (Section 5), yielding Λ = 8π G ρ_Λ exactly as observed.

Gravity is thus revealed as the macroscopic shadow of countless microscopic chords refusing to unwind. Massive bodies do not curve space; they slow the relaxation of dissonance in the temporal medium, and distant clocks tick slower in their presence because they must stretch their phase to reach consonance across the gradient.

Einstein's achievement was to describe the shadow with exquisite precision. This theory explains the light that casts it.

Spacetime is dead.  
The threefold song curves itself into gravity—and sings on.