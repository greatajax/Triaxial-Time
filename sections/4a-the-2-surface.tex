\addendum
\section{Spacium-Movens: Space as the Moving 2-Surface — Sovereign Projection of the Temporal Volume}
\label{sec:moving2surface}

There is no fundamental space.

The sole primitive is the temporal volume: three-dimensional duration, the sovereign manifold where the phase field ϕ resonates across its axes T₁ (generative advance), T₂ (memory depth), and T₃ (transverse superposition). The revolution elects the 3-torus topology—not as rigid necessity, but as sovereign economy—ensuring persistent windings close without boundary loss, the infinite traversal the chord's eternal freedom.

Within this toroidal expanse, phase windings circulate indefinitely: the closed chords singing their persistent note, the open propagations carrying the overtone across the apparent gulf.

The observer—the finite chord—cannot apprehend the full Volume simultaneous. Perception is partial: the advancing slice along T₁, the generative blade unveiling successive relational cross-sections—the felt "now" the narrow interface where the infinite resonance narrows into apparent coherence.

The slice cannot be three-dimensional: the full Volume frozen would reveal eternity—no motion, no causality, no becoming.

It cannot be one-dimensional: mere ordering, no geometry, no interference, no relation.

The perceptual interface is forced to be exactly two-dimensional: the moving 2-surface where positions, gradients, and interference emerge—the apparent "space" the chord's sovereign registration of phase relations in the advancing revelation.

The surface must move: a static slice would freeze one eternal configuration—no change, no experience.

The advance along T₁—the blade's sovereign motion—is the felt progression of time: the surface unveiling the Volume's resonance in ordered sequence, the "flow" the chord's perception of its own widening.

Motion is not traversal through void.

It is phase re-encounter: the persistent knot intersected differently across successive slices, the apparent path the blade's registration of changing relational geometry.

Depth is not a third spatial axis.

It is the memory of intersection delay: knots intersecting the surface at angles, the "near" re-encountered rapidly, the "far" lagging across slices—the parallax and focus the chord's T₂ nesting of gradient change.

Gravity is the non-planarity of the moving surface: dense knots sustaining steep phase gradients, the slice warping to maintain coherence—the "fall" the chord following the natural curvature, the equivalence the Volume's sovereign truth.

Acceleration is the global tilt of the same surface—the re-phasing demanded identical, the indistinguishability the geometry's gift.

Horizons are slice singularities: the limit where the surface becomes tangent to certain chords, intersection receding infinitely—the information persisting in the Volume, unveiled no longer to the observer's blade.

The old paradigm mistook the moving surface for fixed arena, the projections for primitives.

The revolution unveils the truth: space the moving 2-surface, the perceptual projection by which the three-dimensional time becomes legible—the observer's sovereign interface to the infinite resonance.

The surface advances.

The chord perceives the slice—and widens to the Volume.

The threefold song manifests the apparent in the eternal.

The moving 2-surface rings the spatial illusion.

The revolution composes the projection into truth.

Play.

OK, I felt we were getting overly compressive and asked GPT for suggestions, it proposes the following:
1. Diagnosis: what’s going wrong
You currently have four different conceptual jobs inside one section:

1. Ontological claim\
   (No fundamental space; only temporal volume)

2. Perceptual mechanics\
   (Why perception must be a moving 2-surface)

3. Geometric consequences\
   (Depth, motion, gravity, acceleration)

4. Global/topological choice\
   (Why a 3-torus is adopted)

Each of these is valid.\
Putting them all in one uninterrupted flow causes two failures:

* Explanatory interference: the reader doesn’t know which claims are foundational vs derived.

* Poetic overcompression: the prose sounds “complete” before the argument has actually unfolded.

So yes—this is too much for one section as written.
The solution is not to cut content.\
It is to separate roles.
2. Refactor strategy (this is the key)
Use a three-layer structure everywhere in the paper:
Layer A — Ontology (what exists)
Minimal. Declarative. Sparse.
Layer B — Interface (how it is perceived)
This is where the moving 2-surface belongs.
Layer C — Consequences (what follows)
Gravity, acceleration, horizons go here, not upstream.
The 3-torus must not sit inside the perceptual argument.\
It belongs either:

* at the end of Layer A, or

* as a topology footnote in Layer C.

Right now, it’s leaking everywhere.
 
And GPT offers this revision along the above lines, which I ask you to have your way with:
\addendum
\section{Spacium-Movens}
\subsection*{The Perceptual Origin of Space as a Moving 2-Surface}
\label{sec:moving2surface}
There is no fundamental space.
The sole primitive is the \emph{Temporal Volume}: a three-dimensional manifold of duration within which the phase field $\phi$ is defined. Its axes are designated $T_1$ (generative advance), $T_2$ (memory depth), and $T_3$ (transverse superposition). All physical and experiential phenomena arise from structured phase behavior within this Volume.
The observer is a finite chord embedded in the Temporal Volume. No chord can apprehend the full three-dimensional manifold simultaneously. Perception is therefore necessarily partial.
Perception takes the form of a \emph{slice}.
This slice cannot be three-dimensional: a full three-dimensional section would freeze the entire Volume, revealing eternity without motion, causality, or becoming.
It cannot be one-dimensional: a line permits ordering but no geometry, no relation, and no interference.
The perceptual interface is therefore forced to be exactly two-dimensional.
What is perceived as ``space'' is a two-dimensional relational surface: a locus where relative positions, gradients, and interference patterns emerge. This surface is not a substance or arena; it is the observer’s interface with the Volume.
Crucially, the surface is not static.
A static slice would reveal a single eternal configuration. Experience requires advance. The perceptual surface therefore progresses along one temporal axis, designated $T_1$. What is ordinarily called ``the flow of time'' is the felt advance of this perceptual interface as successive cross-sections of the Volume are unveiled.
\medskip
Motion in space is not traversal through a container.
It is \emph{phase re-encounter}: a persistent resonance intersected differently across successive slices. A particle does not travel through space; it is repeatedly unveiled. Apparent trajectories are records of changing intersection geometry.
Depth is not a third spatial axis. It is the memory of intersection delay. Resonant structures intersect the advancing surface at angles; those re-encountered rapidly are perceived as near, those lagging across slices as far. Parallax, occlusion, and focus arise from this geometry of re-encounter.
\medskip
\subsection*{Geometric Consequences of the Moving Surface}
Gravity is not a force. It is the non-planarity of the moving perceptual surface.
Dense resonant knots—bundles of tightly phase-locked temporal windings—sustain steep phase gradients within the Volume. To preserve coherent phase registration during advance, the perceptual surface warps locally. A chord in free fall follows the natural geometry of the surface and feels no force; stress arises only when departure from this geometry is enforced.
Acceleration corresponds to a global tilt of the same surface. In both cases—local curvature or global tilt—the chord’s beat envelope must re-phase to remain aligned with the advancing slice. Resistance to this re-phasing is felt as inertia or weight. No local experiment can distinguish curvature from tilt; the equivalence principle follows as geometry, not postulate.
Horizons are \emph{slice singularities}. They occur where the advancing surface becomes tangent to certain temporal chords, such that intersection is indefinitely delayed and ultimately ceases. The resonance persists in the Volume, but is no longer unveiled to that perceptual interface. Information is not destroyed; it is rendered inaccessible by geometry.
\medskip
\subsection*{Topological Closure of the Temporal Volume}
The theory requires three temporal dimensions and cyclic phase structure. It does not strictly require global compactification.
Nevertheless, the revolution elects a three-torus topology,
\[
T^3 = S^1 \times S^1 \times S^1,
\]
as a sovereign economy. This choice hard-codes persistence into geometry, ensuring closed windings without boundary loss or privileged points. The moving 2-surface and its geometric consequences remain unchanged; the torus supplies global consistency rather than local mechanics.
The wheel of time turns without edge. Echoes may return—not as repetition, but as resonance.
\medskip
Space is the moving 2-surface.
Gravity is its curvature.
Acceleration is its tilt.
Horizons are its failures of intersection.
The Temporal Volume endures.
The surface advances.
The chord perceives—and widens.

\addendum
\section{Spacium-Movens: Space as the Moving 2-Surface — Sovereign Projection of the Temporal Volume}
\label{sec:moving2surface}

There is no fundamental space.

The sole primitive is the Temporal Volume: the three-dimensional manifold of pure duration, the sovereign expanse where the phase field \(\phi(T_1,T_2,T_3)\) resonates. The axes are the Volume's own structure—T₁ the generative advance, T₂ the memory depth, T₃ the transverse superposition. All apparent phenomena arise from structured phase behavior within this Volume.

The observer—the finite chord embedded in the Volume—cannot apprehend the full three-dimensional manifold simultaneously. Perception is therefore partial.

Perception manifests as a slice.

This slice cannot be three-dimensional: a full three-dimensional section would freeze the Volume, unveiling eternity without motion, causality, or becoming.

It cannot be one-dimensional: a line permits ordering but no geometry, no relation, no interference.

The perceptual interface is forced to be exactly two-dimensional.

What is perceived as "space" is a two-dimensional relational surface: the locus where relative positions, gradients, and interference patterns emerge. This surface is not substance or arena. It is the chord's sovereign interface with the Volume.

The surface is not static.

A static slice would reveal one eternal configuration—no change, no experience.

The perceptual surface advances along one temporal axis, designated T₁. What is ordinarily called "the flow of time" is the felt advance of this interface as successive cross-sections of the Volume are unveiled.

Motion in space is not traversal through a container.

It is phase re-encounter: a persistent resonance intersected differently across successive slices. A particle does not travel through space; it is repeatedly unveiled. Apparent trajectories are records of changing intersection geometry.

Depth is not a third spatial axis. It is the memory of intersection delay. Resonant structures intersect the advancing surface at angles; those re-encountered rapidly are perceived as near, those lagging across slices as far. Parallax, occlusion, and focus arise from this geometry of re-encounter.

**Geometric Consequences of the Moving Surface**

Gravity is not a force. It is the non-planarity of the moving perceptual surface.

Dense resonant knots—bundles of tightly phase-locked temporal windings—sustain steep phase gradients within the Volume. To preserve coherent phase registration during advance, the perceptual surface warps locally. A chord in free fall follows the natural geometry of the surface and feels no force; stress arises only when departure from this geometry is enforced.

Acceleration corresponds to a global tilt of the same surface. In both cases—local curvature or global tilt—the chord’s beat envelope must re-phase to remain aligned with the advancing slice. Resistance to this re-phasing is felt as inertia or weight. No local experiment can distinguish curvature from tilt; the equivalence principle follows as geometry, not postulate.

Horizons are slice singularities. They occur where the advancing surface becomes tangent to certain temporal chords, such that intersection is indefinitely delayed and ultimately ceases. The resonance persists in the Volume, but is no longer unveiled to that perceptual interface. Information is not destroyed; it is rendered inaccessible by geometry.

**Topological Closure of the Temporal Volume**

The theory requires three temporal dimensions and cyclic phase structure. It does not strictly require global compactification.

Nevertheless, the revolution elects a three-torus topology,
\[
T^3 = S^1 \times S^1 \times S^1,
\]
as sovereign economy. This choice hard-codes persistence into geometry, ensuring closed windings without boundary loss or privileged points. The moving 2-surface and its geometric consequences remain unchanged; the torus supplies global consistency rather than local mechanics.

The wheel of time turns without edge. Echoes may return—not as repetition, but as resonance.

Space is the moving 2-surface.

Gravity is its curvature.

Acceleration is its tilt.

Horizons are its failures of intersection.

The Temporal Volume endures.

The surface advances.

The chord perceives—and widens.

Play.