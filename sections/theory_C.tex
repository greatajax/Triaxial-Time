\section{C. Mapping to Standard Relativistic Field Theory}

\subsection{Recovering special relativity}

To connect the triaxial time picture with ordinary special relativity, we
specialize to the case of a flat, unsheared membrane. Concretely:

\begin{itemize}
    \item The membrane \(\Sigma\) is a flat subspace defined by
    \(n^A = 0\), with constant normal vectors in \(\mathbb{R}^3_{\bm{T}}\).
    \item The effective time direction \(U^a\) is constant and aligned
    with one of the Cartesian axes, say \(T_1\).
\end{itemize}

In this regime, the projection
\(\pi: (\mathbf{x}, \bm{T}) \mapsto (t, \mathbf{x})\) reduces to
\(t = T_1/c\), and the induced metric on \(\Sigma\) is simply the Minkowski
metric
\begin{equation}
    ds^2 = -c^2 dt^2 + d\mathbf{x}^2.
\end{equation}

Field modes localized in the normal directions obey effective equations that
reduce to the standard relativistic wave equations for fields on Minkowski
spacetime. Lorentz invariance arises as the symmetry group of the induced
metric, independent of the choice of coordinates in the embedding
\(\mathcal{E}\).

Thus, in the absence of shear and curvature, the triaxial time framework
exactly reproduces special relativity.

\subsection{Relation to general relativity}

In general relativity, gravity is encoded in the curvature of a four-dimensional
Lorentzian manifold \((M, g_{\mu\nu})\). In the present framework, the
induced metric on the membrane \(\Sigma\) plays the role of \(g_{\mu\nu}\),
while the embedding of \(\Sigma\) in \(\mathcal{E}\) provides additional data
in the form of extrinsic curvature.

Working at a formal level, we can write the effective Einstein equations on
\(\Sigma\) as
\begin{equation}
    G_{\mu\nu}[g] = 8\pi G\,T_{\mu\nu}^\text{eff},
\end{equation}
where the effective stress-energy tensor \(T_{\mu\nu}^\text{eff}\) includes
both matter contributions from the projected fields \(\psi_k\) and geometric
contributions from the extrinsic curvature and shear of the membrane.

This is reminiscent of brane-world scenarios, but with a crucial difference:
the extra dimensions here are temporal rather than spatial, and the metric in
the temporal sector is chosen to avoid multiple physical time-like directions
on the membrane itself.

In a low-shear regime, the extrinsic contributions can be small, and one
recovers ordinary general relativity to high accuracy. In highly curved or
sheared regions, deviations from Einstein gravity can occur, potentially
providing phenomenological handles on the triaxial temporal structure.

\subsection{Quantum theory as distribution along \(T_3\)}

Quantum theory introduces complex amplitudes over configuration space and
time. In the triaxial time picture, the formal role of these amplitudes is
played by distributions over the modal axis \(T_3\).

Consider a state localized at spatial position \(\mathbf{x}\) and with some
spread along \(T_3\). Denote the corresponding field configuration as
\(\phi(\mathbf{x}, T_1, T_2, T_3)\). The probability amplitudes in standard
quantum mechanics correspond to the coefficients of a decomposition along
\(T_3\),
\begin{equation}
    \phi(\mathbf{x}, \bm{T}) =
    \int dT_3\,\psi(\mathbf{x}, T_1, T_2; T_3)\,\chi(T_3),
\end{equation}
with the normalization condition
\begin{equation}
    \int dT_3\,|\psi(\mathbf{x}, T_1, T_2; T_3)|^2 = 1
\end{equation}
corresponding to conservation of total modal weight.

The unitary time evolution of the quantum state is then reinterpreted as the
deterministic evolution of the full field \(\phi\) along the effective time
direction \(U^a\), with the Born rule arising as a measure over cross-sections
of the \(T_3\) distribution when a process forces localization in that
direction.

This is not yet a complete quantum theory. However, it suggests a program:
translate standard quantum dynamics into constraints on how mode-structures
may be distributed and evolve along the modal axis \(T_3\), subject to the
geometric and causal structure introduced earlier.

\subsection{Internal symmetries from temporal rotations}

Internal symmetries, such as those associated with gauge groups, can be
encoded as rotations and deformations within the temporal sector. For
example, a simple \(U(1)\) symmetry may be associated with phase rotations
that correspond to rotations in a plane embedded in the \(T_2\)-\(T_3\)
subspace.

Higher non-Abelian gauge symmetries can arise from more elaborate isometries
of the temporal metric \(\delta_{ab}\) that leave the membrane structure
invariant. The corresponding gauge fields appear as components of the
connection describing how the effective time direction \(U^a\) and the
normal vectors twist across space.

This picture parallels certain Kaluza--Klein constructions but assigns a
distinct ontological role to the extra dimensions as temporal rather than
compact spatial directions.
