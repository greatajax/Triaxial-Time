\section*{Appendix H: Psychological Phenomena in T1--T2--T3 Resonance Dynamics}
\addcontentsline{toc}{section}{Appendix H: Psychological Phenomena in T1--T2--T3 Resonance Dynamics}

This appendix reframes core psychological processes within the temporal geometry of the
triaxial time formalism. The purpose is not reduction but clarification: the lived structures
of mind follow directly from the theory's harmonic architecture.

\subsection*{H.1 \; T2 as the Harmonic Record of Past Interaction}

T2 is the stable harmonic scaffold produced by accumulated experience. It does not store
memories as discrete entries; it stores the \emph{form} etched by repeated interactions.
Reinforced modes become low-energy attractors, manifesting psychologically as habit,
intuition, trauma, skill, and personality.

\subsection*{H.2 \; T3 as the Field of Forward Possibility}

T3 is not a record but a lattice of possible forward extensions: a high-frequency network
encoding the system's potential futures. Psychologically, T3 corresponds to imagination,
anticipation, creative recombination, anxiety, strategy, and the organism's sense that the
world could unfold otherwise. Where T2 is sculpted by what has happened, T3 is shaped by
what could happen.

\subsection*{H.3 \; T1 as Active Probing}

T1 is the organism's probing axis. A T1 probe into T2 returns the lowest-energy mode
compatible with the present situation; a probe into T3 returns an anticipatory vector field.
Emotion is the former; motivation and foresight are the latter.

\subsection*{H.4 \; T1--T2 and T1--T3 Return Signals}

Emotion corresponds to the return signal from a T1~$\rightarrow$~T2~$\rightarrow$~T1 loop:
a stabilization based on past reinforcement. Anticipation corresponds to the
T1~$\rightarrow$~T3~$\rightarrow$~T1 loop: a mapping of likely, dangerous, or promising futures.
This distinction clarifies why feelings grounded in memory differ from feelings oriented
toward possibility.

\subsection*{H.5 \; Automaticity and Creativity}

Automatic reactions are low-energy T2 modes preferentially excited by T1. Creative insights
arise from low-barrier T3 couplings: forward branches whose resonance thresholds have dropped
sufficiently to be excited spontaneously. What appears as ``an idea out of nowhere'' is
T1 falling into an available T3 channel.

\subsection*{H.6 \; Dysregulation: Rumination and Over-Prediction}

Psychological distress arises when T3 overwhelms T2. Excessive T3 excitation yields unstable
anticipatory branching, experienced as rumination, dread, or runaway prediction. Therapeutic
processes seek to reinforce T2's stabilizing modes while pruning T3 overgrowth.

\subsection*{H.7 \; Introspection as Tri-Axial Reshaping}

Introspection is the deliberate re-excitation and restructuring of both the T2 and T3
landscapes. It requires injecting work along T1 against entrenched resonance gradients. This
corresponds to the lived effort of reframing the past, altering habitual emotional modes,
and reshaping future expectations.

\subsection*{H.8 \; Unified Psychological Map}

\begin{center}
\begin{tabular}{lll}
\textbf{Component} & \textbf{Function} & \textbf{Psychological Expression} \\
\hline
T2 & Past-shaped harmonic scaffold & Memory, habit, emotion origins \\
T3 & Future-shaped possibility field & Imagination, anticipation, creativity, worry \\
T1 & Probing and excitation axis & Attention, will, intention \\
T1$\rightarrow$T2 return & Mode selection & Emotion \\
T1$\rightarrow$T3 return & Branch likelihood & Hope, dread, strategic vision \\
T1 coordination & Probe harmonization & Stability, agency, introspection \\
\end{tabular}
\end{center}

Together these axes yield the organism's complete psychological architecture: past form,
present excitation, and future possibility.
