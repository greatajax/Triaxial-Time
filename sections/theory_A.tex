\section{A. Geometry of Triaxial Time}

\subsection{The extended temporal manifold}

Let \(M\) denote the usual \(3+1\)-dimensional spacetime manifold of standard
relativistic physics, with local coordinates \((t, \mathbf{x}) = (t, x^i)\),
\(i = 1,2,3\), and with a Lorentzian metric \(g_{\mu\nu}\).

In the present framework we replace the single time coordinate \(t\) with a
three-component temporal vector
\begin{equation}
    \bm{T} = (T_1, T_2, T_3) \in \mathbb{R}^3.
\end{equation}
The extended kinematic arena is then
\begin{equation}
    \mathcal{E} = \mathbb{R}^3_{\mathbf{x}} \times \mathbb{R}^3_{\bm{T}},
\end{equation}
equipped with a metric that is locally the product of a spatial metric and a
positive-definite metric on the temporal sector:
\begin{equation}
    ds^2 = h_{ij}\,dx^i dx^j - c^2\,\delta_{ab}\,dT^a dT^b,
\end{equation}
where \(h_{ij}\) may be taken to be Euclidean at small scales and \(a,b\)
index the temporal coordinates.

We deliberately \emph{do not} identify any of the \(T_a\) individually with
the physical time coordinate of ordinary relativity. Instead, the physical
time direction emerges as a \emph{derived} one-dimensional subspace in the
three-dimensional temporal manifold.

\subsection{The spacetime membrane}

An observer does not have direct access to all three temporal axes. Rather,
physical processes are confined to a four-dimensional submanifold
\(\Sigma \subset \mathcal{E}\) of codimension two in the temporal sector. At
each point \((\mathbf{x}, \bm{T})\in \Sigma\), there is:

\begin{itemize}
    \item A \emph{tangent direction} in the temporal space, identified with
    the effective physical time direction \(u^a(\mathbf{x}, \bm{T})\).
    \item Two \emph{normal directions} in the temporal space, which we
    interpret as directions along which the underlying mode-structure can
    extend without being directly available to the observer.
\end{itemize}

Locally, one can choose coordinates so that the membrane is specified by
constraints of the form
\begin{equation}
    \Phi_\alpha(\mathbf{x}, \bm{T}) = 0, \quad \alpha = 1,2,
\end{equation}
with the effective physical time coordinate given by a parameter \(\tau\)
along the integral curves of the vector field
\begin{equation}
    U = U^a \frac{\partial}{\partial T^a}
\end{equation}
tangent to \(\Sigma\) in the temporal sector.

In this language, standard spacetime is the image of \(\Sigma\) under the
projection
\begin{equation}
    \pi: \Sigma \to M, \qquad
    \pi(\mathbf{x}, \bm{T}) = (t(\bm{T}), \mathbf{x}),
\end{equation}
for some function \(t: \mathbb{R}^3_{\bm{T}} \to \mathbb{R}\) that extracts
the scalar ``clock time'' from the triaxial temporal coordinates.

\subsection{Interpretation of the axes}

While the theory does not require a unique psychological interpretation of
the three axes, it is useful to have a working picture:

\begin{itemize}
    \item \(T_1\): the direction approximately aligned with the parameter
    measured by physical clocks; it represents the advance of events.
    \item \(T_2\): the direction in which persistent structures---records,
    correlations, and memories---extend, giving them effective temporal
    thickness.
    \item \(T_3\): the direction along which near-by alternative evolutions
    and modal branches are parameterized; quantum amplitudes can be thought of
    as distributions across this axis.
\end{itemize}

The effective physical time direction \(U^a\) at a point is, in general,
a linear combination of these axes. Its orientation reflects the relative
weight that the local mode-structure places on persistence versus
exploration. A fully deterministic classical trajectory corresponds to a mode
that is sharply localized in \(T_3\) and extended primarily along a
\(T_1\)-\(T_2\) plane, while a quantum-coherent process has a nontrivial
spread along \(T_3\).

\subsection{Shear and curvature in temporal space}

The geometry of the membrane is controlled by how the physical time direction
\(U^a\) and the normal directions vary with position \(\mathbf{x}\) and
temporal location \(\bm{T}\). The key quantities are:

\begin{itemize}
    \item \textbf{Temporal curvature:} variations of the normal vectors as
    one moves along spatial directions, analogous to curvature of an embedded
    surface.
    \item \textbf{Temporal shear:} gradients in the orientation of \(U^a\)
    across space, which we will later see can be associated with gravitational
    and gauge-like fields.
\end{itemize}

Informally, gravity arises from the way the membrane bends and shears in the
three-dimensional temporal space. A freely falling particle follows a
geodesic of the induced metric on \(\Sigma\); but that geodesic, when viewed
in the embedding space \(\mathcal{E}\), may correspond to a nontrivial motion
through the temporal manifold.

The rest of the paper develops these geometric ideas into a dynamical
framework consistent with known physics.
