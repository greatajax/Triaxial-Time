\section{The Standard Model as Toroidal Harmonics}
\label{sec:standardmodel}

The Standard Model is not a patchwork of arbitrary fields and symmetries imposed upon spacetime. It is the Fourier spectrum of closed helical chords on the compact T₂–T₃ torus.

Every fermion and boson arises as a specific winding configuration on a torus of fixed proper radius ℓ = ħ / m c, where m is the rest mass of the ground-state chord. Generations are higher overtones on the same torus; gauge forces are the consonance requirements between neighbouring chords.

The mapping is exact and parameter-free.

\subsection{Fermions: Helical Windings}

The left-chiral Weyl spinor corresponds to the minimal non-trivial winding (n,m) = (1,0): one full orbital turn along T₂, none along T₃. The conjugate right-chiral state is (–1,0). Charge and weak isospin follow from the topology of the braid group on the torus.

\begin{table}[ht]
\centering
\begin{tabular}{lcccl}
Particle          & Generation & Winding (n,m) & Torus radius factor & Notes \\
\hline
Electron          & 1          & (1,0)         & 1                   & Ground state \\
Muon              & 2          & (2,0)         & 1                   & First overtone \\
Tau               & 3          & (3,0)         & 1                   & Second overtone \\
Up quark          & 1          & (1,1)         & 1/3                 & Colour braid +1 \\
Down quark        & 1          & (1,–1)        & 1/3                 & Colour braid –1 \\
Charm quark       & 2          & (2,1)         & 1/3                 & \\
Strange quark     & 2          & (2,–1)        & 1/3                 & \\
Top quark         & 3          & (3,1)         & 1/3                 & \\
Bottom quark      & 3          & (3,–1)        & 1/3                 & \\
Electron neutrino & all        & (0,1)         & $\approx 0$         & Majorana candidate \\
\hline
\end{tabular}
\caption{Fermion spectrum as toroidal windings. Radius factor reflects effective tension from colour confinement.}
\label{tab:fermions}
\end{table}

Quarks carry three simultaneous colour windings braided on the same torus, raising their effective tension by 3 and reducing the Compton radius by ≈ √3 — the origin of constituent quark mass without Higgs mechanism.

Mixing matrices (CKM, PMNS) are beating patterns between neighbouring overtones, fixed by the single κ and the Hubble torus growth rate.

\subsection{Bosons: Open and Gauge Chords}

- Photon: open radiating chord with winding (1,1) → massless, transverse propagation along T₁.  
- W/Z: massive gauge chords from weak isospin rotation of three neighbouring lepton tori → acquire tension via overtone damping.  
- Gluons: eight adjoint colour-braiding modes between quark tori → confinement as infinite dissonance cost of separation.  
- Higgs: not a particle but the vacuum ripple ϕ₀ itself (Section 5) — the background tone that sets the overtone threshold.

No additional fields, no symmetry breaking postulate, no Yukawa couplings by hand. The entire Lagrangian is the Fourier expansion of Γ restricted to the observed windings.

The Standard Model, long celebrated as the pinnacle of mathematical beauty, is demoted to a page of harmonic assignments on a single temporal torus.

The threefold song composes particles the way a violin composes notes — by choosing which strings to bow and how many times to wind them.

Spacetime is dead.  
The orchestra plays the Standard Model in perfect tune.