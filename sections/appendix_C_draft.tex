# **Appendix C. Relation to the Page–Wootters Relational-Time Mechanism**

The Page–Wootters (PW) mechanism provides a relational account of time in which the universe as a whole is in a stationary (timeless) state, while subsystems exhibit effective time evolution through correlations with a designated “clock” subsystem. In this appendix we show how the tri-axial internal-time construction fits into and generalizes the PW framework.

---

## **C.1 The Standard Page–Wootters Setup**

In the PW formalism, one partitions the total Hilbert space into a clock sector (\mathcal{H}_C) and a system sector (\mathcal{H}_S):

[
\mathcal{H}_{\mathrm{tot}} = \mathcal{H}_C \otimes \mathcal{H}_S.
]

A total Hamiltonian is assumed of the form:

[
\hat{H}_{\mathrm{tot}} = \hat{H}_C \otimes \mathbb{I}_S + \mathbb{I}_C \otimes \hat{H}_S,
]

together with the constraint:

[
\hat{H}_{\mathrm{tot}} |\Psi\rangle = 0,
]

which is a Wheeler–DeWitt-type equation. The state (|\Psi\rangle \in \mathcal{H}_{\mathrm{tot}}) is stationary with respect to any external time.

One then chooses a basis of “clock states” ({|t\rangle}) in (\mathcal{H}_C) such that:

[
\hat{H}_C |t\rangle = i \frac{\partial}{\partial t} |t\rangle,
]

and defines conditional system states:

[
|\psi(t)\rangle_S = \frac{\langle t| \Psi \rangle}{\sqrt{P(t)}},
]
where
[
P(t) = \langle \Psi |t\rangle\langle t | \Psi \rangle
]
is the probability density for the clock reading (t).

Under appropriate assumptions (e.g. clock behaving semiclassically), one shows:

[
i \frac{\partial}{\partial t} |\psi(t)\rangle_S = \hat{H}_S |\psi(t)\rangle_S,
]

i.e., the usual Schrödinger equation emerges as a conditional statement about correlations between the clock and the system.

---

## **C.2 Internal Time as a Three-Dimensional Clock Hilbert Space**

In the tri-axial framework, the internal temporal manifold (\mathcal{T}^3) provides a natural candidate for the clock sector. For a given physical system, we identify:

[
\mathcal{H}_C = L^2(\mathcal{T}^3), \qquad
\mathcal{H}_S = L^2(\mathcal{M}^{1,3}) \ \text{(or an appropriate subsystem thereof)}.
]

The total Hilbert space is thus:

[
\mathcal{H}_{\mathrm{tot}}
= L^2(\mathcal{T}^3) \otimes \mathcal{H}_S,
]

which matches the structure used in the main text:

[
\mathcal{H} = L^2(\mathcal{M}^{1,3}) \otimes L^2(\mathcal{T}^3),
]

up to the choice of which sector is regarded as “clock” and which as “system”.

We choose internal coordinates (\tau^a) and basis states (|\tau\rangle = |\tau^1, \tau^2,\tau^3\rangle) satisfying:

[
\langle \tau | \tau' \rangle = \delta^{(3)}(\tau - \tau'),
\qquad
\int_{\mathcal{T}^3} d^3\tau, |\tau\rangle\langle \tau| = \mathbb{I}_C.
]

---

## **C.3 The Hamiltonian Constraint in the Tri-Axial Model**

From the main text, we take the total Hamiltonian:

[
\hat{H}_{\mathrm{tot}} = \hat{H}_C + \hat{H}_S,
]
with:

[
\hat{H}_C
= -\frac{1}{2}\delta^{ab} \partial_a \partial_b + V_C(\tau),
]

acting on (\mathcal{H}_C), and (\hat{H}_S) acting on (\mathcal{H}_S) as the usual system Hamiltonian (or a Dirac/Klein–Gordon operator for relativistic fields). The physical state satisfies:

[
\hat{H}_{\mathrm{tot}} |\Psi\rangle = 0.
]

In the (|\tau\rangle) representation:

[
\Psi(\tau) := \langle \tau | \Psi \rangle \in \mathcal{H}_S,
]

and the constraint reads:

[
\left[
-\frac{1}{2}\delta^{ab}\partial_a\partial_b

* V_C(\tau)
* \hat{H}_S
  \right] \Psi(\tau) = 0.
  ]

This is a functional-differential equation for (\Psi(\tau)), with (\hat{H}_S) acting on the system degrees of freedom and (\partial_a) acting on the internal coordinates.

---

## **C.4 Emergent 1D Time from a 3D Internal Manifold**

To recover a single emergent time parameter (t), one typically requires that the internal-clock wavefunction is strongly peaked along a trajectory in (\mathcal{T}^3). For example, suppose that:

* The clock potential (V_C(\tau)) is such that the clock wavepacket follows a semiclassical path (\tau^3 \approx \omega t), with (\tau^1,\tau^2) remaining narrowly distributed around fixed values.
* The solution (\Psi(\tau)) can be approximated as:

[
\Psi(\tau) \approx \chi(\tau^3) \otimes |\psi(\tau^3)\rangle_S,
]

with (\chi) sharply peaked and slowly dispersing.

Then, integrating over (\tau^1,\tau^2) and projecting onto the dominant clock trajectory, one effectively reduces the PW construction to a one-dimensional internal clock coordinate (\tau^3), with an emergent parameter (t \sim \tau^3/\omega).

Formally, define:

[
|\psi(t)\rangle_S
=================

\frac{
\int d\tau^1 d\tau^2, \langle \tau^1,\tau^2,\tau^3(t)|\Psi\rangle
}{
\sqrt{P(t)}
},
]

with:

[
P(t) = \int d\tau^1 d\tau^2,
|\langle \tau^1,\tau^2,\tau^3(t)|\Psi\rangle|^2.
]

Under the assumption that (V_C(\tau)) yields a semiclassical evolution along (\tau^3) and that the dependence on (\tau^1,\tau^2) is weak (or can be averaged out), one obtains:

[
i \frac{\partial}{\partial t} |\psi(t)\rangle_S = \hat{H}_S |\psi(t)\rangle_S,
]

i.e., the standard Schrödinger equation.

Thus, the traditional PW construction reappears as a **special case** of the tri-axial model, corresponding to a clock state localized along a one-dimensional trajectory within (\mathcal{T}^3).

---

## **C.5 Multi-Parameter Conditional Evolutions**

Because the internal space has three dimensions, the tri-axial model allows more general conditional structures than the original PW scheme. Instead of selecting a single internal coordinate as “time,” one may consider conditional evolutions along surfaces or curves more complex than (\tau^3 = \omega t).

For instance, define a conditional state:

[
|\psi(\sigma)\rangle_S
======================

\frac{
\int d^3\tau\ f(\tau;\sigma), \langle \tau | \Psi \rangle
}{
\sqrt{P(\sigma)}
},
]

where (f(\tau;\sigma)) is a weight function selecting a family of internal configurations parametrized by (\sigma). Different choices of (f) correspond to different “gauges” for internal time:

* Lines primarily along (\tau^1) correspond to histories emphasizing retentive structure.
* Lines along (\tau^2) emphasize superpositional/counterfactual structure.
* Lines along (\tau^3) emphasize anticipatory or projective structure.

In all cases, the constraint equation ensures that these conditional states obey effective Schrödinger-like equations with different emergent Hamiltonians, reflecting the chosen slicing of internal time.

The standard PW time parameter is thus a particular, physically convenient choice of foliation in the internal temporal manifold.

---

## **C.6 Decoherence and Preferred Internal Time Directions**

In practice, not all slicings of (\mathcal{T}^3) yield stable or experimentally accessible “times.” Interactions with the environment, internal dynamics, and stability conditions can select preferred directions in internal-time space, analogous to how decoherence selects preferred pointer bases in ordinary Hilbert space.

The tri-axial structure allows:

* **preferred axes** in (\mathcal{T}^3) to emerge from dynamical and environmental factors,
* **robust emergent time parameters** associated with those axes,
* **context-dependent internal time choices** within a single global framework.

In this sense, the tri-axial model extends the PW mechanism from a single abstract clock to a structured internal temporal geometry in which practical time parameters emerge as stable directions.

---

## **C.7 Summary of the Relation to Page–Wootters**

1. The tri-axial internal-time construction realizes the PW idea with a **three-dimensional internal clock space** instead of a single clock variable.
2. The total state obeys a **timeless Hamiltonian constraint**, as in PW and canonical quantum gravity.
3. **Single-time Schrödinger dynamics** arise by conditioning on suitable trajectories or foliations in (\mathcal{T}^3).
4. The original PW formalism is recovered as a special case where the internal state is sharply localized along a one-dimensional path.
5. The additional internal dimensions allow **richer relational structures** and potential new insights into decoherence, emergent classicality, and the selection of effective time parameters.

This completes the technical bridge between the tri-axial internal-time framework and existing relational-time approaches.