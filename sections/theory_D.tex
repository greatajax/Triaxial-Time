\section{D. Empirical Constraints and Observational Windows}

\subsection{Compatibility with existing tests}

Any extension of the temporal structure of physics is severely constrained by
the empirical success of special and general relativity and of quantum field
theory. At a minimum, the triaxial time model must satisfy:

\begin{itemize}
    \item No detectable violation of Lorentz invariance in regimes where it
    has been well tested.
    \item No gross modification of gravitational dynamics at solar system and
    binary-pulsar scales.
    \item No conflict with high-energy collider experiments and precision
    quantum tests.
\end{itemize}

In the present framework, these constraints are met by requiring that, on
accessible scales:

\begin{enumerate}
    \item The membrane \(\Sigma\) is nearly flat in the temporal sector, with
    small extrinsic curvature relative to experimentally probed lengths.
    \item Shear in the effective time direction \(U^a\) is mild and slowly
    varying, such that its contribution to the effective dynamics can be
    renormalized into standard couplings.
\end{enumerate}

This is analogous to how many higher-dimensional models evade low-energy
constraints by confining accessible physics to a nearly flat submanifold.

\subsection{Where deviations might appear}

If the triaxial temporal structure is real, it should manifest in regimes
where the membrane is significantly curved or sheared in the temporal sector,
or where mode-structures become highly extended in \(T_2\) or \(T_3\).

Candidate regimes include:

\begin{itemize}
    \item \textbf{Strong gravity:} near black holes or in the early universe,
    extrinsic curvature in the temporal sector could be large, leading to
    deviations from classical general relativity in the form of modified
    horizon structure, evaporation dynamics, or singularity resolution.
    \item \textbf{Macroscopic quantum coherence:} systems that maintain
    coherent superpositions across long durations (in \(T_1\)) and over many
    degrees of freedom may probe the structure of the modal axis \(T_3\).
    \item \textbf{Low-energy anomalies:} certain persistent anomalies in
    cosmology or astrophysics (e.g. dark energy phenomenology, long-range
    correlations) might conceivably be reinterpreted as manifestations of
    temporal shear rather than new matter components.
\end{itemize}

At this stage the discussion is necessarily qualitative. The purpose of this
section is less to claim concrete predictions than to delineate where a
triaxial temporal ontology would be most plausibly testable.

\subsection{Conceptual predictions}

The model also makes conceptual predictions about how certain puzzles should
eventually be resolved:

\begin{itemize}
    \item \textbf{Arrow(s) of time:} rather than a single thermodynamic
    arrow aligned with a unique time direction, there should be several
    distinct arrows associated with different projections of the mode-structure
    onto the three temporal axes. Entropy increase in physical processes
    corresponds to a particular alignment of these arrows with the effective
    time direction \(U^a\).
    \item \textbf{Quantum measurement:} wavefunction ``collapse'' should be
    replaced by a geometric localization of support along \(T_3\), triggered
    when mode-structures become entangled with sufficiently extended
    structures along \(T_2\) (records, memories).
    \item \textbf{Nonlocal correlations:} spacelike quantum correlations may
    be reinterpreted as local relations in the extended temporal manifold,
    where modes share structure along \(T_2\) and \(T_3\) even when separated
    along \(\mathbf{x}\).
\end{itemize}

These are not yet empirical predictions in the strict sense, but they
constitute constraints on how a completed triaxial time theory would have to
interface with unresolved foundational issues.

