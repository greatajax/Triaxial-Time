---

# **Appendix D. Explicit Gauge-Field Derivations from Internal-Time Isometries**

In this appendix we derive explicit U(1) and SU(2) gauge structures from the isometries of the internal temporal manifold. The key idea is that internal isometries act as symmetries on the field (\Psi(x,\tau)), and their localization in spacetime requires the introduction of gauge connections, which are precisely the familiar gauge fields.

We first treat the Abelian case (U(1)) using a simple rotation in an internal plane, then move to a non-Abelian example (SU(2)) realized as the isometry group of an internal 3-sphere.

---

## **D.1 General Framework: Killing Vectors and Internal Isometries**

Let ((\mathcal{T}, h_{ab})) be the internal temporal manifold with metric (h_{ab}). Its continuous isometries are generated by Killing vector fields (K_A^a(\tau)) satisfying:

[
\nabla_a K_{B,b} + \nabla_b K_{B,a} = 0,
]
where (\nabla_a) is the Levi-Civita covariant derivative on (\mathcal{T}).

These Killing vectors obey a Lie algebra:

[
[K_A, K_B]^a = f_{AB}{}^{C} K_C^a,
]
with structure constants (f_{AB}{}^{C}) characteristic of the isometry group (G_{\mathcal{T}}).

A field (\Psi(x,\tau)) transforms under a global internal isometry as:

[
\delta \Psi(x,\tau) = \epsilon^A K_A^a(\tau), \partial_a \Psi(x,\tau),
]
where (\epsilon^A) are constant parameters.

We now promote (\epsilon^A) to functions of spacetime, (\epsilon^A(x)), and introduce gauge fields so that the theory remains invariant under these **local** internal isometries.

---

## **D.2 U(1) from a Rotation in an Internal Plane**

We begin with a particularly transparent U(1) example. Take the internal manifold to be (\mathcal{T}^3) with coordinates (\tau^1,\tau^2,\tau^3) and flat metric (\delta_{ab}). Consider a rotation in the ((\tau^2,\tau^3))-plane:

[
\begin{pmatrix}
\tau^2 \
\tau^3
\end{pmatrix}
\mapsto
\begin{pmatrix}
\cos\alpha & -\sin\alpha \
\sin\alpha & \cos\alpha
\end{pmatrix}
\begin{pmatrix}
\tau^2 \
\tau^3
\end{pmatrix}.
]

The associated Killing vector field is:

[
K(\tau)
= \tau^2 \frac{\partial}{\partial \tau^3}

* \tau^3 \frac{\partial}{\partial \tau^2}.
  ]

For a small global rotation (\alpha), the field transforms as:

[
\delta \Psi(x,\tau)
= \alpha, K(\tau)\Psi(x,\tau).
]

The internal metric and measure are invariant under this transformation, so the free action remains symmetric.

---

### **D.2.1 Localizing the Rotation: Introducing the U(1) Gauge Field**

To promote (\alpha) to a spacetime-dependent function (\alpha(x)), we define the local transformation:

[
\delta \Psi(x,\tau)
= \alpha(x), K(\tau)\Psi(x,\tau).
]

The spacetime derivative transforms as:

[
\delta(\partial_\mu \Psi)
=========================

# \partial_\mu(\delta\Psi)

(\partial_\mu \alpha(x)), K\Psi

* \alpha(x), K(\partial_\mu\Psi).
  ]

The term with (\partial_\mu\alpha) breaks invariance, as usual. To restore invariance, introduce a spacetime gauge field (A_\mu(x)) and define the covariant derivative:

[
D_\mu \Psi
= \partial_\mu\Psi + g, A_\mu(x), K(\tau)\Psi,
]

where (g) is a coupling constant.

Under a local transformation:

[
\delta\Psi = \alpha(x), K\Psi,
]
we require:

[
\delta A_\mu(x)
= -\frac{1}{g}, \partial_\mu \alpha(x),
]

so that:

[
\delta(D_\mu\Psi)
= \alpha(x) K(D_\mu\Psi).
]

Thus the kinetic term in the Lagrangian:

[
\mathcal{L}*{\mathrm{kin}}
= -\frac12 g^{\mu\nu}(D*\mu\Psi)^\dagger(D_\nu\Psi)
]

remains invariant under local internal rotations.

This is exactly the transformation law of a U(1) gauge field, with:

[
A_\mu \to A_\mu - \frac{1}{g}\partial_\mu \alpha(x).
]

---

### **D.2.2 Field Strength and Maxwell Term**

Define the field strength as the commutator of covariant derivatives:

[
[D_\mu, D_\nu]\Psi
==================

g F_{\mu\nu} K(\tau)\Psi,
]

with:

[
F_{\mu\nu} = \partial_\mu A_\nu - \partial_\nu A_\mu.
]

The gauge-invariant kinetic term for the gauge field is:

[
\mathcal{L}*{\mathrm{gauge}}
= -\frac{1}{4} F*{\mu\nu} F^{\mu\nu},
]

which is precisely the Maxwell Lagrangian.

Hence, a single internal-plane rotation, localized in spacetime, yields the standard structure of a U(1) gauge theory, with:

* matter fields transforming under internal rotations,
* a gauge connection (A_\mu) with standard transformation rule,
* and a Maxwell field strength built from the commutator of covariant derivatives.

---

## **D.3 SU(2) from an Internal S³ Manifold**

To obtain a non-Abelian example, it is natural to consider an internal manifold whose isometry group includes an SU(2) factor. A canonical example is the 3-sphere (S^3), which can be identified with the group manifold of SU(2):

[
\mathcal{T}^3 \cong S^3 \cong SU(2).
]

The left-invariant vector fields on SU(2) generate the Lie algebra (\mathfrak{su}(2)), with structure constants given by the Levi-Civita symbol (\epsilon_{ABC}).

---

### **D.3.1 Killing Vectors on S³ and the SU(2) Algebra**

Let (\tau \in S^3) be a point on the 3-sphere, identified with an SU(2) group element (U(\tau)). The left-invariant vector fields (L_A) act as:

[
L_A f(\tau)
= \left.\frac{d}{d\alpha}\right|_{\alpha=0}
f\big(\exp(i \alpha T_A), \tau\big),
]

where ({T_A}) form a basis of (\mathfrak{su}(2)) (e.g. (T_A = \frac12 \sigma_A), the Pauli matrices). These vector fields satisfy:

[
[L_A, L_B] = f_{AB}{}^{C} L_C,
\qquad f_{AB}{}^{C} = \epsilon_{AB}{}^{C}.
]

The (L_A) are Killing vectors of the round metric on (S^3), generating the left-acting SU(2) isometry.

A field (\Psi(x,\tau)) on (\mathcal{M}^{1,3} \times S^3) transforms under a global SU(2) isometry as:

[
\delta\Psi(x,\tau) = \epsilon^A L_A \Psi(x,\tau),
]
with constant parameters (\epsilon^A).

---

### **D.3.2 Local SU(2) Gauge Symmetry**

To promote this to a **local** SU(2) gauge symmetry, take (\epsilon^A = \epsilon^A(x)). Define the covariant derivative:

[
D_\mu \Psi(x,\tau)
==================

\partial_\mu \Psi(x,\tau)

* g A_\mu^A(x) L_A \Psi(x,\tau),
  ]

where (A_\mu^A(x)) are SU(2) gauge fields, and (g) is a coupling.

Under a local SU(2) transformation:

[
\delta \Psi = \epsilon^A(x) L_A \Psi,
]

we demand:

[
\delta A_\mu^A
= -\frac{1}{g} \partial_\mu \epsilon^A

* f_{BC}{}^{A} A_\mu^B \epsilon^C.
  ]

Then:

[
\delta(D_\mu\Psi)
= \epsilon^A L_A (D_\mu\Psi),
]

and the covariant derivative transforms homogeneously.

---

### **D.3.3 Non-Abelian Field Strength**

The field strength is defined through:

[
[D_\mu, D_\nu] \Psi
===================

g F_{\mu\nu}^A L_A \Psi,
]
with:

[
F_{\mu\nu}^A
============

\partial_\mu A_\nu^A

* \partial_\nu A_\mu^A

- g f_{BC}{}^{A} A_\mu^B A_\nu^C.
  ]

The Yang–Mills Lagrangian is:

[
\mathcal{L}_{\mathrm{YM}}
=========================

-\frac{1}{4} F_{\mu\nu}^A F^{\mu\nu}_A,
]

which is exactly the standard SU(2) gauge kinetic term.

Thus, by taking the internal temporal manifold to be (S^3) with its natural SU(2) isometry, the act of localizing internal isometries yields a full SU(2) gauge theory with conventional transformation properties, field strengths, and interactions.

---

## **D.4 Matter Representations and Charge**

In both the U(1) and SU(2) examples, the effective “charge” of a matter field is determined by how it transforms under internal isometries:

* For U(1), charges are eigenvalues associated with the generator (K(\tau)).
* For SU(2), matter fields transform under representations induced by the action of (L_A).

Explicitly, we can write the action on (\Psi(x,\tau)) as:

[
L_A \Psi(x,\tau)
= (T_A)^{\alpha}{}_{\beta} , \Psi^\beta(x,\tau),
]

where (T_A) are representation matrices acting on an internal index (\alpha). This structure translates straightforwardly into standard gauge-covariant matter sectors after integrating over the internal manifold and projecting onto low-lying modes.

At low energies (with all KK excitations decoupled), the resulting effective theory is indistinguishable from ordinary U(1) or SU(2) gauge theory, including:

* covariant derivatives with representation-dependent couplings,
* gauge-invariant Yukawa and scalar interactions,
* anomaly and renormalization structure inherited from the chosen representations.

---

## **D.5 Summary of Appendix D**

We have shown explicitly that:

1. A rotation in an internal temporal plane yields a **U(1)** gauge field with the standard transformation law and Maxwell field strength.
2. Taking (\mathcal{T}^3) as (S^3 \cong SU(2)), the left-invariant isometries generate a full **SU(2)** non-Abelian gauge theory upon localization.
3. Matter fields carry effective charges and representations determined by their transformation under internal isometries.
4. At low energies, after compactification and mode truncation, the effective gauge sector matches standard Yang–Mills theory.

This provides concrete mathematical support for the claim in the main text: **gauge symmetries emerge naturally from isometries of the tri-axial internal temporal manifold**.

---