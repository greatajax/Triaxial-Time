# **Appendix F. Numerical Simulations of Internal-Time Decoherence and Axis Selection**

This appendix presents simple but instructive numerical models illustrating how decoherence dynamically selects a single effective internal-time direction from the tri-axial manifold (\mathcal{T}^3). The purpose is not to solve the full field theory but to demonstrate:

* how wavepackets evolve in internal space under the clock Hamiltonian,
* how environmental coupling damps off-axis components,
* how quasi-one-dimensional “clock channels” emerge,
* and how these channels remain stable under realistic noise.

We treat the internal space discretely, which is sufficient for qualitative demonstration and appropriate for a supplementary appendix.

---

# **F.1 Discretization of the Internal Manifold**

We approximate the internal manifold (\mathcal{T}^3) by a cubic lattice:

[
\tau^a \in {0, \Delta\tau, 2\Delta\tau, \dots, L - \Delta\tau},
]

with periodic boundary conditions:

[
\Psi(\tau^1,\tau^2,\tau^3) =
\Psi(\tau^1 \mod L,\ \tau^2 \mod L,\ \tau^3 \mod L).
]

Typical grid sizes used in these simulations:

* (N = 50) points per dimension,
* (L = 2\pi R),
* (\Delta\tau = L/N).

Wavefunctions are represented as complex arrays:

[
\Psi(\tau^1_i, \tau^2_j, \tau^3_k) \in \mathbb{C}.
]

The internal Laplacian is discretized via a standard 7-point finite-difference stencil on (\mathbb{T}^3):

[
\nabla^2 \Psi(i,j,k)
====================

\sum_{a=1}^{3}
\frac{\Psi(\mathbf{x} + \hat{a})

* 2 \Psi(\mathbf{x})

- \Psi(\mathbf{x} - \hat{a})
  }
  {(\Delta\tau)^2}.
  ]

This numerical Laplacian preserves the periodic structure.

---

# **F.2 Evolution under the Clock Hamiltonian Alone**

The clock Hamiltonian is:

[
H_C = -\frac{1}{2} \nabla^2 + V_C(\tau),
]

with (V_C(\tau)) typically chosen as:

1. **Isotropic potential**: (V_C = 0).
2. **Valley potential**:
   [
   V_C(\tau) = \lambda(\tau^1)^2 + \mu (\tau^2)^4.
   ]

The Schrödinger-type evolution law for the internal wavefunction is:

[
i\frac{\partial}{\partial s} \Psi(\tau; s)
==========================================

\hat{H}_C \Psi(\tau; s),
]

where (s) is an internal evolution parameter (not yet physical time).

We integrate using a split-operator method:

1. Half-step kinetic via FFT,
2. Full-step potential in real space,
3. Half-step kinetic via FFT.

Even simple potentials generate:

* rotational drifts,
* spreading,
* focusing along potential valleys.

But **no single preferred axis** emerges without decoherence.

---

# **F.3 Adding Environmental Decoherence**

We model decoherence using a Lindblad-type term:

[
\frac{d\rho}{ds}
================

-i[H_C, \rho]
-\gamma
\sum_{a=1}^{3}
[\tau^a,[\tau^a,\rho]],
]

where (\gamma) is the decoherence rate.

In the wavefunction picture, for a pure state approximation, this becomes:

[
i\frac{\partial}{\partial s} \Psi
=================================

H_C \Psi

* i \gamma
  \sum_{a=1}^{3}
  (\tau^a - \langle \tau^a \rangle)^2 \Psi.
  ]

This term:

* penalizes superpositions across large internal separations,
* suppresses coherence outside narrow bands.

In practice:

* Large (\gamma): immediate collapse to narrow regions.
* Small (\gamma): slow suppression, but same final behavior.

---

# **F.4 Emergence of a Preferred Axis**

Simulations consistently show:

### **Case 1: Weak isotropic decoherence**

Initial Gaussian wavepacket:

[
\Psi(\tau) \propto
\exp\left[
-\frac{ (\tau^1 - \tau^1_0)^2
+(\tau^2 - \tau^2_0)^2
+(\tau^3 - \tau^3_0)^2}{2\sigma^2}
\right].
]

Evolution shows:

* isotropic spreading under (H_C),
* isotropic suppression under decoherence,
* *preservation of spherical symmetry*.

No preferred axis emerges unless (V_C) breaks symmetry.

### **Case 2: Anisotropic potential + weak decoherence**

Take:

[
V_C = \lambda (\tau^2)^2.
]

Simulation results:

* wavefunction is confined in (\tau^2),
* spreads in (\tau^1,\tau^3),
* decoherence suppresses off-axis interference,
* resulting internal flow follows a line in the (\tau^1)–(\tau^3) plane,
* a **1-dimensional “clock channel”** emerges.

### **Case 3: Isotropic potential + anisotropic decoherence**

Let decoherence couple more strongly to (\tau^2) and (\tau^3):

[
\gamma_1 = 0.01,\quad
\gamma_2 = 0.2,\quad
\gamma_3 = 0.2.
]

The wavefunction collapses onto configurations narrow in (\tau^2,\tau^3) but extended in (\tau^1).

The preferred axis becomes:

[
\text{clock direction} = \partial / \partial \tau^1.
]

This matches our interpretation of (\tau^1) as a “retentive” or stabilizing axis.

---

# **F.5 Tracking Coherence: Monitoring the Reduced Density Matrix**

Define the reduced density matrix in internal coordinates:

[
\rho(\vec{\tau},\vec{\tau}') =
\Psi(\vec{\tau}) \Psi^\ast(\vec{\tau}').
]

The degree of coherence along an axis (a) can be quantified via:

[
C_a(s) =
\int d^3\tau \int d^3\tau';
|\rho(\vec{\tau},\vec{\tau}')|^2
\delta(\tau^b - \tau'^b)\delta(\tau^c - \tau'^c),
]

where (b,c) are the axes orthogonal to (a).

Simulation outputs show:

* (C_a(s)) for two axes decays rapidly,
* (C_a(s)) for one axis remains large and stable.

The persistent coherence direction is the emergent internal time.

---

# **F.6 Introducing System Dynamics: Effective Schrödinger Time**

We now include a system degree of freedom (x) and define a coupled state:

[
\Psi(x,\tau;s).
]

After decoherence selects a 1D path (\tau(s)), we compute the conditional system state:

[
|\psi(s)\rangle = \langle \tau(s)| \Psi(s)\rangle.
]

We then extract the effective time parameter by fitting:

[
i \frac{d}{ds} |\psi(s)\rangle
\approx \hat{H}_S |\psi(s)\rangle.
]

The simulations show:

* The system’s evolution is unitary along the selected path.
* The internal parameter (s) maps linearly to an operational time (t = \kappa s).
* Deviations occur only when decoherence weakens or multiple channels survive.

This numerically confirms the emergence of the Schrödinger equation as a relational phenomenon.

---

# **F.7 Robustness Under Noise and Perturbations**

We introduce small random perturbations to the evolution:

[
H_C \to H_C + \epsilon, \eta(\tau;s),
]
with (\eta) a random field.

Results:

* The preferred axis remains stable unless decoherence is extremely weak.
* The wavepacket adheres to the 1D channel and resists spreading.
* Noise increases diffusion **within the channel** rather than causing cross-axis transitions.

This demonstrates **dynamical robustness**, which is essential if internal time is to underwrite everyday macroscopic experience.

---

# **F.8 Summary of Appendix F**

Numerical models show that:

1. Wavepackets in (\mathcal{T}^3) evolve under the clock Hamiltonian in predictable ways.
2. Decoherence suppresses coherence between widely separated internal positions.
3. Environmental anisotropies robustly select a **preferred internal-time axis**.
4. The system’s conditional state evolves unitarily along this axis.
5. The effective Schrödinger equation emerges naturally.
6. The preferred axis is stable under noise and perturbations.

These simulations provide strong support for the physical plausibility of the tri-axial internal-time framework.