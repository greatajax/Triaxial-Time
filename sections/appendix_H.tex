%===================================================
\section{Foundations}
%===================================================

\subsection{The Refusal of Spatial Primacy}

There is no pre-existing stage, no empty space waiting to be filled.  
There is only time — three-dimensional time — and within it the self-organising music of harmonics.

\subsection{The Three Axes of Time}

\begin{description}
\item[$T_1$ — Generative Sequence (Universal Tempo)]  
The conductor’s relentless down-beat.  
Each indivisible advance along $T_1$ is one universal pulse that reprints the entire configuration of the $T_2$–$T_3$ plane.  
$T_1$ is experienced as the pure forward thrust of “now → next”.  
It is the source of causality, energy, duration, and the arrow of time.  
$T_1$ has no frequency of its own; it is the tempo against which all frequencies are measured.

\item[$T_2$ — Accumulated Structure (Historical Depth)]  
The axis along which patterns persist and grow.  
Whatever repeats, reinforces, crystallises, or is remembered extends along $T_2$.  
A proton, a granite mountain, a childhood memory, a written law — all possess long coherence length in $T_2$ because their internal resonances have not dispersed very little from one $T_1$-beat to the next.  
$T_2$ is the direction of habit, identity, and inertia.

\item[$T_3$ — Modal Breadth (Simultaneous Alternatives)]  
The axis along which a single harmonic pattern can spread into many closely related but slightly mistuned versions of itself — exactly like a guitar string that has been brushed sideways and momentarily rings with a bundle of neighbouring pitches.  
Quantum superpositions are nothing other than genuine breadth along $T_3$.  
When such a spread-out chord strongly couples with a much stiffer, macroscopically tuned chord (a measuring device, a grain of silver bromide, a retina), the combined system now contains a steep tension gradient across $T_3$.  
Tension relaxation — the same universal drive that makes a buzzing guitar string quickly settle into one clear note — rapidly pulls almost all of the amplitude into the single narrowest, lowest-tension region of $T_3$.  
The generative advance of $T_1$ then continues only that clear note.  
The residual mistuned breadth remains present in the temporal volume but carries negligible amplitude and no longer rings.  
$T_3$ is therefore the axis of possibility, imagination, and free will.
These three axes are jointly continuous, mutually orthogonal in function, and jointly exhaustive.  
No fourth axis is ever required.

\subsection{The Temporal Volume}

The temporal volume $\mathcal{V} = T_1 \times T_2 \ T_3$ is the sole arena of existence.  
Points $(t, s, m) \in \mathcal{V}$ are pure temporal addresses carrying no spatial meaning.

\subsection{Tension}

Tension $\mathcal{T}$ is the single scalar quantity measuring local unresolved dissonance.  
Tension is always $\geq 0$.  
Perfect consonance carries zero tension.  
All dynamics — from radioactive decay to human longing — is tension relaxation subject to conservation laws.

\subsection{Harmonic Modes}

A harmonic mode is any cyclic, self-reinforcing pattern living primarily in the $T_2$–$T_3$ plane.  
Modes possess frequency (rotation rate), amplitude, phase, timbre, and coherence length in $T_2$ and $T_3$.

\subsection{Resonance, Dissonance, Modulation}

\begin{itemize}
  \item \textbf{Resonance} — mutual reinforcement into simple integer ratios; sharply lowers tension.
  \item \textbf{Dissonance} — deviation from simple ratios; raises tension.
  \item \textbf{Modulation} — controlled shifting of frequency, phase, or amplitude.  
        In inanimate systems modulation follows tension gradients;  
        in minds it can be initiated deliberately via $T_3$ exploration.
\end{itemize}

\subsection{The Musical Vocabulary Made Ontological}

\begin{itemize}
  \item A single mode = a tone.
  \item Several simultaneously resonant modes = a chord.
  \item A chord that closes perfectly upon itself = a closed chord (the entity experienced as a massive particle).
  \item A pure circular mode that refuses to close = a carrier (the entity experienced as a photon).
  \item The felt distance between two sub-chords = their interval (tension ratio).
  \item The entire experienced universe at one moment = the instantaneous global chord at a fixed $t \in T_1$.
\end{itemize}

With only these primitives — three-axis time, tension, and harmonic modes — the rest of the theory follows necessarily.  
Space, electromagnetism, gravity, quantum behaviour, and consciousness will all emerge as special cases of the same temporal music.