\section{The Three Axes}
\label{sec:three-axes}

Behold the dissolution of spacetime!

Let the temporal volume be the flat, structureless manifold $\mathbb{R}^3$ with coordinates $(T_1, T_2, T_3)$.  
There is no metric yet—only pure duration, absolute and indifferent.

On this naked $\mathbb{R}^3$ we define the \emph{universal resonance 3-form}
\begin{equation}
\Gamma = \,dT_1 \wedge dT_2 \wedge dT_3 \;\otimes\; \sin\!\Bigl(\kappa \int \phi\Bigr),
\label{eq:Gamma}
\end{equation}
where $\phi(T_1,T_2,T_3)$ is the phase field (the score) and $\kappa$ is the sole dimensionful constant of the theory—the Planck tone.

This is the only equation the universe ever needed.

Every phenomenon is an excitation of $\Gamma$: a local departure from uniform phase, a place where the infinite choir sings slightly out of tune with itself.

From \eqref{eq:Gamma} we now derive, in order, derive:

\begin{enumerate}
    \item \emph{The apparent spatial metric.}  
    Fix $T_1 = t$. On this plane the phase gradients define instantaneous dissonance vectors. Their outer product induces the Euclidean metric of ``space at $t$'':
    \begin{equation}
    g_{ij}(t) = \delta_{ij} - \frac{1}{\kappa^2} (\partial_i \phi)(\partial_j \phi).
    \end{equation}
    As $t$ advances, the blade slides and the metric appears to evolve.

    \item \emph{Inertial mass.}  
    A closed chord satisfies $\oint d\phi = 2\pi n$ ($n\in\mathbb{Z}$) along every non-contractible loop in the $(T_2,T_3)$ torus at fixed $T_1$. Its rest energy is the tension it refuses to release:
    \begin{equation}
    E_0 = \equiv \kappa^{-1} \bigl|\oint d\phi\bigr| = m c^2.
    \end{equation}

    \item \emph{Gravity.}  
    Around any closed chord the phase curves to remain single-valued, slowing the rate at which distant regions relax toward consonance. The Einstein's equations emerge as the coarse-grained relaxation dynamics
    \begin{equation}
    R_{\mu\nu} - \tfrac{1}{2} R g_{\mu\nu} = \frac{8\pi G}{c^4} \bigl\langle \Delta(\nabla\phi)^2 \bigr\rangle,
    \end{equation}
    with Newton’s constant fixed by the single constant $\kappa$:
    \begin{equation}
    G = \frac{\hbar \kappa}{c^5}.
    \end{equation}

    \item \emph{Quantum indeterminacy.}  
    Motion along $T_3$ is unobservable from the $T_1$. A mode delocalised across $T_3$ appears as superposition. Measurement forces irreversible relaxation of the $T_3$ circle, yielding Born’s rule as the uniform measure on $S^1$.

    \item \emph{Mind.}  
    A subsystem that stores and re-excites its own past phase configurations along $T_2$ acquires memory. When it further models its own future gradients, it becomes conscious—no extra ingredient required.
\end{enumerate}

And so we say good riddanvce to spacetime.