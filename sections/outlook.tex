%===================================================
\section{Open Problems and Next Steps}
\label{sec:outlook}
%===================================================

The cleaned outline clarifies the work still required to elevate the framework from suggestive geometry to tested science.

\begin{enumerate}
    \item \textbf{Balanced metrics:} derive the class of connections \(\Gamma^i{}_a\) and shear tensors \(\Xi_{ab}\) compatible with the zero-net-energy constraint and demonstrate how Einstein-like dynamics emerge from that restricted set.
    \item \textbf{Quantum mapping:} derive explicit correspondences between \(T_3\) spread and standard Hilbert-space amplitudes, including how decoherence scales with environmental coupling and how phase arises from braiding.
    \item \textbf{Gauge structure:} specify how internal symmetries arise from rotations in the normal temporal planes and what constraints they impose on particle-like modes, all without appealing to background space.
    \item \textbf{Experimental design:} compress the numerous speculative addenda into three executable protocols: one laboratory, one astrophysical, and one cognitive. Each requires quantitative predictions with error bars tied to the temporal parameters.
    \item \textbf{Narrative discipline:} keep non-technical explanations confined to an appendix so the main text remains a scientific argument.
\end{enumerate}

These steps replace the former sprawl of appendices with a finite agenda that can be tracked and either validated or refuted.
