\documentclass[12pt]{article}
\usepackage{amsmath, amssymb}
\usepackage{graphicx}
\usepackage{hyperref}
\usepackage{geometry}
\geometry{margin=1in}

\title{Temporal Harmonics and the Emergence of Space}

\author{Mark Lindholm}
\date{December 2025}

\begin{document}

\maketitle
\tableofcontents
\newpage

%===================================================
\section*{Abstract}
%===================================================
Time is three-dimensional. Within this temporal volume, self-reinforcing harmonic patterns arise. Particles, electromagnetic phenomena, gravitation, macroscopic bodies, and minds are all stable or evolving configurations of these harmonics. Three-dimensional space is not fundamental; it is the relational pattern of dissonance among simultaneous harmonics seen in successive cross-sections taken along the generative axis $T_1$. The theory offers a unified dynamics of nature and consciousness rooted solely in time and resonance.

\newpage

%===================================================
\section{Foundations}
%===================================================
\subsection{The Three Axes of Time}
\begin{itemize}
  \item $T_1$ \quad Generative Sequence — the direction of “happening”
  \item $T_2$ \quad Accumulated Structure — the axis of persistence, memory, habit
  \item $T_3$ \quad Modal Breadth — the axis of simultaneous alternatives, comparison, and ratio-making
\end{itemize}

\subsection{Temporal Volume}
The continuous product $T_1 \times T_2 \times T_3$.

\subsection{Tension}
Positive scalar measuring local unresolved dissonance. All dynamics is tension relaxation.

\subsection{Harmonic Mode}
A cyclic, self-reinforcing pattern primarily in the $T_2$–$T_3$ plane.

\subsection{Resonance / Dissonance / Modulation}
As in music, made precise in Appendix A.

\subsection{Quantum Superposition and Measurement}

A harmonic configuration sharply localised in $T_3$ sounds as a single definite note.  
A configuration extended in $T_3$ is a chord that has been struck off-centre and momentarily contains many neighbouring overtones — the state we call superposed.

Measurement is simply strong coupling between this spread-out chord is forced to ring against a much heavier, already perfectly tuned chord (the apparatus).  
The sudden dissonance across $T_3$ is intolerable to the universal drive toward consonance.  
In microseconds, tension relaxation does exactly what it does to any mistuned instrument: it damps the overtones and concentrates the vibrating energy into the one pure tone that best resolves the dissonance with the apparatus.

The generative tempo $T_1$ then continues reprinting only that pure tone.  
What we experience as “collapse” is nothing more than the ordinary settling of a musical instrument into its final, clear note.

No newcommand{\collapsequote}{the ordinary settling of a musical instrument into its final, clear note}

No extra postulate is required: the same principle that silences a buzzing piano string or aligns two detuned violins is the principle that produces definite outcomes in laboratories.

\newpage

%===================================================
\section{Core Vocabulary}
%===================================================
\begin{description}
  \item[Chord] Simultaneously resonant set of modes forming one identity.
  \item[Closed Chord] A chord whose resonances loop back on themselves, unable to disperse
  \item[Carrier] A pure circular mode that propagates pattern without persisting itself (photon-like).
  \item[Timbre] Distinctive overtone mixture of a chord.
  \item[Interval] Tension ratio between two modes.
  \item[Coherence Length] Extent in $T_2$ and $T_3$ over which phase relations are preserved.
\end{description}

Full glossary in Appendix A.

\newpage

%===================================================
\section{Emergence of Apparent Space}
%===================================================
At each instant $t \in T_1$ the entire $T_2$–$T_3$ plane forms one global chord.  
Successive global chords generate the experienced world.  
Perceived spatial separation = degree of dissonance between sub-chords.  
No background space exists; “space” is dissonance geometry within each slice.

\newpage

%===================================================
\section{Common Existents as Harmonic Configurations}
%===================================================
\subsection{Carriers (Photons)}
Pure circular modes in $T_2$–$T_3$. Frequency = rotation rate, polarization = plane of rotation.

\subsection{Mass-Bearing Closed Chords (Particles)}
Closed chords requiring continuous tension to remain closed. Inertia = resistance to forced reopening.

\subsection{Electromagnetism}
Electric and magnetic phenomena arise from traveling shear and twist in the $T_2$–$T_3$ plane:
\begin{itemize}
  \item Electric aspect: radial tension gradient around a closed chord.
  \item Magnetic aspect: rotational shear (circulating dissonance) around the same chord.
  \item Electromagnetic wave: helical carrier whose axis precesses, simultaneously carrying both shear and twist.
  \item Charge: topological winding number of the closed chord in $T_3$.
\end{itemize}

\subsection{Bodies}
Phase-locked orchestras of countless closed chords and carriers.

\subsection{Minds}
Self-modulating, high-coherence attractors able to explore and select among $T_3$ alternatives.

\subsection{Gravitation}
Ambient tension raised by massive closed chords distorts dissonance gradients in successive slices → experienced curvature of apparent space.

\newpage

%===================================================
\section{Dynamics and Mathematics}
%===================================================
State described by complex amplitude density $\psi(t,s,m)$ on the temporal volume.  
Tension functional (candidate):
\[
\mathcal{T}[\psi] = \int \left( |\partial_t \psi|^2 + |\nabla_{s,m} \psi|^2 + V(|\psi|^2) \right) \, dt\,ds\,dm
\]
with $V$ favoring rational-frequency ratios.  
Equations from $\delta \mathcal{T}=0$ support carriers, closed chords, and modulating attractors.

\newpage

%===================================================
\section{Cognition and Agency}
%===================================================
\subsection{Memory}
High-amplitude frozen modes along $T_2$.

\subsection{Rationality}
\begin{itemize}
  \item Ratio-identification: recognition of simple integer relations within $T_2$ structure.
  \item Ratio-making: active construction of new consonant relations between $T_2$ and $T_3$ branches.
\end{itemize}

\subsection{Emotion}
Global tension gradients felt as valence and pleasure.

\subsection{Free Will}
The capacity of a mind to choose which $T_3$ branch receives continued generative flow from $T_1$.

\newpage

%===================================================
\section{Testable Predictions — With Emphasis on Electromagnetism}
%===================================================
\begin{enumerate}
  \item \textbf{Polarization-dependent propagation delay in strong gravitational fields} (temporal lensing of light).
  \item \textbf{Violation of Malus’s law in extremely high-intensity fields} when closed chords begin to reopen, producing anomalous transmission through crossed polarizers.
  \item \textbf{Spontaneous vacuum birefringence detectable in cavity experiments} caused by transient closed-chord pair production aligning with cavity axis.
  \item \textbf{Decision-correlated electromagnetic micro-signals}: narrow-band emissions in the 1–50 Hz range from human brains immediately preceding voluntary action, predicted to show harmonic ratios absent in control periods.
  \item \textbf{Aharmonic resonance in superconducting cavities}: certain cavity geometries should exhibit unexplained persistent modes when driven at ratios derived from the golden mean or simple Pythagorean intervals, even when classical theory predicts damping.
\end{enumerate}

\newpage

%===================================================
\section{Open Problems}
%===================================================
\begin{itemize}
  \item Exact tension potential favoring musical intervals
  \item Complete topology of stable closed chords
  \item Quantitative mapping from $T_2$–$T_3$ dissonance to perceived metric
  \item Design of experiments 2–5 above
\end{itemize}

\newpage

%===================================================
\appendix
%===================================================
\section{Glossary of Temporal-Harmonic Terms}
\section{Detailed Tension Candidates}
\section{Topology of Closed Chords}
\section{Mathematics of the Spatial Illusion}
\section{Formal Model of Mind}
\section{Neurochemical Effects on $T_3$ Mobility}
\section{Psychological Interpretation of the Three Axes}

\end{document}