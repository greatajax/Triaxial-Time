\section{G. Conclusion}

We have sketched a framework in which physical time is not fundamentally
one-dimensional but instead has a three-dimensional structure, with axes
\((T_1, T_2, T_3)\) corresponding, roughly, to advance, persistence, and
modal exploration. The observed \(3+1\)-dimensional spacetime of standard
physics is treated as a membrane embedded in this extended temporal manifold.

Within this picture:

\begin{itemize}
    \item Special relativity emerges in the limit of a flat, unsheared
    membrane with a constant effective time direction.
    \item General relativity is recovered as the effective dynamics of the
    induced metric on the membrane, with extrinsic curvature supplying
    potential corrections.
    \item Quantum phenomena are interpreted in terms of distributions along
    the modal axis \(T_3\), with measurement-like processes corresponding to
    geometric localization in that direction.
    \item Mental and intentional phenomena are modeled as extended
    mode-structures in the temporal manifold, offering a way to reconcile
    free will and thick temporal experience with a causally closed physics.
\end{itemize}

The proposal is intentionally conservative in mathematics and aggressive in
ontology. It does not introduce exotic nonlocal operators, non-computable
dynamics, or violations of established symmetries at accessible scales. It
does, however, relocate several interpretative burdens---quantum measurement,
the arrow(s) of time, the status of possibility, and the nature of mind---into
the geometry and dynamics of a higher-dimensional temporal structure.

Much remains to be done. The most pressing tasks are:
\begin{enumerate}
    \item To develop explicit, solvable models that interpolate between
    standard relativistic field theories and genuinely new triaxial temporal
    behavior.
    \item To identify concrete observational signatures, especially in strong
    gravity, cosmology, and macroscopic quantum systems.
    \item To integrate the framework with realistic models of neural and
    cognitive dynamics, testing whether the temporal ontology clarifies or
    merely reframes existing puzzles.
\end{enumerate}

Whether or not the triaxial view of time proves empirically correct, it
demonstrates that the apparent thinness of time in current physical theory is
not forced by mathematics alone. One can formulate a higher-dimensional
temporal ontology that is both mathematically legal and empirically
compatible, while giving conceptual structure to phenomena that otherwise sit
at the edge of physics. That alone suggests that the interior of duration
remains an open frontier for theoretical exploration.
