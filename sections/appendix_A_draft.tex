# **Appendix A (continued): Interacting Fields and Fermionic Models on (\mathcal{M}^{1,3} \times \mathcal{T}^3)**

Having established the free scalar-field example, we now generalize the construction to include:

1. **self-interacting scalar fields**,
2. **Yukawa-type couplings**,
3. **fermionic fields**,
4. **gauge interactions acting through internal isometries**.

These explicit examples demonstrate that the internal-time manifold supports coherent quantum field theory without modification to the Standard Model at low energies.

---

# **A.5 Interacting Scalar Toy Model**

Consider the interacting scalar Lagrangian on the product manifold:

[
\mathcal{L} =
-\frac12 g^{\mu\nu}\partial_\mu\Phi\partial_\nu\Phi
-\frac12 \delta^{ab}\partial_a\Phi\partial_b\Phi
-\frac12 m_0^2 \Phi^2
-\frac{\lambda}{4!}\Phi^4.
]

Expand in internal Fourier modes:

[
\Phi(x,\tau)
= \sum_{\vec{n}} \phi_{\vec{n}}(x), e^{i\vec{n}\cdot\vec{\tau}}.
]

Plugging in and integrating over (\mathbb{T}^3), the quadratic part generates the familiar KK mass tower, while the interaction term produces:

[
\int d^3\tau, \Phi^4
====================

\sum_{\vec{n}_1+\vec{n}_2+\vec{n}_3+\vec{n}*4=0}
\phi*{\vec{n}*1}\phi*{\vec{n}*2}\phi*{\vec{n}*3}\phi*{\vec{n}_4}.
]

### **Selection Rule**

Only mode combinations whose internal momenta sum to zero can interact at tree level:

[
\vec{n}_1 + \vec{n}_2 + \vec{n}_3 + \vec{n}_4 = 0.
]

This conservation law is a consequence of the internal-space translational symmetry. It mimics internal-momentum conservation in Kaluza–Klein theory, reinforcing the physicality of the internal manifold.

### **Low-Energy Effective Theory**

At energies (E \ll R_a^{-1}), only the zero mode (\phi_{\vec{0}}(x)) is relevant, because all other modes are too massive to be on-shell.

Thus the low-energy effective Lagrangian is exactly:

[
\mathcal{L}_{\mathrm{eff}}
==========================

-\frac12 \partial_\mu\phi_0 \partial^\mu\phi_0
-\frac12 m_0^2 \phi_0^2
-\frac{\lambda}{4!} \phi_0^4,
]

identical to the usual 4D interacting scalar theory.

---

# **A.6 Fermionic Field on the Internal-Time Manifold**

Consider a Dirac field (\Psi(x,\tau)). The action is:

[
S = \int d^4x, d^3\tau,
\bar{\Psi}(x,\tau)
\left[
i\gamma^\mu \partial_\mu
+\gamma^5 \gamma^a \partial_a

* m
  \right]\Psi(x,\tau).
  ]

Here:

* (\gamma^\mu) are the usual 4D Dirac matrices,
* (\gamma^a) are three additional gamma matrices acting on the internal manifold,
* (\gamma^5) ensures the internal derivatives couple chirally, preserving Lorentz symmetry on (\mathcal{M}^{1,3}).

---

## **A.6.1 Internal Mode Expansion**

Expand:

[
\Psi(x,\tau) =
\sum_{\vec{n}}
\psi_{\vec{n}}(x),
e^{i \vec{n}\cdot\vec{\tau}}.
]

Substituting into the action yields:

[
S = \sum_{\vec{n}} \int d^4x;
\bar{\psi}*{\vec{n}}(x)
\left(
i\gamma^\mu \partial*\mu - m - \gamma^5 \gamma^a \frac{n_a}{R_a}
\right)
\psi_{\vec{n}}(x).
]

Thus each internal mode acquires an effective mass:

[
m_{\vec{n}}^2
=============

m^2 + \sum_{a=1}^{3} (n_a/R_a)^2.
]

This reproduces the KK tower for fermions.

---

## **A.6.2 Chirality and Internal Motion**

Internal momenta appear in the term:

[
\gamma^5 \gamma^a \frac{n_a}{R_a},
]

which mixes chirality with internal direction.
However, at low energies, only the zero mode survives, and the theory reduces to ordinary 4D Dirac fermions.

No chiral anomalies arise because:

* the internal manifold is flat and compact,
* the internal derivative term preserves gauge invariance,
* the internal spectrum is vector-like at the KK level.

Thus the internal-time structure is compatible with the Standard Model’s chiral fermions.

---

# **A.7 Yukawa Couplings**

Take:

* a fermion (\Psi(x,\tau)),
* a scalar (\Phi(x,\tau)).

The Yukawa interaction is:

[
S_Y =
-\int d^4x, d^3\tau;
y , \bar{\Psi}, \Phi, \Psi.
]

Expanding in modes:

[
\bar{\Psi}\Phi\Psi
==================

\sum_{\vec{n}_1,\vec{n}_2,\vec{n}*3}
\bar{\psi}*{\vec{n}*1}
\phi*{\vec{n}*2}
\psi*{\vec{n}_3}
;e^{i(\vec{n}_1+\vec{n}_2+\vec{n}_3)\cdot\vec{\tau}}.
]

Integrating over internal space yields:

[
S_Y =
-y\sum_{\vec{n}_1+\vec{n}_2+\vec{n}*3=0}
\int d^4x;
\bar{\psi}*{\vec{n}*1}
\phi*{\vec{n}*2}
\psi*{\vec{n}_3}.
]

This is consistent with standard KK selection rules and indicates that the Yukawa sector is well-behaved.

At low energies, only the (\vec{0}) modes survive:

[
S_Y^{\mathrm{eff}}
= -y \int d^4x;
\bar{\psi}_0 \phi_0 \psi_0,
]

identical to the ordinary 4D Yukawa interaction.

---

# **A.8 Gauge-Coupled Toy Model**

Gauge fields arise from internal isometries (Section 8).
To show the consistency with matter fields, take a U(1) rotation in the ((\tau^2,\tau^3))-plane:

[
\Psi(x,\tau) \rightarrow
e^{iq\alpha} \Psi(x, R(\alpha)\tau).
]

Introduce the covariant derivative:

[
D_\mu \Psi =
\partial_\mu \Psi

* iq A_\mu L \Psi,
  ]

where (L) generates rotations on (\mathcal{T}^3).

The gauge-invariant action is:

[
S = \int d^4x, d^3\tau;
\bar{\Psi}
(i\gamma^\mu D_\mu + \gamma^5\gamma^a \partial_a - m)\Psi
-\frac14 F_{\mu\nu}F^{\mu\nu}.
]

Expanding in internal modes:

* the zero mode couples exactly like a standard charged fermion,
* higher modes yield heavy charged fermions with masses (\ge 10!-!100~\mathrm{TeV}).

Thus the gauge sector remains consistent and invisible at accessible energies.

---

# **A.9 Summary of Appendix A**

The toy models in this appendix demonstrate:

1. The internal manifold supports free and interacting scalar fields.
2. Fermionic fields couple consistently to internal derivatives.
3. Yukawa interactions generalize naturally.
4. Gauge interactions integrate cleanly with internal-time dynamics.
5. The low-energy limit reduces exactly to the Standard Model.
6. All deviations are suppressed by the KK scale.
7. The framework fits seamlessly into familiar QFT techniques.