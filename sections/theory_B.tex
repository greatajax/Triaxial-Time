\section{B. Dynamics, Modes, and Shear}

\subsection{Field modes as temporal wavepackets}

Let \(\phi(\mathbf{x}, \bm{T})\) denote a generic field defined on the extended
temporal space. For concreteness we can consider a scalar field, but the
construction generalizes. We assume that physical excitations correspond to
\emph{mode-structures} that are localized in the normal directions of the
temporal space and propagate along the membrane.

In a local frame adapted to the membrane, we split the temporal coordinates
into one tangential parameter \(\tau\) and two normal coordinates
\(n^A\), \(A=1,2\), so that
\begin{equation}
    \bm{T} = \bm{T}(\tau, n^A).
\end{equation}
A physical mode is then characterized by a profile
\begin{equation}
    \phi(\mathbf{x}, \bm{T}) \approx \psi(\mathbf{x}, \tau)\,\chi(n^A),
\end{equation}
where \(\chi(n^A)\) is strongly localized around \(n^A=0\). Integrating out
the normal directions yields an effective four-dimensional field
\(\psi(\mathbf{x}, \tau)\) on the membrane.

The dynamics of \(\phi\) in the embedding space induces effective dynamics
for \(\psi\). A simple illustrative Lagrangian is
\begin{equation}
    \mathcal{L} = \frac{1}{2}\left(
        \partial_i \phi \,\partial^i \phi
        - c^2\,\delta^{ab}\,\partial_a \phi \,\partial_b \phi
        - m^2 \phi^2
    \right),
\end{equation}
where indices \(i\) run over spatial coordinates and indices \(a\) over the
three temporal directions. Upon mode decomposition and projection we obtain
an effective action for \(\psi\) of the familiar Klein--Gordon type, but with
corrections governed by the shape and dynamics of \(\chi\).

\subsection{Shear as an effective interaction}

Consider now how variations in the orientation of the membrane and the
effective time direction affect the projected dynamics. Let
\(U^a(\mathbf{x}, \tau)\) be the unit vector tangent to the membrane in the
temporal sector. The derivative \(\nabla_i U^a\) measures how the physical
time direction tilts as one moves in space.

A key observation is that the kinetic term in the temporal sector,
\(\delta^{ab}\partial_a\phi\,\partial_b\phi\), when written in the adapted
frame, splits into components along \(U^a\) and along the normal directions.
Tilting \(U^a\) relative to the fixed basis in \(\mathbb{R}^3_{\bm{T}}\)
introduces mixing terms between tangential and normal derivatives.

For localized modes, this mixing manifests as effective potentials and
gauge-like couplings in the projected dynamics of \(\psi\). Schematically,
one obtains terms of the form
\begin{equation}
    \mathcal{L}_\text{eff} \supset
    \frac{1}{2}(\partial_\tau \psi)^2
    - \frac{1}{2}(\nabla \psi)^2
    - V_\text{geom}(\mathbf{x}, \tau)\,\psi^2
    - A^\mu(\mathbf{x}, \tau)\,\psi\,\partial_\mu \psi
    + \dots
\end{equation}
where the ``geometric potential'' \(V_\text{geom}\) and effective vector
field \(A^\mu\) are functionals of \(\nabla_i U^a\) and related quantities.

This is where shear enters the picture: spatial gradients in the orientation
of the effective time direction play a dynamical role akin to gravitational
and gauge fields. In rough terms:

\begin{itemize}
    \item Slow, large-scale variations of \(U^a\), smooth over many mode
    wavelengths, reproduce gravitational effects via curvature of the
    induced metric on \(\Sigma\).
    \item Faster, more localized variations, especially those with internal
    symmetries in the temporal sector, can behave like gauge fields acting
    on internal degrees of freedom of the modes.
\end{itemize}

\subsection{Stable modes and quantization}

The three-dimensional temporal sector provides a natural setting for
quantization via normal-mode analysis. The normal directions \(n^A\) and the
tangential direction \(\tau\) support standing wave solutions \(\chi(n^A)\)
with discrete spectra. Each such normal mode defines a tower of effective
four-dimensional fields \(\psi_k(\mathbf{x}, \tau)\) with characteristic
masses and couplings.

In simple backgrounds one can expand
\begin{equation}
    \phi(\mathbf{x}, \bm{T}) =
    \sum_k \psi_k(\mathbf{x}, \tau)\,\chi_k(n^A),
\end{equation}
and carry the usual quantization procedures for the fields \(\psi_k\). The
three-dimensional time picture adds structure primarily in how the background
geometry and shear determine the spectrum \(\chi_k\) and the coupling of
modes to each other.

\subsection{Causality and consistency}

Because \(\mathcal{E}\) is equipped with a product-like metric with a
single distinguished causal direction induced on the membrane, one can
enforce causal consistency with respect to \(\tau\), even though the
underlying temporal space is three-dimensional.

The key constraints are:

\begin{enumerate}
    \item The membrane \(\Sigma\) must admit a global time function \(\tau\)
    with respect to the induced metric, ensuring no closed timelike curves
    in the effective physical spacetime.
    \item Dynamics in the normal directions must be such that changes in
    mode-structure at a point depend only on data in its past light-cone
    along \(\tau\).
\end{enumerate}

Under these conditions, the triaxial temporal ontology does not introduce
new causal paradoxes beyond those already present---and tightly controlled---
in standard relativistic field theories.
