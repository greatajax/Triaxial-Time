\section{B. Dynamics, Modes, and Shear}
\label{sec:dynamics}

\subsection{Field modes as temporal wavepackets}

Let \(\phi(\bm{T})\) denote a generic field defined on the tri-temporal manifold. A physical excitation is a localized braid with amplitude concentrated near a path \(\bm{T}(\tau)\) tangent to \(U^a\). Its profile can be factorized as
\begin{equation}
    \phi(\bm{T}) \approx \psi(\tau)\,\chi(n^A),
\end{equation}
where \(\tau\) parameterizes motion along \(U^a\) and \(n^A\) are coordinates in the two-dimensional normal plane. The zero-net-energy condition demands that the action carried by \(\psi\) and by the transverse shape \(\chi\) cancel when integrated over \(T_1, T_2, T_3\), preventing hidden energy sources.

A representative Lagrangian in the purely temporal setting is
\begin{equation}
    \mathcal{L} = \frac{1}{2}\left( (\partial_\tau \phi)^2 - \delta^{AB}\partial_A \phi\,\partial_B \phi - m^2 \phi^2 \right),
\end{equation}
which yields, after integrating out the normal directions, an effective equation for \(\psi\) that mirrors Klein--Gordon dynamics without presuming space.

\subsection{Shear as an effective interaction}

Variations in \(U^a\) across \(\mathcal{T}\) introduce mixing between motion along the braid and motion in the normal plane. Writing
\begin{equation}
    \Xi_{ab} = \nabla_a U_b - \nabla_b U_a,
\end{equation}
one finds corrections to the effective Lagrangian of the form
\begin{equation}
    \mathcal{L}_\text{eff} \supset - V_\text{geom}(\tau)\,\psi^2 - A(\tau)\,\psi\,\partial_\tau \psi + \cdots,
\end{equation}
where \(V_\text{geom}\) and \(A\) are functionals of \(\Xi_{ab}\). Smooth components give rise to gravitational effects; structured, oscillatory components behave like gauge couplings. All contributions respect the zero-net-energy constraint because \(\Xi_{ab}\) is derived from the balanced temporal flow.

\subsection{Stable modes and quantization}

Standing waves in the normal plane yield discrete mode families \(\chi_k(n^A)\) and corresponding effective fields \(\psi_k(\tau)\). Quantization proceeds as usual on the \(\psi_k\), but the spectrum and coupling structure are fixed by the temporal geometry rather than by an assumed spatial lattice. The absence of space does not remove locality; it reframes locality as limited overlap of braids within \(\mathcal{T}\).

\subsection{Causality and consistency}

Causality is enforced with respect to the monotonic parameter \(\tau\) along \(U^a\). No closed causal loops are allowed in \(\mathcal{T}\), and changes in \(\psi_k\) depend only on data in their past along \(\tau\). Emergent lightcones in the projected description arise from the connection \(\Gamma^i{}_a\) defined in Section~\ref{sec:geometry}; their stability traces back to the balanced flow condition and the bounded shear encoded in \(\Xi_{ab}\).
