%===================================================
\section{Framework Overview}
\label{sec:framework}
%===================================================

The primitive arena is entirely temporal: three orthogonal axes \(\bm{T} = (T_1, T_2, T_3)\) with no background spatial manifold. Observed spatial order is an emergent index built from how neighboring temporal volumes braid and maintain balance. The condition of zero net flow,
\begin{equation}
    \partial_{T_1} \mathcal{S}_1 + \partial_{T_2} \mathcal{S}_2 + \partial_{T_3} \mathcal{S}_3 = 0,
\end{equation}
for suitably defined action densities \(\mathcal{S}_a\), enforces that energy injected along one axis must be repaid along the others. This balance is the backbone that prevents hidden perpetual motion and anchors conservation laws once the theory is projected into familiar language.

The three axes are interpreted functionally:
\begin{description}
    \item[\(T_1\)] generative advance: the ordered succession of events.
    \item[\(T_2\)] persistence: the thickness that stabilizes structure and memory.
    \item[\(T_3\)] modal breadth: the spread across possibilities that quantum theory currently encodes as phase and amplitude.
\end{description}

Physical time is the direction \(U^a\) chosen by the local braid that respects the zero-net-energy constraint. Apparent space records how bundles of braids remain coherent across many events; curvature and shear in \(U^a\) become, after projection, the fields we currently call gravitational or gauge.

The role of the old appendices is recast as follows:
\begin{itemize}
    \item Descriptive essays (formerly addenda) become case studies showing how \(T_1\), \(T_2\), and \(T_3\) trade off in lived phenomena.
    \item Experimental sketches are collected into a dedicated appendix for future validation instead of scattered anecdotes.
    \item Popular expositions, such as the cognition and pharmacology notes, are cited as phenomenological anchors rather than parallel narratives.
\end{itemize}

This overview replaces the earlier sprawling introductions with a concise stage on which the subsequent technical sections operate.
