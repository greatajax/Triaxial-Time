\section{E. Mind, Information, and Extended Temporal Structure}

\subsection{Minds as extended regions in temporal space}

In ordinary neuroscience and physics, a mind is modeled as a dynamical
pattern in the brain over time. In the present framework, a mind is modeled
as a \emph{temporally extended region} of the membrane and its neighborhood
in the temporal sector.

Informally, one can picture:

\begin{itemize}
    \item A three-dimensional spatial region corresponding to the organism.
    \item A thick ``trail'' along \(T_2\) encoding persistent records and
    synaptic structures: memory as literal extension in the persistence axis.
    \item A cloud-like spread along \(T_3\) corresponding to contemplated
    alternatives, simulations, and counterfactuals: imagination and
    deliberation as extension along the modal axis.
\end{itemize}

The effective physical time direction \(U^a\) in and near such a region is
then shaped by the interplay of these extensions. The mind is not an
instantaneous point but a coherent, self-maintaining mode-structure occupying
finite volume in the three-dimensional temporal manifold.

\subsection{Free will as temporal self-organization}

In this picture, free will is not a primitive, unexplained power, nor is it
an illusion. Rather, it is a specific type of self-organization in the
temporal manifold:

\begin{quote}
A system exhibits free will to the extent that it can stably reshape the
orientation and extension of its own mode-structure along \(T_2\) and \(T_3\)
in a way that is causally consistent along \(T_1\) but not reducible to a
fixed law acting on instantaneous states.
\end{quote}

This is compatible with the dynamical laws being local and deterministic in
the full embedding space \(\mathcal{E}\). The crucial point is that the
state of a mind at a given \((\mathbf{x}, \bm{T})\) includes, as intrinsic
data, its extended structure along \(T_2\) and \(T_3\). That extended
structure embodies past commitments and future evaluations, which in turn
shape the local effective dynamics.

Nothing in this reconstruction violates causal closure along \(T_1\). It
simply recognizes that the causal domain of a mental event is not a thin
slice at a single physical time but a thickened region across the three
temporal axes.

\subsection{Information and records}

Records, memories, and informational structures correspond to patterns that
are extended primarily along \(T_2\). For example:

\begin{itemize}
    \item A written page is a physical system whose field configuration is
    relatively invariant under translations along \(T_2\), up to slow
    degradation.
    \item A neural memory trace is a mode-structure in the brain that
    maintains its shape along \(T_2\) while participating in ongoing
    dynamics along \(T_1\).
\end{itemize}

The robustness of a record correlates with the degree to which its associated
modes are extended and stabilized along the persistence axis. This provides a
geometric underpinning for the intuitive idea that information is that which
\emph{can be carried forward} through time.

\subsection{Modal exploration and deliberation}

Deliberation---considering alternative actions, simulations, or outcomes---is
modeled as temporary expansion of certain structures along \(T_3\). A mind
``explores'' branches in the modal axis, propagating virtual dynamics there
while remaining anchored to a single realized trajectory on the membrane.

When a decision is made, the extended modal structure contracts or reorients,
effectively selecting a narrower band of \(T_3\) consistent with the chosen
course. This is not a literal branching of physical worlds; rather, it is a
reconfiguration of mode-structure in the extended temporal manifold that
leaves a trace along \(T_2\) (a memory of the decision and its rationale).

A full theory of such processes would require coupling the triaxial temporal
geometry to models of neural dynamics. The present section serves only to
indicate how the framework admits a place for mental and intentional
phenomena without abandoning physical causality.
